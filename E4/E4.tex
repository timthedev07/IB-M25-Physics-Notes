\documentclass[a4paper,12pt]{article}
\usepackage{setspace}
\usepackage{sectsty}
\usepackage{siunitx}
\usepackage{graphicx}
\usepackage[a4paper, total={3in, 9in}, textwidth=16cm,bottom=1in,top=1.4in]{geometry}
\usepackage[dvipsnames]{xcolor}
\usepackage{amsmath}
\usepackage{esvect}
\usepackage{soul}
\usepackage{amsthm}
\usepackage{hyperref}
\usepackage{longtable}
\usepackage{float}
\usepackage{amssymb}
\usepackage{outlines}
\usepackage{caption}
\usepackage{fancyvrb}
\usepackage{subcaption}
\usepackage{esdiff}
\usepackage{colortbl}
\usepackage{booktabs}
\usepackage{setspace}
\usepackage{mathtools}
\usepackage{tikz,pgfplots}
\usepackage[most]{tcolorbox}
\usepackage{draftwatermark}
\usepackage{helvet}
\renewcommand{\familydefault}{\sfdefault}
\SetWatermarkText{timthedev07}
\SetWatermarkScale{4}
\SetWatermarkColor[gray]{0.97}
\usetikzlibrary{positioning,decorations.markings,arrows.meta}
\DeclarePairedDelimiter{\ceil}{\lceil}{\rceil}
\newtheorem{lemma}{Lemma}
\newtheorem{proposition}{Proposition}
\newtheorem{remark}{Remark}
\newtheorem{observation}{Observation}
\doublespacing
\let\oldsection\section
\renewcommand\section{\clearpage\oldsection}
\newcommand{\RNum}[1]{\uppercase\expandafter{\romannumeral #1\relax}}
\let\oldsi\si
\renewcommand{\si}[1]{\oldsi[per-mode=reciprocal-positive-first]{#1}}
\usepackage{enumitem}
\newcommand{\subtitle}[1]{%
  \posttitle{%
    \par\end{center}
    \begin{center}\large#1\end{center}
    \vskip0.5em}%
}
\newcommand{\degsym}{^{\circ}}
\newcommand{\Mod}[1]{\ (\mathrm{mod}\ #1)}
\usepackage{hyperref}
\hypersetup{
  colorlinks=true,
  linkcolor = blue
}
\newcommand{\lb}{\\[8pt]}
\newenvironment*{cell}[1][]{\begin{tabular}[c]{@{}c@{}}}{\end{tabular}}
\newcommand{\img}[4]{\begin{center}
  \begin{figure}[H]
    \centering
    \includegraphics[width=#2\textwidth]{#1}
    \caption{#3}
    \label{fig:#4}
  \end{figure}
\end{center}}
\parindent=0pt
\usepackage{fancyhdr}
\fancyfoot{}
\fancypagestyle{fancy}{\fancyfoot[R]{\vspace*{1.5\baselineskip}\thepage}}
\renewcommand{\contentsname}{Table of Contents}
\newcommand{\angled}[1]{\langle{#1}\rangle}
\newcommand{\paren}[1]{\left(#1\right)}
\newcommand{\sqb}[1]{\left[#1\right]}
\newcommand{\coord}[3]{\angled{#1,\, #2,\, #3}}
\newcommand{\pair}[2]{\paren{#1,\, #2}}
\newcommand{\atom}[3]{{}^{#1}_{#2}\text{#3}}
\usepackage[
  noabbrev,
  capitalise,
  nameinlink,
]{cleveref}

\crefname{lemma}{Lemma}{Lemmas}
\crefname{proposition}{Proposition}{Propositions}
\crefname{remark}{Remark}{Remarks}
\crefname{observation}{Observation}{Observations}

\newtcolorbox[auto counter]{prob}[2][]{fonttitle=\bfseries, title=\strut Problem~\thetcbcounter: #2,#1,colback=Orchid!5!white,colframe=Orchid!75!black,top=5mm,bottom=5mm}

\newtcolorbox[auto counter]{rem}[1][]{fonttitle=\bfseries, title=\strut Remark.~\thetcbcounter,colback=purple!5!white,colframe=purple!65!gray,top=5mm,bottom=5mm}

\newtcolorbox[auto counter]{defin}[1][]{fonttitle=\bfseries, title=\strut Definition.~\thetcbcounter,colback=black!5!white,colframe=black!65!gray,top=5mm,bottom=5mm}

\newtcolorbox[auto counter]{obs}[1][]{fonttitle=\bfseries, title=\strut Observation.~\thetcbcounter,colback=RedViolet!5!white,colframe=RedViolet!65!gray,top=5mm,bottom=5mm}

\newtcolorbox[auto counter]{lem}[1][]{fonttitle=\bfseries, title=\strut Lemma.~\thetcbcounter,colback=Maroon!5!white,colframe=Maroon!65!gray,top=5mm,bottom=5mm}

\newtcolorbox[auto counter]{prop}[1][]{fonttitle=\bfseries, title=\strut Proposition.~\thetcbcounter,colback=RedOrange!5!white,colframe=RedOrange!65!gray,top=5mm,bottom=5mm}

\newtcolorbox[auto counter]{hint}[1][]{fonttitle=\bfseries, title=\strut Hint.~\thetcbcounter,colback=OliveGreen!5!white,colframe=OliveGreen!75!gray,top=5mm,bottom=5mm}

\setlength{\belowcaptionskip}{-20pt}
\begin{document}


\pagenumbering{arabic}
\pagestyle{fancy}


\begin{titlepage}
  \begin{center}

    \vspace*{8cm}
    \textbf{\Large {IB Physics Topic E4 Nuclear Fission; HL}} \\
    \vspace*{1cm}
    \large{By timthedev07, M25 Cohort}

  \end{center}
\end{titlepage}

\pagebreak
\tableofcontents
\pagebreak

\clearpage
\setcounter{page}{1}
\addtocontents{toc}{\protect\thispagestyle{empty}}

\section{Induced and Spontaneous Fission}

Nuclear fission is the splitting of a heavier parent nucleus (with a nucleon number greater than about 230) splits into two or more nuclei. It can occur in two ways:
\begin{enumerate}
  \item \textbf{Spontaneous} fission: An extremely rare type of fission that occurs \textbf{without any external energy or particles supplied into the system}. This is often naturally occurring in nuclei such as those of thorium-232, uranium-235, and uranium-238. The half life for spontaneous fission is typically around 11 billion years.
  \item \textbf{Neutron-induced} fission: A neutron interacts with a heavy nuclide to produce an unstable isotope of the original, with a neutron number one greater.
\end{enumerate}

The similarities are
\begin{itemize}
  \item Both release energy.
  \item Both have a large number of possible pairs of products
\end{itemize}

\section{The Fission Mechanism}

Two nuclear fuels commonly used nowadays are uranium-235 and plutonium-239.
\begin{itemize}
  \item Uranium occurs naturally.
  \item Plutonium are produced in the rocks when natural uranium-238 captures a neutron from the two possible origins below
        \begin{enumerate}
          \item Cosmic rays
          \item A natural uranium-235 decay.
        \end{enumerate}
\end{itemize}

A typical example of an induced uranium-235 is as follows
$$\underbrace{\atom{235}{92}{U} + \atom{1}{0}{n}}_\text{\makebox[0pt]{\footnotesize{interaction with neutron to form unstable isotope}}} \rightarrow \atom{144}{56}{Ba} + \atom{90}{36}{Kr} + 2\atom{1}{0}{n} + \Delta E$$
\begin{itemize}
  \item The Ba and Kr are the fission products, both have lighter nuclei.
  \item The example above shows the emission of two electrons, but the number of emitted neutrons can vary. It is typically 2-3 neutrons.
  \item The energy released $\Delta E$ is given by the mass defect between the two sides of the equation $$\Delta E = \paren{m_{\text{U-235}} - \paren{m_\text{Ba-144} + m_\text{Kr-90} + m_{n}}}c^2$$
        where $m_{\text{initial}}$ is the mass of the initial nucleus, and $m_{\text{final}}$ is the mass of the final nucleus.
\end{itemize}

In a fission process, the \hl{principal energy change} is mass-energy of uranium into kinetic energy of fission products.

\pagebreak

\subsection{Fission Products --- Percentage Yield}

\img{u235g.png}{0.5}{\% Yield of fission products of uranium-235 by proton numbers}{u235g}

\begin{itemize}
  \item There are two peaks in the distribution, indicating that fission products tend to cluster around proton numbers 36-38 and 54-58. This means that when U-235 undergoes fission, the resulting nuclei are most likely to have proton numbers in these two ranges.
  \item The nuclei formed in these regions (36-38, 54-58) are more stable than U-235 itself. This is because the binding energy per nucleon (the energy that holds the nucleus together) is greater for these fission products than for U-235.
  \item The daughter nuclei, immediately upon fission, are close together and repel each other. The electrostatic repulsion provides the kinetic energy for the nuclei and particles to move in different directions. A portion of the energy is released through photon emission.
\end{itemize}

\section{Chain Reactions}

When one of the emitted neutrons from a fission event interacts with another nucleus to form a new unstable isotope, triggering a subsequent fission event, and this process carries on with more and more nuclei, we have a chain reaction.

the role of chain reactions in the operation of a nuclear power station is described as follows by the mark scheme:
\begin{itemize}
  \item The four neutrons released in the previous reactions may initiate further fissions
  \item If sufficient U-235 is available, the reaction is self-sustained
  \item Allowing for the continuous production of energy
  \item The number of neutrons available is controlled with control rods to maintain the desired reaction rate
\end{itemize}

\section{Fission Reactors}

The following is a thermal fission reactor, specifically, a pressurized water reactor (PWR) that uses uranium-235 as fuel. The aim of this reactor is to convert nuclear energy into electrical energy, by heating water to produce steam, which drives a turbine and a generator.

\img{reactor.png}{1}{A schematic of a pressurized water reactor}{reactor}

\begin{enumerate}
  \item The reactor's fuel is \textit{enriched} if its proportion of U-235 is boosted to a level greater than the natural 0.7\%.

  \item The enriched fuel is placed in fuel rods, which are then placed in the reactor core.

  \item The neutrons are required to be slowed down to increase the probability of interacting with a uranium nucleus to induce fission. This is done by a moderator, which is typically water or graphite. The typical accepted speed is around $\SI{2.2}{\km\per\s}$ with kinetic energy of $\SI{0.025}{\eV}$. Neutrons at this speed are called \textit{thermal neutrons}, as they would be in equilibrium with matter at about room temperatures.
\end{enumerate}

\subsection{Moderators}

Its aim is to remove the excess of KE from neutrons to facilitate the fission process. Typical moderators may include water, heavy water, and graphite.\lb
When fast-moving neutrons strike the atom within the moderator elastically, they transfer some of their KE to the atom, slowing down themselves. After a succession of such collisions, the neutrons become thermal neutrons.\lb
When a neutron of mass $m$ collides head-on with moderator nucleus of mass $M$, the loss in KE is
$$\Delta E_k = \frac{4mM}{(m+M)^2}E_k$$
Let $r = \dfrac{\Delta E_k}{E_k}$ denote the fraction of KE lost. For one collision with each of the common moderator nuclei
\begin{itemize}
  \item Hydrogen: $r \approx 1$ --- roughly all KE of the neutron is lost.
  \item Deuterium: $r \approx 0.9$; Carbon: $r \approx 0.3$
\end{itemize}
In reality, the number of collisions required to slow down a neutral by a certain factor is well below the theoretical prediction because not all predictions are head-on.\lb
A potential problem with moderators is that U-238 can absorb even fast-moving neutrons. To tackle this problem, the moderators are placed closer to the fuel rods. The neutrons move from the fuel rods to the moderator at random, and the probability of a neutron being absorbed by U-238 is reduced.\lb
Properties of a good moderator include
\begin{itemize}
  \item Poor absorber of neutrons
  \item Being inert (chemically inactive) in the extreme conditions of a reactor
\end{itemize}
Hydrogen absorbs a massive portion of KE, but it is a good absorber of neutrons; thus, it cannot be used directly as a moderator.

\subsection{Control Rods}

Handles the turning-on and shutting-down of the reactor. They are made of boron and good absorbers of thermal neutrons. When the rods are inserted into the reactor core, they absorb the neutrons, lowering the rate of fission. When the rods are removed, the rate of fission increases back up.

\subsection{Heat Exchangers}

The heat energy cannot be directly converted into electrical energy; e.g. the steam cannot be piped directly into the turbine through the reactor vessel because the steam would be radioactive. There must be a closed-system mechanism in place to handle the transformation of energy.
\begin{enumerate}
  \item A primary loop of water passes through the reactor core, absorbing the heat energy. This water is kept at high pressure to prevent boiling, although its temperature already exceeds the boiling point. It is then passed through a heat exchanger --- the steam generator.
  \item When the primary loop passes through the tubes in the steam generator, the heat is transferred to a secondary loop of water initially kept at a lower temperature. Throughout this process, the primary loop remains \textbf{separate from the secondary loop}, preventing the radioactive water from contaminating the secondary loop.
  \item This secondary loop of water is then heated to produce steam, which drives the turbine and generator.
\end{enumerate}
In short, the purpose of the heat exchanger is to
\begin{itemize}
  \item Collect thermal energy from the coolant and deliver it to the gas
  \item Prevent the irradiated coolant from leaving the reactor vessel
\end{itemize}

\pagebreak

\subsection{Shielding}

Safety measures need to be implemented to prevent the penetrative radiation from escaping the reactor. This is typically done in the following ways
\begin{itemize}
  \item The reactor vessel is made of \textbf{thick steel} to absorb alpha and beta radiation, together with some of the gamma photons as well as stray neutrons.
  \item The vessel is also surrounded by a thick \textbf{concrete shield} to further absorb escaping gamma photons and neutrons.
  \item The building is also constructed such that it can \textbf{withstand natural disasters} such as earthquakes and tsunamis. In the even of a disaster and destruction, the reactor would be trapped in this construction.
  \item Emergency safety mechanisms --- immediate operations that are triggered in the event of an issue to shut the reactor down.
  \item The insertion and withdrawal of control rods are operated by machines instead of humans to reduce radioactive exposure.
\end{itemize}

\pagebreak

\subsection{Nuclear Waste Handling}

This is an important aspect that must be considered when operating a nuclear reactor.

\begin{longtable}{|p{0.2\textwidth}|p{0.3\textwidth}|p{0.4\textwidth}|}\hline
  \rowcolor{TealBlue!100!black!20} Problem                                                                     & Descripion                                                                                                                                                                                                               & Mitigation Strategy                                                                                                                                     \\ \hline
  Useless daughter nuclei from fission                                                                         & As more and more fission events take place, there will be more and more daughter nuclei (e.g. krypton, barium, molybdenum, etc.); these are neutron-rich, will likely undergo $\beta^-$ decay, and have long half-lives. & Store the waste in a secure location, such as a deep underground facility.                                                                              \\ \hline
  Distortion of fuel rod's physical structure                                                                  & The creation of two new nuclei changes the structure of the fuel rod --- the smaller nuclei take up a different amount of space.
  In a worst-case scenario, this distortion could cause the fuel rod to jam in its channel inside the reactor. & Regularly removing the fission products from the rod and recycling the remaining U-235 into a new rod.                                                                                                                                                                                                                                                                             \\ \hline
  Spent fuel rods                                                                                              &                                                                                                                                                                                                                          & Reprocessing --- storing the rods in cooling underwater storage for five to ten years before undergoing treatment to recycle the uranium into new rods. \\ \hline
  Long half-life waste                                                                                         & Some of the waste products have half-lives of thousands of years, which means that they will remain radioactive for a long time. But their activity is relatively low, compared to short-half-life waste.                & Store the waste in a secure location, such as a deep underground vault or well-sealed drums.                                                            \\ \hline
  Low-level waste                                                                                              & This includes the gloves and over-shoes used by the workers as well as medical equipment, which are contaminated by the radiation.                                                                                       & Buried under secure conditions.                                                                                                                         \\ \hline
  End of life of the reactor                                                                                   & The reactor has a finite lifespan, and at the end of its life, it must be decommissioned.                                                                                                                                & The reactor is dismantled, and the radioactive parts are stored in a larger shell of concrete, which will be buried underground.                        \\ \hline
\end{longtable}

Spent nuclear fission fuel rods must be kept safe because
\begin{itemize}
  \item Have relatively short half-lives / high activity
  \item Their decay products are usually also radioactive
  \item Volatile / chemically active
  \item Biologically active / easily absorbed by living matter
\end{itemize}

\section{Exam Questions}

\subsection{Producing Waste}

Fuel rods in a nuclear fission reactor contain uranium isotopes U-235 and U-238. Which process taking place in the reactor contributes most significantly to the formation of radioactive waste products?
\begin{enumerate}[label=(\Alph*)]
  \item alpha decay of U-235
  \item alpha decay of U-238
  \item neutron-induced fission of U-235
  \item neutron-induced transmutation of U-238 to plutonium-239
\end{enumerate}
\begin{itemize}
  \item We must first realize the difference between U-235 and U-238. U-235 is the fuel, and U-238 is the fertile material. U-235 undergoes fission, while U-238 undergoes transmutation --- changing into Pu-239 which is a fissile material.
  \item Transmutation does not directly contribute to forming nuclear waste, thus we can eliminate D.
  \item Also, radioactive waste main arises from fission and not alpha decay, thus, we can eliminate A and B.
  \item We are now left with C.
\end{itemize}

\end{document}