\documentclass[a4paper,12pt]{article}
\usepackage{setspace}
\usepackage{sectsty}
\usepackage{siunitx}
\usepackage{graphicx}
\usepackage[a4paper, total={3in, 9in}, textwidth=16cm,bottom=1in,top=1.4in]{geometry}
\usepackage[dvipsnames]{xcolor}
\usepackage{amsmath}
\usepackage{esvect}
\usepackage{soul}
\usepackage{amsthm}
\usepackage{hyperref}
\usepackage{longtable}
\usepackage{float}
\usepackage{amssymb}
\usepackage{outlines}
\usepackage{caption}
\usepackage{fancyvrb}
\usepackage{subcaption}
\usepackage{esdiff}
\usepackage{dirtytalk}
\usepackage{colortbl}
\usepackage{booktabs}
\usepackage{setspace}
\usepackage{mathtools}
\usepackage{tikz,pgfplots}
\usepackage[most]{tcolorbox}
\usepackage{draftwatermark}
\usepackage{helvet}
\usepackage{dirtytalk}
\renewcommand{\familydefault}{\sfdefault}
\DeclareRobustCommand{\hlans}[1]{{\sethlcolor{ForestGreen!30!white}\hl{#1}}}
\sisetup{per-mode=reciprocal}

\SetWatermarkText{timthedev07}
\SetWatermarkScale{0.7}
\SetWatermarkColor[gray]{0.97}
\usetikzlibrary{positioning,decorations.markings,arrows.meta,angles,quotes}
\DeclareSIUnit{\rad}{rad}
\DeclarePairedDelimiter{\ceil}{\lceil}{\rceil}
\newtheorem{lemma}{Lemma}
\newtheorem{proposition}{Proposition}
\newtheorem{remark}{Remark}
\newtheorem{observation}{Observation}
\doublespacing
% section page break
\let\oldsection\section
\renewcommand\section{\clearpage\oldsection}
% subsection page break
\let\oldsubsection\subsection
\renewcommand\subsection{\clearpage\oldsubsection}

\newcommand{\RNum}[1]{\uppercase\expandafter{\romannumeral #1\relax}}
\let\oldsi\si
\renewcommand{\si}[1]{\oldsi[per-mode=reciprocal-positive-first]{#1}}
\usepackage{enumitem}
\newcommand{\subtitle}[1]{%
  \posttitle{%
    \par\end{center}
    \begin{center}\large#1\end{center}
    \vskip0.5em}%
}
\newcommand{\degsym}{^{\circ}}
\newcommand{\Mod}[1]{\ (\mathrm{mod}\ #1)}
\usepackage{hyperref}
\hypersetup{
  colorlinks=true,
  linkcolor = blue
}
\newcommand{\lb}{\\[8pt]}
\newenvironment*{cell}[1][]{\begin{tabular}[c]{@{}c@{}}}{\end{tabular}}
\newcommand{\img}[4]{\begin{center}
  \begin{figure}[H]
    \centering
    \includegraphics[width=#2\textwidth]{#1}
    \caption{#3}
    \label{fig:#4}
  \end{figure}
\end{center}}
\parindent=0pt
\usepackage{fancyhdr}
\fancyfoot{}
\fancypagestyle{fancy}{\fancyfoot[R]{\vspace*{1.5\baselineskip}\thepage}}
\renewcommand{\contentsname}{Table of Contents}
\newcommand{\ans}[1]{\textcolor{ForestGreen}{The answer is #1.}}
\newcommand{\out}[1]{\textcolor{BrickRed}{We rule out option #1}}
\newcommand{\outs}[2]{\textcolor{BrickRed}{We rule out options #1 and #2}}
\newcommand{\angled}[1]{\langle{#1}\rangle}
\newcommand{\paren}[1]{\left(#1\right)}
\newcommand{\sqb}[1]{\left[#1\right]}
\newcommand{\coord}[3]{\angled{#1,\, #2,\, #3}}
\newcommand{\pair}[2]{\paren{#1,\, #2}}
\newcommand{\atom}[3]{{}^{#1}_{#2}\text{#3}}
\usepackage[
  noabbrev,
  capitalise,
  nameinlink,
]{cleveref}

\crefname{lemma}{Lemma}{Lemmas}
\crefname{proposition}{Proposition}{Propositions}
\crefname{remark}{Remark}{Remarks}
\crefname{observation}{Observation}{Observations}

\newtcolorbox[auto counter]{prob}[2][]{fonttitle=\bfseries, title=\strut Problem~\thetcbcounter: #2,#1,colback=Orchid!5!white,colframe=Orchid!75!black,top=5mm,bottom=5mm}

\newtcolorbox[auto counter]{rem}[1][]{fonttitle=\bfseries, title=\strut Remark.~\thetcbcounter,colback=purple!5!white,colframe=purple!65!gray,top=5mm,bottom=5mm}

\newtcolorbox[auto counter]{defin}[1][]{fonttitle=\bfseries, title=\strut Definition.~\thetcbcounter,colback=black!5!white,colframe=black!65!gray,top=5mm,bottom=5mm}

\newtcolorbox[auto counter]{obs}[1][]{fonttitle=\bfseries, title=\strut Observation.~\thetcbcounter,colback=RedViolet!5!white,colframe=RedViolet!65!gray,top=5mm,bottom=5mm}

\newtcolorbox[auto counter]{lem}[1][]{fonttitle=\bfseries, title=\strut Lemma.~\thetcbcounter,colback=Maroon!5!white,colframe=Maroon!65!gray,top=5mm,bottom=5mm}

\newtcolorbox[auto counter]{prop}[1][]{fonttitle=\bfseries, title=\strut Proposition.~\thetcbcounter,colback=RedOrange!5!white,colframe=RedOrange!65!gray,top=5mm,bottom=5mm}

\newtcolorbox[auto counter]{hint}[1][]{fonttitle=\bfseries, title=\strut Hint.~\thetcbcounter,colback=OliveGreen!5!white,colframe=OliveGreen!75!gray,top=5mm,bottom=5mm}

\setlength{\belowcaptionskip}{-20pt}
\begin{document}


\pagenumbering{arabic}
\pagestyle{fancy}


\begin{titlepage}
  \begin{center}

    \vspace*{8cm}
    \textbf{\Large {IB Physics Data Based Question Skills; SL \& HL}} \\
    \vspace*{1cm}
    \large{By timthedev07, M25 Cohort}

  \end{center}
\end{titlepage}

\pagebreak
\tableofcontents
\pagebreak

\clearpage
\setcounter{page}{1}
\addtocontents{toc}{\protect\thispagestyle{empty}}

\section{Introduction}

This document is dedicated to walking through the set of past paper/RevisionVillage questions that cover the skills you will need for the data-based questions in the MCQ or Paper 1B.\lb
There will be no knowledge notes, instead it's purely focused on the questions. If this is what you are looking for, you will find a pdf from the Oxford Study Guide that has everything.\lb
When going through the specimen questions, it's a good idea to bear in mind that these are easier than what you should expect in your final exam.

\section{Specimen Paper 1B}

\oldsubsection{Question 1}

[Maximum mark: 12]

A group of students investigate the motion of a conducting ball suspended from a long string.
The ball is between two vertical metal plates that have an electric potential difference V
between them. The ball is touched to one plate so that it becomes electrically charged and
is repelled from the plate. For a given potential difference, the ball bounces between the
plates with a constant period.

\img{ex/1.png}{0.3}{A diagram of the experiment.}{1}

\begin{enumerate}[label=(\alph*)]
  \item The students vary $V$ and measure the time $T$ for the ball to move \textbf{once} from one plate to the other. The table shows some of the data.
        \begin{table}[H]
          \centering
          \begin{tabular}{|c|c|}
            \hline
            \textit{V} / kV & \textit{T} / s $\pm$ 0.1s \\ \hline
            3.00            & 1.4                       \\ \hline
            5.00            & 0.8                       \\ \hline
            7.00            & 0.6                       \\ \hline
          \end{tabular}
        \end{table}

        \begin{enumerate}[]
          \item $V$ is provided by two identical power supplies connected in series. The potential difference of each of the power supplies is known with an uncertainty of 0.01 kV.
                State the uncertainty in the potential difference V. \hfill [1]
                \begin{itemize}
                  \item Essentially, the p.d. we are talking about here is the total p.d. across the two power supplies. If each has a p.d. uncertainty of 0.01 kV, then the total p.d. uncertainty is \hlans{0.02 kV}.
                \end{itemize}
          \item $T$ is measured with an electronic stopwatch that measures to the nearest 0.1 s.
                Describe how an uncertainty in $T$ of less than 0.1 s can be achieved using this stopwatch. \hfill [2]
                \begin{itemize}
                  \item The key idea here is that we can think of $$T = \frac{\text{total time for $n$ bounces}}{n}$$
                  \item The uncertainty here is then $$\Delta T = \frac{\Delta (\Sigma t)}{n}$$
                  \item So the more bounces we measure, the smaller the uncertainty in $T$.
                  \item In conclusion: by measuring the time for many bounces and dividing the result by the number of bounces.
                \end{itemize}

                The graph shows the variation of T with V. The uncertainty in V is not plotted.

                \img{ex/2.png}{0.7}{The graph of T against V.}{2}
          \item Outline why it is unlikely that the relationship between $T$ and $V$ is linear. \hfill[1]
                \begin{itemize}
                  \item There is not a line of best fit that passes \hl{through all error bars}.

                \end{itemize}
          \item Calculate the largest fractional uncertainty in $T$ for these data. \hfill [2]
                \begin{itemize}
                  \item Recall that the fractional uncertainty is given by $$\frac{\Delta T}{T}$$
                  \item So, to pick the point that has the largest fractional uncertainty given that they all have the same absolute uncertainty $\Delta T$, we must use the point that has the smallest value of $T$.
                        $$\frac{\Delta T}{T} = \frac{0.1}{0.5} = 0.2$$
                \end{itemize}
        \end{enumerate}

  \item The students suggest the following theoretical relationship between $T$ and $V$:
        $$T = \frac{A}{V}$$
        where A is a constant.

        To verify the relationship, the variation of $T$ with $\dfrac{1}{V}$ is plotted.


        \img{ex/3.png}{0.7}{The graph of T against $\dfrac{1}{V}$.}{3}

        \begin{enumerate}[label=(\roman*)]
          \item Determine $A$ by drawing the line of best fit.\hfill [3]
                \begin{itemize}
                  \item First, plot the error bars so that you can verify your line of best fit at least passes through all the data points and their error margins.
                  \item The next mark is given for a taking a segment on the line at least half of it (e.g. from its start point to end point), and then calculating the gradient of that segment by drawing the triangle (to visualise the change in $y$ and $x$).
                  \item The third mark point is given for the gradient, $A$ correctly calculated, e.g.
                        $$A = \frac{1.6 - 0}{0.40 - 0} = 4$$
                \end{itemize}

          \item State the units of A. (Note that it doesn't say "base SI units", so we do not need a conversion to base SI units.) \hfill [1]
                $$V\times T \equiv \text{(kV)} \times \text{s} = \text{kV s}$$

          \item The theoretical relationship assumes that the ball is only affected by the electric force. Suggest why, in order to test the relationship, the length of the string should be much greater than the distance between the plates. \hfill [2]
                \begin{itemize}
                  \item The question is really asking for a change to the experiment that can ensure that the only force acting on the ball is the electric force, because it's a prerequisite of the theoretical relationship.
                  \item Now let's ask ourselves the following question: What are potentially other forces acting on the ball that would violate the prerequisite?
                  \item The ball is suspended by a string, so there is a tension force acting on the ball. Only the horizontal component of the tension force will affect the ball's motion, so let's think about ways to minmise that.
                  \item If we think about the geometry of the situation, the higher up the point of suspension is -- the smaller the angle of the string to the vertical -- the smaller the horizontal component of the tension force.
                  \item \hlans{Hence, we should make the string much longer than the distance between the plates, so that the angle of the string to the vertical is very small.}
                \end{itemize}
        \end{enumerate}
\end{enumerate}

\subsection{Question 2}

[Maximum mark: 8]

A group of students investigate the bending of a plastic ruler that is clamped horizontally at one end. A weight $W$ attached to the other end causes the ruler to bend. The weight is contained in a scale pan.\lb
The students fix the length $L$ of the ruler and vary $W$. For each value of $W$, the group measures the deflection $d$ of the end of the ruler to which the weight is attached.
\img{ex/4.png}{0.3}{A diagram of the experiment.}{4}
\begin{enumerate}[label=(\alph*)]
  \item The group obtains the following repeated readings for $d$ for \textbf{one} value of $W$.
        \begin{table}[H]
          \centering
          \begin{tabular}{|*{7}{c|}}
            \hline
            \textbf{Reading} & 1   & 2   & 3   & 4   & 5   & 6   \\ \hline
            \textbf{d} / cm  & 2.7 & 2.9 & 3.6 & 2.7 & 2.8 & 2.9 \\ \hline
          \end{tabular}
        \end{table}
        The group divides into two subgroups, A and B, to analyse the data.\lb
        Group A quotes the mean value of d as 2.93 cm.\lb
        Group B quotes the mean value of d as 2.8 cm.\lb
        Discuss the values that the groups have quoted. \hfill [2]
        \begin{itemize}
          \item Immediately, when reading the dataset, we can spot an outlier on the third reading, which is 3.6 cm. The fact that we are given two different mean values suggests that one of the groups has included the outlier in their calculation, and the other group has excluded it.
          \item \hlans{Group B has excluded the outlier in their calculation, and so their mean value 2.8 is a better representation of the data.}
          \item \hlans{Group A also has the problem that the number of significant figures is not suitable for the data. The data is given to 2 s.f., so the mean value should also be given to 2 s.f. (2.9 cm).}
        \end{itemize}

  \item The variation of $d$ with $W$ is shown.

        \img{ex/5.png}{0.7}{The graph of d against W.}{5}

        Outline one experimental reason why the graph does not go through the origin. [1]
        \begin{itemize}
          \item At the point where $W = 0$ we have a non-zero deflection because the scale pan itself has a weight, and so the ruler will bend downwards even when there is no additional weight on it.
          \item The mark scheme also brings up the idea of a systematic error. This is a good point, but I think it is a bit too vague to be a good answer.
        \end{itemize}
  \item Theory predicts that
        $$d \propto \frac{W^xL^y}{EI}$$
        where $E$ and $I$ are constants. The fundamental units of $I$ are $m^4$ and those of $E$ are $\si{\kg\per\m\per\s\squared}$.

        Calculate $x$ and $y$. \hfill [2]

        \begin{itemize}
          \item This is a question of dimensional analysis, i.e. equating the dimensions of both sides of the equation.
                \begin{itemize}
                  \item $W \equiv \si{\kg\m\per\s\squared}$
                  \item $L \equiv \si{\m}$
                \end{itemize}
          \item Then, we have the following eqution
                \begin{align*}
                  dEI                                                              & \propto {W^xL^y}                     \\
                  m\left(\si{\kg\per\m\per\s\squared}\right)\left(\si{\m}^4\right) & \equiv \si{\kg^x\m^{x}\s^{-2x}}(m^y) \\
                \end{align*}
          \item This looks messy, but let's do a \say{compare coefficients} here, and we should obtain a system of equations.
          \item If we look at the m powers, we have
                $$1 - 1 + 4 = x + y$$
          \item If we look at the kg powers
                $$1 = x$$
          \item So we have $x =1$ and $y = 3$.

        \end{itemize}

        \pagebreak
  \item The ruler has cross-sectional area $A = a\times b$, where $a = (28 \pm 1)$ mm and $b = (3.00 \pm 0.05)$ mm.

        \begin{enumerate}
          \item Suggest an appropriate measuring instrument for determining $b$. \hfill [1]
                \begin{itemize}
                  \item A micrometer screw gauge / Vernier caliper / travelling microscope is a good answer here, because it can measure to the nearest 0.05 mm.
                \end{itemize}
          \item Calculate the percentage uncertainty in the value of $A$.
                \begin{align*}
                  \frac{\Delta a}{a} & = \frac{1}{28}                            \\
                  \frac{\Delta b}{b} & = \frac{0.05}{3.00}                       \\
                  \frac{\Delta A}{A} & = \frac{\Delta a}{a} + \frac{\Delta b}{b} \\
                                     & = \frac{1}{28} + \frac{0.05}{3.00}        \\
                                     & = 0.05 = 5\%
                \end{align*}
        \end{enumerate}

\end{enumerate}

\section{Old Syllabus Questions}

These questions are taken from the old syllabus' past papers, but still have relevance.\lb
In the new syllabus, these questions appear in Paper 1B, and if you ever wonder where you can find them in the old papers, they are located in Paper 3 section A.

\subsection{Uncertainty in Repeated Readings}

A student measures the time for 20 oscillations of a pendulum. The experiment is repeated four times. The measurements are:

\begin{center}
  10.45 s

  10.30 s

  10.70 s

  10.55 s
\end{center}

What is the best estimate of the uncertainty in the average time for 20 oscillations?

\begin{enumerate}[label=\Alph*.]
  \item 0.01 s
  \item 0.05 s
  \item 0.2 s
  \item 0.5 s
\end{enumerate}

\begin{itemize}
  \item The ideas is to first find the average and then look for the smallest uncertainty that allows all values in the list to be covered.
  \item The average turns out to be 10.5
  \item the largest and smallest are 10.7 and 10.3 respectively
  \item An uncertainty of 0.2 s would cover all values.
  \item \ans{C}
  \item The pitfall is that some students may mistakenly divide anything by 20 to find the time/uncertainty per oscillation. This wrong, because the question asks for the average time for 20 oscillations, not per oscillation.
\end{itemize}

\subsection{Uncertainty in Changes}

The uncertainty in reading a laboratory thermometer is 0.5°C. The temperature of a liquid falls from 20°C to 10°C as measured by the thermometer. What is the percentage uncertainty in the change in temperature?

\begin{enumerate}[label=\Alph*.]
  \item 2.5\%
  \item 5\%
  \item 7.5\%
  \item 10\%
\end{enumerate}

\begin{itemize}
  \item The change in temperature is 10 degrees.
  \item The absolute uncertainty in the change is twice the uncertainty in the thermometer reading, i.e. 1 degree...
  \item which is 10\% of the change in temperature.
  \item \ans{D}
\end{itemize}

The logic is that when you find the difference between values each with abs. uncertainty $0.5$, the maximum error in the difference is $0.5 + 0.5 = 1$, i.e. $(20 + 0.5) - (10 - 0.5) = 10 + 1$, and the maximum negative error occurs when $(20 - 0.5) - (20 + 0.5) = 10 - 1$.

\subsection{Experiments (1)}

A student studies the relationship between the centripetal force applied to an object undergoing circular motion and its period \( T \).\lb
The object (mass \( m \)) is attached by a light inextensible string, through a tube, to a weight $W$ which hangs vertically. The string is free to move through the tube. A student swings the mass in a horizontal, circular path, adjusting the period $T$ of the motion until the radius \( r \) is constant. The radius of the circle and the mass of the object are measured and remain constant for the entire experiment.

\img{ex/6.png}{0.7}{A diagram of the experiment.}{6}


The student collects the measurements of $T$ five times, for weight \( W \). The weight is then doubled (\( 2W \)) and the data collection repeated. Then it is repeated with \( 3W \) and \( 4W \). The results are expected to support the relationship
\[ W = \frac{4\pi^2mr}{T^2}, \]

\begin{enumerate}[label=(\alph*)]
  \item State why the experiment is repeated with different values of $W$.
        \begin{itemize}
          \item to confirm the proportionality relationship between $W$ and $\dfrac{1}{T^2}$.

                \textbf{OR}

          \item because $W$ is the independent variable




                \textbf{OR}

          \item to draw a graph of $W$ against $\dfrac{1}{T^2}$ to find the gradient.



        \end{itemize}
\end{enumerate}

In reality, there is friction in the system, so in this case \( W \) is less than the total centripetal force in
the system. A suitable graph is plotted to determine the value of $mr$ experimentally. The value of
$mr$ was also calculated directly from the measured values of $m$ and $r$.

\begin{enumerate}[label=(\alph*)]
  \setcounter{enumi}{1}
  \item Predict from the equation whether the value of $mr$ found experimentally will be larger, the same, or smaller than the value of $mr$ calculated directly.
        \begin{itemize}
          \item Friction in this case is a systematic error and so if we consider the graph of $W$ against $\dfrac{1}{T^2}$, there will only be a vertical shift in the graph, but the gradient will remain the same.
          \item This means that the value of $mr$ is the same.
        \end{itemize}
  \item
        The measurements of $T$ were collected five times.
        \begin{enumerate}[label=(\roman*)]
          \item Explain how repeated measurements of $T$ reduced the random error in the final experimental value of.
                \begin{itemize}
                  \item Taking an average of the five measurements reduces the random error, as the random fluctuations are smoothed out.
                  \item This reduces the uncertainty in the final value of $T$.
                \end{itemize}
          \item Outline why repeated measurements of $T$ would not reduce any systematic error in $T$.
                \begin{itemize}
                  \item Because systematic errors are constant and present in all measurements, and so they will not be reduced by taking an average.
                \end{itemize}
        \end{enumerate}
\end{enumerate}

\subsection{Experiments (2)}

A spherical soap bubble is made of a thin film of soapy water. The bubble has an internal air pressure \( P_i \) and is formed in air of constant pressure \( P_o \). The theoretical prediction for the variation of \( (P_i - P_o) \) is given by the equation

\[
  (P_i - P_o) = \frac{4\gamma}{R}
\]

where \( \gamma \) is a constant for the thin film and \( R \) is the radius of the bubble.

Data for \( (P_i - P_o) \) and \( R \) were collected under controlled conditions and plotted as a graph showing the variation of \( (P_i - P_o) \) with \( \frac{1}{R} \).

\img{ex/7.png}{0.7}{The graph of $P_i - P_o$ against $\dfrac{1}{R}$.}{7}

\begin{enumerate}[label=(\alph*)]
  \item Suggest whether the data are consistent with the theoretical prediction.
        \begin{itemize}
          \item Theory suggests $P_i - P_o$ is proportional to $\dfrac{1}{R}$
          \item and the best fit graph is linear and passes through the origin and so the data is consistent with the theory.
        \end{itemize}
  \item \begin{enumerate}[label=(\roman*)]
          \item Show that the value of is about 0.03.
                \begin{itemize}
                  \item To make our lives easier, we pick two points whose coordinates are most easily read off the graph, i.e. positioned on the grid lines and are ideally on nice values.
                  \item We can pick $(20.00, 2.00)$ and $(25.00, 2.50)$
                        \begin{align*}
                          \text{Gradient} & = \frac{2.50 - 2.00}{25.00 - 20.00} \\
                                          & = \frac{0.5}{5}                     \\
                                          & = 0.1                               \\
                          \text{Gradient} & = 4\gamma                           \\
                          \gamma          & = \frac{0.1}{4}                     \\
                                          & = 0.025
                        \end{align*}
                \end{itemize}
          \item Identify the fundamental units of $\gamma$
                \begin{itemize}
                  \item A dimensional analysis question shows that $\gamma$ is essentially pressure times by length.
                  \item This is the equivalent of force over area and then times by length $\equiv$ force over length, with units $$\frac{\si{\kg\m\per\s\squared}}{\si{\m}}\equiv \si{\kg\per\s\squared}$$
                \end{itemize}
          \item In order to find the uncertainty for $\gamma$, a maximum gradient line would be drawn. On the graph, sketch the maximum gradient line for the data.
                \begin{itemize}
                  \item This line has to have the steepest slope possible while passing through all error bars.
                        \img{ex/8.png}{0.7}{The graph of $P_i - P_o$ against $\dfrac{1}{R}$.}{8}
                \end{itemize}
          \item The percentage uncertainty for $\gamma$ is 15\%. State $\gamma$, with its absolute uncertainty.
                \begin{align*}
                  0.15\times 0.025 & = 0.00375 \\
                \end{align*}
                This should be rounded to the same number of decimal places as the value of $\gamma$, namely 3 d.p., and hence \hlans{$\pm$ 0.004}
          \item The expected value of $\gamma$ is 0.027. Comment on your result.
                \begin{itemize}
                  \item The experiment concluded that $\gamma = 0.025 \pm 0.004$, and 0.027 is within this uncertainty range. Hence, the experimental value is consistent with the theoretical value.
                \end{itemize}
        \end{enumerate}
\end{enumerate}

\subsection{Experiments (3)}

A student investigates how the period \( T \) of a simple pendulum varies with the maximum speed \( v \) of the pendulum's bob by releasing the pendulum from rest from different initial angles. A graph of the variation of \( T \) with \( v \) is plotted.

\img{ex/9.png}{0.7}{The graph of $T$ against $v$.}{9}

\begin{enumerate}[label=(\alph*)]
  \item Suggest, by reference to the graph, why it is unlikely that the relationship between $T$ and $v$ is linear.
        \begin{itemize}
          \item A straight best-fit line cannot be drawn through all error bars
        \end{itemize}
  \item Determine the fractional uncertainty in $v$ when T = 2.115 s, correct to \textbf{one} significant figure.
        \begin{align*}
          v(T = 2.115)       & = 1.15              \\
          \frac{\Delta v}{v} & = \frac{0.05}{1.15} \\
                             & = 0.04 \equiv 4\%
        \end{align*}
  \item The student hypothesizes that the relationship between \( T \) and \( v \) is \( T = a + bv^2 \), where \( a \) and \( b \) are constants. To verify this hypothesis, a graph showing the variation of \( T \) with \( v^2 \) is plotted. The graph shows the data and the line of best fit.
        \img{ex/10.png}{1}{The graph of $T$ against $v^2$.}{10}
        Determine $b$, giving an appropriate unit for $b$.
        \begin{itemize}
          \item We recognize that $b$ is the gradient of the graph. Again, we pick two convenient points to compute the gradient.
                \begin{align*}
                  \text{Gradient} & = \frac{2.154 - 2.104}{4.4 - 0.5}    \\
                                  & = \SI{0.013}{\s\cubed\per\m\squared}
                \end{align*}
        \end{itemize}
  \item The lines of the minimum and maximum gradient are shown.
        \img{ex/11.png}{1}{The graph of $T$ against $v^2$.}{11}
        Estimate the absolute uncertainty in $a$.
        \begin{itemize}
          \item The graph provides us two pieces of information about $a$ through the intercepts -- $a_{\max} \approx 2.101$ and $a_{\min} \approx 2.095$.
                \begin{align*}
                  \Delta a & = \frac{a_{\max} - a_{\min}}{2} = \frac{2.101 - 2.095}{2} = 0.003
                \end{align*}
        \end{itemize}
\end{enumerate}

\subsection{Experiments (4)}

In an experiment to measure the acceleration of free fall a student ties two different blocks of masses $m_1$ and $m_2$ to the ends of a string that passes over a frictionless pulley.

\img{ex/12.png}{0.2}{A diagram of the experiment.}{12}

The student calculates the acceleration \( a \) of the blocks by measuring the time taken by the heavier mass to fall through a given distance. Their theory predicts that

$
  a = g \dfrac{m_1 - m_2}{m_1 + m_2}$ and this can be re-arranged to give $
  g = a \dfrac{m_1 + m_2}{m_1 - m_2}.
$

\begin{enumerate}[label=(\alph*)]
  \item In a particular experiment, the student calculates that \( a = (0.204 \pm 0.002) \, \mathrm{ms^{-2}} \) using \( m_1 = (0.125 \pm 0.001) \, \mathrm{kg} \) and \( m_2 = (0.120 \pm 0.001) \, \mathrm{kg} \).

        \begin{enumerate}[label=(\roman*)]
          \item Calculate the percentage error in the measured value of g.
                \begin{itemize}
                  \item The numerator and denominator each have an absolute uncertainty of 0.002.
                  \item The percentage error in the numerator is
                        $$\frac{0.002}{0.245} = 0.8\%$$
                  \item The percentage error in the denominator is
                        $$\frac{0.002}{0.005} = 40\%$$
                  \item Since they are divided, the percentage errors add up.
                        $$\frac{0.8 + 40}{100} = 40.8\%$$
                  \item Finally, this fraction is mulitplied to $a$, which has an uncertainty of $$\frac{0.002}{0.204} = 1\%$$
                  \item We should add this to the previous percentage error since they are all multiplied together
                        $$\frac{40.8 + 1}{100} = 41.8\%$$
                \end{itemize}
          \item Deduce the value of $g$ and its absolute uncertainty for this experiment.
                \begin{align*}
                  g = (0.204)\frac{0.245}{0.005} & = 9.996                          \\
                  \Delta g                       & = 0.418 \times 9.996 \approx 4.2 \\
                  g                              & = 10.0 \pm 4.2
                \end{align*}

        \end{enumerate}
  \item There is an advantage and a disadvantage in using two masses that are almost equal.

        \begin{enumerate}[label=(\roman*)]
          \item State and explain the advantage with reference to the magnitude of the acceleration that is obtained.
                \begin{itemize}
                  \item the acceleration would be small
                  \item a longer time of fall would be easier to measure and would reduce the percentage uncertainty in the time.
                \end{itemize}
          \item State and explain the disadvantage with reference to your answer to (a)(ii).
                \begin{itemize}
                  \item the percentage error in the difference of the masses is large
                  \item leading to a large percentage uncertainty in $g$
                \end{itemize}
        \end{enumerate}
\end{enumerate}

\subsection{Experiments (5)}

A student carries out an experiment to determine the variation of intensity of the light with distance from a point light source. The light source is at the centre of a transparent spherical cover of radius $C$. The student measures the distance $x$ from the surface of the cover to a sensor that measures the intensity $I$ of the light.

\img{ex/13.png}{0.7}{A diagram of the experiment.}{13}

The light source emits radiation with a constant power $P$ and all of this radiation is transmitted through the cover. The relationship between $I$ and $x$ is given by

\[
  I = \frac{P}{4\pi (x + C)^2}
\]

This relationship can also be rewritten as follows.

$$
  \frac{1}{\sqrt{I}} = 2\sqrt{\frac{\pi}{P}}(x + C)
$$

\begin{enumerate}[label=(\alph*)]
  \item The student obtains a set of data and uses this to plot a graph of the variation of $\dfrac{1}{\sqrt{I}}$ with $x$
        \img{ex/14.png}{1}{The graph of $\dfrac{1}{\sqrt{I}}$ against $x$.}{14}
        \begin{enumerate}[label=(\roman*)]
          \item Estimate $C$
                \begin{itemize}
                  \item From the relation, we see that the gradient and $C$ multiply to give the intercept.
                  \item The gradient can be found with any pair of points, picking the two end points give
                        $$\text{Gradient} = \frac{36 - 33}{25 - 22.5} = 1.2$$

                  \item If we extend the line and intercept the y-axis, we can read off the intercept as $5$.
                        $$C = \frac{5}{1.2} = 4.2\si{\cm}$$
                \end{itemize}
          \item Determine $P$, to the correct number of significant figures including its unit.
                \begin{align*}
                  \text{Gradient} & = 2\sqrt{\frac{\pi}{P}}          \\
                  P               & = \frac{2\pi}{\text{Gradient}^2} \\
                                  & = 8.7 \times 10^{-4} \si{\watt}
                \end{align*}
        \end{enumerate}
  \item Explain the disadvantage that a graph of $I$ against $\dfrac{1}{x^2}$ has for the analysis in (b)(i) and (b)(ii).
        \begin{itemize}
          \item This graph will be a curve and not a straight line
          \item Hence, it would be more difficult to determine the gradient and intercept.
        \end{itemize}
\end{enumerate}

\section{RevisionVillage Questions}

Don't panic if you don't have a RV subscription: Trust me, the questions are not too dissimilar to the ones from past papers. In fact, you could argue they are scammers...

\oldsubsection{Dimensional Analysis}

Reynolds number, \( Re \), is a dimensionless number (a value with no units) used in fluid mechanics to predict the nature of flow of a fluid. It is given by:

\[
  Re = v l \rho^\alpha \eta^\beta
\]

where \( v \) is the velocity of the flow, \( l \) is length, \( \rho \) is the density of the fluid and \( \eta \) its viscosity. What are the exponents \( \alpha \) and \( \beta \)?

\begin{itemize}
  \item Since the Reynolds number is dimensionless, we can equate the dimensions of both sides of the equation.
        \begin{align*}
          1\equiv \left(\si{\m\per\s}\right)\left(\si{\m}\right)\left(\si{\kg\per\m\cubed}\right)^\alpha\left(\eta\right)^\beta
        \end{align*}
  \item If you are like me and cannot immediately recall the units of viscosity, we can think about Stoke's Law
        \begin{align*}
          F    & = 6\pi r \eta v                                               \\
          \eta & \equiv \frac{\si{\kg\m\per\s\squared}}{\si{\m\squared\per\s}} \\
               & \equiv \si{\kg\per\m\per\s}
        \end{align*}
  \item Then
        $$1\equiv \left(\si{\m\per\s}\right)\left(\si{\m}\right)\left(\si{\kg\per\m\cubed}\right)^\alpha\left(\si{\kg\per\m\per\s}\right)^\beta$$
  \item Our goal is to cancel out the exponents of the units on the RHS, so that the whole quantity becomes dimensionless.
        \begin{itemize}
          \item The sum of the exponents of m is $1 + 1 - 3\alpha - \beta = 0$. Hence,
                $$3\alpha + \beta = 2$$
          \item The sum of the exponents of kg is $\alpha + \beta = 0$.
          \item Solving this system of equations gives us $\alpha = 1$ and $\beta = -1$.
        \end{itemize}
\end{itemize}


\end{document}