\documentclass[a4paper,12pt]{article}
\usepackage{setspace}
\usepackage{sectsty}
\usepackage{siunitx}
\usepackage{graphicx}
\usepackage[a4paper, total={3in, 9in}, textwidth=16cm,bottom=1in,top=1.4in]{geometry}
\usepackage[dvipsnames]{xcolor}
\usepackage{amsmath}
\usepackage{esvect}
\usepackage{soul}
\usepackage{amsthm}
\usepackage{hyperref}
\usepackage{float}
\usepackage{amssymb}
\usepackage{outlines}
\usepackage{caption}
\usepackage{fancyvrb}
\usepackage{subcaption}
\usepackage{esdiff}
\usepackage{setspace}
\usepackage{mathtools}
\usepackage{tikz,pgfplots}
\usepackage{dirtytalk}
\usepackage{draftwatermark}
\usepackage[most]{tcolorbox}
\SetWatermarkText{timthedev07}
\SetWatermarkScale{4}
\SetWatermarkColor[gray]{0.97}
\usetikzlibrary{positioning,decorations.markings,calc}
\DeclarePairedDelimiter{\ceil}{\lceil}{\rceil}
\newtheorem{lemma}{Lemma}
\newtheorem{proposition}{Proposition}
\newtheorem{remark}{Remark}
\newtheorem{observation}{Observation}
\doublespacing
\let\oldsection\section
\renewcommand\section{\clearpage\oldsection}
\newcommand{\RNum}[1]{\uppercase\expandafter{\romannumeral #1\relax}}
\let\oldsi\si
\renewcommand{\si}[1]{\oldsi[per-mode=reciprocal-positive-first]{#1}}
\usepackage{enumitem}
\newcommand{\subtitle}[1]{%
  \posttitle{%
    \par\end{center}
    \begin{center}\large#1\end{center}
    \vskip0.5em}%
}
\newcommand{\degsym}{^{\circ}}
\newcommand{\Mod}[1]{\ (\mathrm{mod}\ #1)}
\usepackage{hyperref}
\hypersetup{
  colorlinks=true,
  linkcolor = blue
}
\newcommand{\lb}{\\[8pt]}
\newenvironment*{cell}[1][]{\begin{tabular}[c]{@{}c@{}}}{\end{tabular}}
\newcommand{\img}[4]{\begin{center}
  \begin{figure}[H]
    \centering
    \includegraphics[width=#2\textwidth]{#1}
    \caption{#3}
    \label{fig:#4}
  \end{figure}
\end{center}}
\parindent=0pt
\usepackage{fancyhdr}
\fancyfoot{}
\newcommand{\vect}[3]{\begin{bmatrix}
  #1 \\
  #2 \\
  #3
\end{bmatrix}}
\fancypagestyle{fancy}{\fancyfoot[R]{\vspace*{1.5\baselineskip}\thepage}}
\renewcommand{\contentsname}{Table of Contents}
\newcommand{\angled}[1]{\langle{#1}\rangle}
\newcommand{\paren}[1]{\left(#1\right)}
\newcommand{\sqb}[1]{\left[#1\right]}
\newcommand{\coord}[3]{\angled{#1,\, #2,\, #3}}
\newcommand{\pair}[2]{\paren{#1,\, #2}}
\usepackage[
  noabbrev,
  capitalise,
  nameinlink,
]{cleveref}
\crefname{lemma}{Lemma}{Lemmas}
\crefname{proposition}{Proposition}{Propositions}
\crefname{remark}{Remark}{Remarks}
\crefname{observation}{Observation}{Observations}

\newtcolorbox[auto counter]{defin}[1][]{fonttitle=\bfseries, title=\strut Definition.~\thetcbcounter,colback=black!5!white,colframe=black!65!gray,top=5mm,bottom=5mm}

\newtcolorbox[auto counter]{obs}[1][]{fonttitle=\bfseries, title=\strut Observation.~\thetcbcounter,colback=RedViolet!5!white,colframe=RedViolet!65!gray,top=5mm,bottom=5mm}

\setlength{\belowcaptionskip}{-20pt}

\begin{document}


\pagenumbering{arabic}
\pagestyle{fancy}


\begin{titlepage}
  \begin{center}

    \vspace*{8cm}
    \textbf{\Large {IB Physics Topic C5 The Doppler Effect; SL \& HL}} \\
    \vspace*{1cm}
    \large{By timthedev07, M25 Cohort}


  \end{center}
\end{titlepage}

\pagebreak
\tableofcontents
\pagebreak

\clearpage
\setcounter{page}{1}
\addtocontents{toc}{\protect\thispagestyle{empty}}

\section{The Doppler Effect for Sound}
\hl{The Doppler effect is a change in observed the frequency and wavelength of a wave that arises from relative motion between the source and the observer}. In the following parts we will look at how one can compute the shifted frequencies; it must be noted, however, that these only hold for low speeds, below $0.2c$. Otherwise, one must use the relativistic Doppler effect, which is not covered.


\subsection{Case I: Stationary Observer, Moving Source}

\begin{minipage}{0.45\textwidth}
  \img{1A.png}{1}{Moving towards the observer}{1A}
\end{minipage}\hspace*{0.1\textwidth}%
\begin{minipage}{0.45\textwidth}
  \img{1B.png}{1}{Moving away from the observer}{1B}
\end{minipage}

\begin{itemize}
  \item In the first sub-case, where the \hl{relative velocity} between the two is $u_s$, the \hl{the observed frequency is higher}, as the wavefronts are compressed. The formula for the shifted frequency is given by the following, \textbf{where $v$ is the wave speed}
        \begin{equation}
          f' = f\paren{\frac{v}{v - |u_s|}} > f
        \end{equation}
  \item In the second sub-case, where the source is moving away from the observer, the \hl{observed frequency is lower}, as the wavefronts are stretched.
        \begin{equation}
          f' = f\paren{\frac{v}{v + |u_s|}} < f
        \end{equation}
\end{itemize}

\pagebreak


\subsection{Case II: Moving Observer, Stationary Source}

When the source is stationary and the observer is moving, there are also shifts in the observed frequency.
\begin{itemize}
  \item If the observer is moving towards the source, the observed frequency is higher, as it will \hl{cross more wavefronts in a given time} than if it were stationary.
        \begin{equation}
          f' = f\paren{\frac{v + |u_o|}{v}} > f
        \end{equation}
  \item Inversely, if the observer is moving away from the source, the observed frequency is lower, as it will \hl{cross fewer wavefronts in a given time}.
        \begin{equation}
          f' = f\paren{\frac{v - |u_o|}{v}} < f
        \end{equation}
\end{itemize}

\subsection{The Combined Effect}

The complete equation that encapsulates both cases is given by
\begin{equation}
  f' = f\paren{\frac{v\pm |u_O|}{v\pm |u_S|}}
\end{equation}
where
\begin{itemize}
  \item $f$ is the frequency of the source
  \item $f'$ is the observed frequency
  \item $v$ is the wave speed
  \item $u_O$ is the velocity of the observer
  \item $u_S$ is the velocity of the source
\end{itemize}

The ones discussed before are just special cases of this general relation.
\subsection{Doppler with Acceleration}

\begin{itemize}
  \item If the movement is under a \textbf{constant speed}, then, the frequency will remain constant but at a different value.
  \item If the movement is under an \textbf{acceleration}, then, the frequency will change.
\end{itemize}

Example question: A microphone M on a moving train detects the sound from a stationary loudspeaker L placed on the track. The track is straight. L emits a sound of constant frequency. The frequency detected by M is continuously decreasing. The train is moving

\begin{enumerate}[label=\Alph*.]
  \item away from L at an increasing speed.
  \item away from L at a constant speed.
  \item towards L at a constant speed.
  \item towards L at an increasing speed.
\end{enumerate}

Firstly, if the frequency is decreasing, the train must be moving away from the source. \textcolor{red}{This eliminates options C and D.} Then, if the frequency is changing instead of remaining at a constant lower value, the train must be accelerating. Thus, the answer is \textcolor{red}{A}.

\section{The Doppler Effect for E.M. Waves}

The previous analysis does not apply to light for the following reasons:
\begin{itemize}
  \item E.M. waves do not require a medium
  \item Light travels at the same speed in all frames of reference, but this is not the case with sound.
  \item In special relativity, there is no such concept as a source and an observer, as all motion is relative. Thus, only the relative velocity between the source and the observer matters.
\end{itemize}

When $\Delta u$, \textbf{the relative velocity between the two is much less than the speed of light}, we may use a relation that is modified by
\begin{itemize}
  \item substituting the speed of light for the wavespeed
  \item using the relative velocity between the source and the observer, denote it as $\Delta u$
\end{itemize}
Eventually we arrive with the relation
\begin{equation}
  f' = f\paren{1 - \frac{\Delta u}{c}}
\end{equation}
If we denote the change in frequency as $\Delta f = f - f'$, we can rewrite the equation as
\begin{equation}
  \Delta f = f\frac{\Delta u}{c} \iff \frac{\Delta \lambda}{\lambda} = \frac{\Delta f}{f} = \frac{\Delta u}{c}
\end{equation}

\pagebreak

\subsection{Stellar and Galatic Motion}

The Doppler effect is used to determine the motion of stars and galaxies. \lb
Observe the light spectra of light emitted by a star at two distinct timestamps.
\begin{itemize}
  \item If the most recent one's spectra lines are shifted toward the blue terminal of the spectrum, which represents a \hl{blue shift} (increase in frequency and hence decrease in wavelength), the star is moving towards the observer.
  \item If the most recent one's spectra lines are shifted toward the red terminal of the spectrum, which represents a \hl{red shift} (decrease in frequency and hence increase in wavelength), the star is moving away from the observer.
\end{itemize}

\section{Applications of the Doppler Effect}

\subsection{Blood Flow Measurement}

\img{bloodvessel.png}{0.5}{Blood flow measurement}{bloodvessel}

\begin{equation}
  \frac{\Delta f}{f} = \frac{2u\cos\theta}{v}
\end{equation}
where
\begin{itemize}
  \item $u$ is the speed of the blood (wrong in the diagram)
  \item $\theta$ is the angle between the direction of the blood flow and the direction of the sound wave
  \item $v$ is the speed of sound in the blood
\end{itemize}

\pagebreak

\subsection{RADAR}

This includes the following applications:
\begin{itemize}
  \item flow measurements, e.g. medical, rain cloud speed measurements, weather forecasting
  \item vehicle speed determinations (police speed traps)
  \item remote sensing of ocean currents
  \item measurement of turbulence in river and ocean flow
\end{itemize}

\section{Exam Questions}

\subsection{Intensity and Frequency Changes}

A train approaches a station and sounds a horn of constant frequency and constant intensity. An observer waiting at the station detects a frequency $f_\text{obs}$ and an intensity $I_\text{obs}$. What are the changes, if any, in $f_\text{obs}$ and $I_\text{obs}$ as the train slows down?
\begin{table}[H]
  \centering
  \begin{tabular}{c | c}
    $I_\text{obs}$ & $f_\text{obs}$ \\
    \hline
    no change      & decreases      \\
    increases      & increases      \\
    no change      & increases      \\
    increases      & decreases
  \end{tabular}
\end{table}
\begin{itemize}
  \item It is tempted to say that $f_\text{obs}$ increases because the source is approaching the observer. However, the question is not asking for $f_\text{obs}$ relative to $f_\text{source}$, but rather the change in $f_\text{obs}$ itself as the train slows down. In this case, the source is approaching the observer at a slower rate, so while $f_\text{obs}$ will be higher than $f_\text{source}$, it will decrease as the train slows down.
  \item Recall that the \textbf{observed intensity} $I_\text{obs}$ obeys the inverse square law
        $$I_\text{obs} = \frac{P_\text{source}}{4\pi r^2}$$
        we are given that the intensity of the source is constant and thus the power of the source is constant. As the train slows down, the distance between the source and the observer decreases, thus increasing $I_\text{obs}$.
\end{itemize}

\pagebreak

\subsection{Applying the Doppler Effect}

Sea waves move towards a beach at a constant speed of $2.0 \mathrm{~m} \mathrm{~s}^{-1}$. They arrive at the beach with a frequency of 0.10 Hz . A girl on a surfboard is moving in the sea at right angles to the wave fronts. She observes that the surfboard crosses the wave fronts with a frequency of 0.40 Hz .
\img{ex/sea.png}{0.5}{Sea waves}{sea}
What is the speed of the surfboard and the direction of motion of the surfboard relative to the beach? To tackle Doppler effect questions,
\begin{enumerate}
  \item Identify the source and observer --- in this case, the source is somewhere in the sea and stationary (no explicit references to the source moving); the observer is the girl.
  \item Identify the wave speed --- this is given as $v = 2.0 \mathrm{~m} \mathrm{~s}^{-1}$.
  \item Identify the direction of motion of the observer relative to the source --- this is not explicitly given, but if we think about the clue in the question that the observer detects a higher frequency, we can infer that the observer is moving towards the source, so we know that the desired relation is
        $$f' = f\paren{\frac{v + |u_o|}{v}}$$
  \item Plug in the values and solve for $u_o$.
\end{enumerate}

\subsection{Reflected Doppler Effect}

\img{ex/plate.png}{0.5}{Plate}{plate}

A plate performs simple harmonic oscillations with a frequency of 39 Hz and an amplitude of 8.0 cm. Sound of frequency 2400 Hz is emitted from a stationary source towards the oscillating plate. The speed of sound is $\SI{340}{\meter\per\second}$. Determine the maximum frequency of the sound that is received back at the source after reflection at the plate.
\begin{enumerate}
  \item \hl{The reflected sound wave will have the highest frequency when the plate is at its maximum speed towards the source}. Consider kicking a soccer ball approaching you, if you kick in the opposite motion of the approaching ball with the greatest strength and speed, the ball will have the highest speed when it goes back.
  \item By s.h.m. equations in C1, we conclude that the highest speed of the plate is $\SI{19.6}{\meter\per\second}$.
  \item Putting all this together, let's first calculate the frequency of the sound that arrives at the plate when it is at $\SI{19.6}{\meter\per\second}$ towards the source:
        $$f' = f\paren{\frac{v + |u_o|}{v}} = 2400\paren{\frac{340 + 19.6}{340}} = 2400\paren{\frac{359.6}{340}} = 2538\mathrm{~Hz}$$
  \item Now, we must consider the reflected sound. In this half of the journey, the plate acts as the \say{source} and it is moving towards the actual source. Thus, the frequency of the reflected sound is
        $$f'' = f'\paren{\frac{v}{v - |u_s|}} = 2538\paren{\frac{340}{340 - 19.6}} = 2694 \approx 2700 \mathrm{~Hz}$$
\end{enumerate}

\pagebreak

\subsection{Misc \#1}

The diagram shows a point source of sound S on the edge of a horizontal turntable that rotates about a vertical axis. The sound is detected using a small stationary frequency meter placed 0.78 m from the axis of the turntable. The turntable has a radius of 0.28 m. The linear speed of S is much less than the speed of sound.

\img{ex/1.png}{0.9}{Turntable}{1}

The graph shows the variation of the detected frequency with rotation angle $\theta$ for one revolution of the turntable.

\img{ex/2.png}{0.9}{Graph}{2}

\begin{enumerate}[label=(\alph*)]
  \item State, on the diagram, the position of S for which the detected frequency is at a maximum. Label this position A.
        \img{ex/3.png}{0.9}{Position A}{3}
  \item Outline why this maximum frequency shift does not occur when $\theta= 90\degsym$ or when $\theta= 270\degsym$.
        \begin{itemize}
          \item At the angle of 90 degrees, the tangential velocity is pointing directly to the left, but to obtain the maximum frequency shift, the tangential velocity must be pointing directly towards the observer.
        \end{itemize}
  \item Determine the angular speed of the turntable. The speed of sound is $\SI{330}{\m\per\s}$. State an appropriate unit for your answer.
        \begin{itemize}
          \item We pick the maximum point of the graph for calculation, which is at the frequency $450.5 \mathrm{~Hz}$.
        \end{itemize}
        \begin{align*}
          \frac{\Delta f}{f} = \frac{\Delta u}{v} \\
          \frac{10.5}{440} = \frac{\Delta u}{330} \\
          \Delta u = 7.9 \mathrm{~m} \mathrm{~s}^{-1}
        \end{align*}
        \begin{itemize}
          \item The quantity we are getting here is the tangential velocity, which is related to the angular speed by
                $$\Delta u = r\omega \iff \omega = \frac{\Delta u}{r}$$
                Hence
                $$\omega = \frac{7.9}{0.28} = 28.2 \mathrm{~rad} \mathrm{~s}^{-1}$$
        \end{itemize}
\end{enumerate}

\pagebreak

\subsection{Acceleration in Doppler Effect}

On approaching a stationary observer, a train sounds its horn and decelerates at a constant rate. At time t the train passes by the observer and continues to decelerate at the same rate. Which diagram shows the variation with time of the frequency of the sound measured by the observer?

\img{ex/4.png}{0.9}{Train}{4}

\begin{itemize}
  \item When the train decelerates towards the station, the observed frequency is still higher than the original frequency but decreasing. \textcolor{red}{This eliminates options A and C.}
  \item When the train passes the observer and continues to decelerate, the observed frequency will be lower than the original frequency but will be increasing instead (decelerating when moving away is like accelerating towards the observer). \textcolor{ForestGreen}{This gives option D.}
\end{itemize}

\end{document}