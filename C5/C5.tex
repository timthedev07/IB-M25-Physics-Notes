\documentclass[a4paper,12pt]{article}
\usepackage{setspace}
\usepackage{sectsty}
\usepackage{siunitx}
\usepackage{graphicx}
\usepackage[a4paper, total={3in, 9in}, textwidth=16cm,bottom=1in,top=1.4in]{geometry}
\usepackage[dvipsnames]{xcolor}
\usepackage{amsmath}
\usepackage{esvect}
\usepackage{soul}
\usepackage{amsthm}
\usepackage{hyperref}
\usepackage{float}
\usepackage{amssymb}
\usepackage{outlines}
\usepackage{caption}
\usepackage{fancyvrb}
\usepackage{subcaption}
\usepackage{esdiff}
\usepackage{setspace}
\usepackage{mathtools}
\usepackage{tikz,pgfplots}
\usepackage{dirtytalk}
\usepackage{draftwatermark}
\usepackage[most]{tcolorbox}
\SetWatermarkText{timthedev07}
\SetWatermarkScale{4}
\SetWatermarkColor[gray]{0.97}
\usetikzlibrary{positioning,decorations.markings,calc}
\DeclarePairedDelimiter{\ceil}{\lceil}{\rceil}
\newtheorem{lemma}{Lemma}
\newtheorem{proposition}{Proposition}
\newtheorem{remark}{Remark}
\newtheorem{observation}{Observation}
\doublespacing
\let\oldsection\section
\renewcommand\section{\clearpage\oldsection}
\newcommand{\RNum}[1]{\uppercase\expandafter{\romannumeral #1\relax}}
\let\oldsi\si
\renewcommand{\si}[1]{\oldsi[per-mode=reciprocal-positive-first]{#1}}
\usepackage{enumitem}
\newcommand{\subtitle}[1]{%
  \posttitle{%
    \par\end{center}
    \begin{center}\large#1\end{center}
    \vskip0.5em}%
}
\newcommand{\degsym}{^{\circ}}
\newcommand{\Mod}[1]{\ (\mathrm{mod}\ #1)}
\usepackage{hyperref}
\hypersetup{
  colorlinks=true,
  linkcolor = blue
}
\newcommand{\lb}{\\[8pt]}
\newenvironment*{cell}[1][]{\begin{tabular}[c]{@{}c@{}}}{\end{tabular}}
\newcommand{\img}[4]{\begin{center}
  \begin{figure}[H]
    \centering
    \includegraphics[width=#2\textwidth]{#1}
    \caption{#3}
    \label{fig:#4}
  \end{figure}
\end{center}}
\parindent=0pt
\usepackage{fancyhdr}
\fancyfoot{}
\newcommand{\vect}[3]{\begin{bmatrix}
  #1 \\
  #2 \\
  #3
\end{bmatrix}}
\fancypagestyle{fancy}{\fancyfoot[R]{\vspace*{1.5\baselineskip}\thepage}}
\renewcommand{\contentsname}{Table of Contents}
\newcommand{\angled}[1]{\langle{#1}\rangle}
\newcommand{\paren}[1]{\left(#1\right)}
\newcommand{\sqb}[1]{\left[#1\right]}
\newcommand{\coord}[3]{\angled{#1,\, #2,\, #3}}
\newcommand{\pair}[2]{\paren{#1,\, #2}}
\usepackage[
  noabbrev,
  capitalise,
  nameinlink,
]{cleveref}
\crefname{lemma}{Lemma}{Lemmas}
\crefname{proposition}{Proposition}{Propositions}
\crefname{remark}{Remark}{Remarks}
\crefname{observation}{Observation}{Observations}

\newtcolorbox[auto counter]{defin}[1][]{fonttitle=\bfseries, title=\strut Definition.~\thetcbcounter,colback=black!5!white,colframe=black!65!gray,top=5mm,bottom=5mm}

\newtcolorbox[auto counter]{obs}[1][]{fonttitle=\bfseries, title=\strut Observation.~\thetcbcounter,colback=RedViolet!5!white,colframe=RedViolet!65!gray,top=5mm,bottom=5mm}

\setlength{\belowcaptionskip}{-20pt}

\begin{document}


\pagenumbering{arabic}
\pagestyle{fancy}


\begin{titlepage}
  \begin{center}

    \vspace*{8cm}
    \textbf{\Large {IB Physics Topic C5 The Doppler Effect; SL \& HL}} \\
    \vspace*{1cm}
    \large{By timthedev07, M25 Cohort}


  \end{center}
\end{titlepage}

\pagebreak
\tableofcontents
\pagebreak

\clearpage
\setcounter{page}{1}
\addtocontents{toc}{\protect\thispagestyle{empty}}

\section{The Doppler Effect for Sound}
The Doppler effect is a change in observed the frequency and wavelength of a wave that arises from relative motion between the source and the observer. In the following parts we will look at how one can compute the shifted frequencies; it must be noted, however, that these only hold for low speeds, below $0.2c$. Otherwise, one must use the relativistic Doppler effect, which is not covered.


\subsection{Case I: Stationary Observer, Moving Source}

\begin{minipage}{0.45\textwidth}
  \img{1A.png}{1}{Moving towards the observer}{1A}
\end{minipage}\hspace*{0.1\textwidth}%
\begin{minipage}{0.45\textwidth}
  \img{1B.png}{1}{Moving away from the observer}{1B}
\end{minipage}

\begin{itemize}
  \item In the first sub-case, where the source is moving towards the observer at speed $u_s$, the \hl{the observed frequency is higher}, as the wavefronts are compressed. The formula for the shifted frequency is given by the following, \textbf{where $v$ is the wave speed}
        \begin{equation}
          f' = f\paren{\frac{v}{v - |u_s|}} > f
        \end{equation}
  \item In the second sub-case, where the source is moving away from the observer, the \hl{observed frequency is lower}, as the wavefronts are stretched.
        \begin{equation}
          f' = f\paren{\frac{v}{v + |u_s|}} < f
        \end{equation}
\end{itemize}

\pagebreak


\subsection{Case II: Moving Observer, Stationary Source}

When the source is stationary and the observer is moving, there are also shifts in the observed frequency.
\begin{itemize}
  \item If the observer is moving towards the source, the observed frequency is higher, as it will \hl{cross more wavefronts in a given time} than if it were stationary.
        \begin{equation}
          f' = f\paren{\frac{v + |u_o|}{v}} > f
        \end{equation}
  \item Inversely, if the observer is moving away from the source, the observed frequency is lower, as it will \hl{cross fewer wavefronts in a given time}.
        \begin{equation}
          f' = f\paren{\frac{v - |u_o|}{v}} < f
        \end{equation}
\end{itemize}

\subsection{The Combined Effect}

The complete equation that encapsulates both cases is given by
\begin{equation}
  f' = f\paren{\frac{v\pm |u_O|}{v\pm |u_S|}}
\end{equation}
where
\begin{itemize}
  \item $f$ is the frequency of the source
  \item $f'$ is the observed frequency
  \item $v$ is the wave speed
  \item $u_O$ is the velocity of the observer
  \item $u_S$ is the velocity of the source
\end{itemize}

The ones discussed before are just special cases of this general relation.

\section{The Doppler Effect for Light}

The previous analysis does not apply to light for the following reasons:
\begin{itemize}
  \item E.M. waves do not require a medium
  \item Light travels at the same speed in all frames of reference, but this is not the case with sound.
  \item In special relativity, there is no such concept as a source and an observer, as all motion is relative. Thus, only the relative velocity between the source and the observer matters.
\end{itemize}

When $\Delta u$, \textbf{the relative velocity between the two is much less than the speed of light}, we may use a relation that is modified by
\begin{itemize}
  \item substituting the speed of light for the wavespeed
  \item using the relative velocity between the source and the observer, denote it as $\Delta u$
\end{itemize}
Eventually we arrive with the relation
\begin{equation}
  f' = f\paren{1 - \frac{\Delta u}{c}}
\end{equation}
If we denote the change in frequency as $\Delta f = f - f'$, we can rewrite the equation as
\begin{equation}
  \Delta f = f\frac{\Delta u}{c} \iff \frac{\Delta \lambda}{\lambda} = \frac{\Delta f}{f} = \frac{\Delta u}{c}
\end{equation}

\pagebreak

\subsection{Stellar and Galatic Motion}

The Doppler effect is used to determine the motion of stars and galaxies. \lb
Observe the light spectra of light emitted by a star at two distinct timestamps.
\begin{itemize}
  \item If the most recent one's spectra lines are shifted toward the blue terminal of the spectrum, which represents a \hl{blue shift} (increase in frequency and hence decrease in wavelength), the star is moving towards the observer.
  \item If the most recent one's spectra lines are shifted toward the red terminal of the spectrum, which represents a \hl{red shift} (decrease in frequency and hence increase in wavelength), the star is moving away from the observer.
\end{itemize}

\section{Applications of the Doppler Effect}

\subsection{Blood Flow Measurement}

\img{bloodvessel.png}{0.5}{Blood flow measurement}{bloodvessel}

\begin{equation}
  \frac{\Delta f}{f} = \frac{2u\cos\theta}{v}
\end{equation}
where
\begin{itemize}
  \item $u$ is the speed of the blood (wrong in the diagram)
  \item $\theta$ is the angle between the direction of the blood flow and the direction of the sound wave
  \item $v$ is the speed of sound in the blood
\end{itemize}

\pagebreak

\subsection{RADAR}

This includes the following applications:
\begin{itemize}
  \item flow measurements, e.g. medical, rain cloud speed measurements, weather forecasting
  \item vehicle speed determinations (police speed traps)
  \item remote sensing of ocean currents
  \item measurement of turbulence in river and ocean flow
\end{itemize}

\end{document}