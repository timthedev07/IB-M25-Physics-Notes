\documentclass[a4paper,12pt]{article}
\usepackage{setspace}
\usepackage{sectsty}
\usepackage{siunitx}
\usepackage{graphicx}
\usepackage[a4paper, total={3in, 9in}, textwidth=16cm,bottom=1in,top=1.4in]{geometry}
\usepackage[dvipsnames]{xcolor}
\usepackage{amsmath}
\usepackage{esvect}
\usepackage{soul}
\usepackage{amsthm}
\usepackage{hyperref}
\usepackage{longtable}
\usepackage{float}
\usepackage{amssymb}
\usepackage{outlines}
\usepackage{caption}
\usepackage{fancyvrb}
\usepackage{subcaption}
\usepackage{esdiff}
\usepackage{dirtytalk}
\usepackage{colortbl}
\usepackage{booktabs}
\usepackage{setspace}
\usepackage{mathtools}
\usepackage{tikz,pgfplots}
\usepackage[most]{tcolorbox}
\usetikzlibrary{positioning,decorations.markings,arrows.meta,angles,quotes}
\DeclarePairedDelimiter{\ceil}{\lceil}{\rceil}
\newtheorem{lemma}{Lemma}
\newtheorem{proposition}{Proposition}
\newtheorem{remark}{Remark}
\newtheorem{observation}{Observation}
\doublespacing
\let\oldsection\section
\renewcommand\section{\clearpage\oldsection}
\newcommand{\RNum}[1]{\uppercase\expandafter{\romannumeral #1\relax}}
\let\oldsi\si
\renewcommand{\si}[1]{\oldsi[per-mode=reciprocal-positive-first]{#1}}
\usepackage{enumitem}
\newcommand{\subtitle}[1]{%
  \posttitle{%
    \par\end{center}
    \begin{center}\large#1\end{center}
    \vskip0.5em}%
}
\newcommand{\degsym}{^{\circ}}
\newcommand{\Mod}[1]{\ (\mathrm{mod}\ #1)}
\usepackage{hyperref}
\hypersetup{
  colorlinks=true,
  linkcolor = blue
}
\newcommand{\lb}{\\[8pt]}
\newenvironment*{cell}[1][]{\begin{tabular}[c]{@{}c@{}}}{\end{tabular}}
\newcommand{\img}[4]{\begin{center}
  \begin{figure}[H]
    \centering
    \includegraphics[width=#2\textwidth]{#1}
    \caption{#3}
    \label{fig:#4}
  \end{figure}
\end{center}}
\parindent=0pt
\usepackage{fancyhdr}
\fancyfoot{}
\fancypagestyle{fancy}{\fancyfoot[R]{\vspace*{1.5\baselineskip}\thepage}}
\renewcommand{\contentsname}{Table of Contents}
\newcommand{\angled}[1]{\langle{#1}\rangle}
\newcommand{\paren}[1]{\left(#1\right)}
\newcommand{\sqb}[1]{\left[#1\right]}
\newcommand{\coord}[3]{\angled{#1,\, #2,\, #3}}
\newcommand{\pair}[2]{\paren{#1,\, #2}}
\newcommand{\atom}[3]{{}^{#1}_{#2}\text{#3}}
\usepackage[
  noabbrev,
  capitalise,
  nameinlink,
]{cleveref}

\crefname{lemma}{Lemma}{Lemmas}
\crefname{proposition}{Proposition}{Propositions}
\crefname{remark}{Remark}{Remarks}
\crefname{observation}{Observation}{Observations}

\newtcolorbox[auto counter]{prob}[2][]{fonttitle=\bfseries, title=\strut Problem~\thetcbcounter: #2,#1,colback=Orchid!5!white,colframe=Orchid!75!black,top=5mm,bottom=5mm}

\newtcolorbox[auto counter]{rem}[1][]{fonttitle=\bfseries, title=\strut Remark.~\thetcbcounter,colback=purple!5!white,colframe=purple!65!gray,top=5mm,bottom=5mm}

\newtcolorbox[auto counter]{defin}[1][]{fonttitle=\bfseries, title=\strut Definition.~\thetcbcounter,colback=black!5!white,colframe=black!65!gray,top=5mm,bottom=5mm}

\newtcolorbox[auto counter]{obs}[1][]{fonttitle=\bfseries, title=\strut Observation.~\thetcbcounter,colback=RedViolet!5!white,colframe=RedViolet!65!gray,top=5mm,bottom=5mm}

\newtcolorbox[auto counter]{lem}[1][]{fonttitle=\bfseries, title=\strut Lemma.~\thetcbcounter,colback=Maroon!5!white,colframe=Maroon!65!gray,top=5mm,bottom=5mm}

\newtcolorbox[auto counter]{prop}[1][]{fonttitle=\bfseries, title=\strut Proposition.~\thetcbcounter,colback=RedOrange!5!white,colframe=RedOrange!65!gray,top=5mm,bottom=5mm}

\newtcolorbox[auto counter]{hint}[1][]{fonttitle=\bfseries, title=\strut Hint.~\thetcbcounter,colback=OliveGreen!5!white,colframe=OliveGreen!75!gray,top=5mm,bottom=5mm}

\setlength{\belowcaptionskip}{-20pt}
\begin{document}


\pagenumbering{arabic}
\pagestyle{fancy}


\begin{titlepage}
  \begin{center}

    \vspace*{8cm}
    \textbf{\Large {IB Physics Topic B3 Gas Laws; SL \& HL}} \\
    \vspace*{1cm}
    \large{By timthedev07, M25 Cohort}

  \end{center}
\end{titlepage}

\pagebreak
\tableofcontents
\pagebreak

\clearpage
\setcounter{page}{1}
\addtocontents{toc}{\protect\thispagestyle{empty}}

\section{Pressure}

Pressure is defined as the force per unit area.
\begin{equation}\label{eq:pressure_def}
  P = \frac{F}{A}
\end{equation}
\begin{itemize}
  \item \textbf{Solid}: The pressure due to the weight $W$ over an area $A$ is given by
        \begin{equation}\label{eq:solid_pressure}
          P = \frac{W}{A}
        \end{equation}
  \item \textbf{Liquid}: The pressure in a liquid at a depth $h$ is given by
        \begin{equation}\label{eq:liquid_pressure}
          P = \rho gh
        \end{equation}
        where $\rho$ is the density of the liquid, $g$ is the acceleration due to gravity, and $h$ is the depth.
  \item \textbf{Gas}: Will be discussed later.
\end{itemize}

\subsection{Avogadro's Number and the Mole}

The \textbf{mole} is the SI unit for the amount of substance. It has been historically defined as the number of atoms in approximately 12 grams of carbon-12. This quantity is known as the Avogadro number, $N_A$,

$$N_A = 6.022 \times 10^{23} \text{ mol}^{-1}$$

\say{Every mol is $N_A$ things}. For instance, 3 mol of electrons is simply the quantity of $3N_A$ electrons.


\subsubsection{Molar Mass}

\textbf{Molar mass} is the mass of one mole of a substance, typically given in grams per mole (g/mol). The molar mass of a substance is numerically equal to the atomic mass of the substance in atomic mass units (u). E.g. water has a molar mass of $18.015 \text{ g/mol} \equiv 18.015 \text{ u}$, and the mass of a single water molecule is $\dfrac{18.015}{N_A}$.

\section{Gas Laws}

\subsection{Boyle's Law}

This states that the pressure of a gas is inversely proportional to its volume at constant temperature. Mathematically, this is
$$P \propto \frac{1}{V}$$

\img{boyle.png}{0.9}{Two graphs that arise from Boyle's Law}{boyle}
\begin{itemize}
  \item Every one of the curves in (a) is an isothermal curve.
  \item In the second graph
        \begin{enumerate}
          \item For a higher temperature and the same mass of gas, the curve is steeper but still linear
          \item For a higher mass of gas and the same temperature, the curve would also be steeper.
        \end{enumerate}
\end{itemize}

\subsection{Charles' Law}

This states that the \textbf{volume} of a gas is \textbf{directly proportional} to its \textbf{temperature} at \textbf{constant pressure}. Mathematically, this is
$$V\propto T$$

\subsection{Gay-Lussac's Law}

This states that the \textbf{pressure} of a gas is \textbf{directly proportional} to its \textbf{temperature} at \textbf{constant volume}. Mathematically, this is
$$P \propto T$$

\subsection{Avogadro's Law}

This states that the \textbf{volume} of a gas is \textbf{directly proportional} to the \textbf{quantity (in mol)} of the gas at \textbf{constant temperature and pressure}. Mathematically, this is
$$V \propto n$$

\subsection{The Ideal Gas Law}
Derived from the previous laws; they combine to give
$$PV = nRT \quad \text{ or } \quad R = \frac{PV}{nT}$$
where $R$ is the ideal gas constant, $8.31 \text{ J mol}^{-1} \text{ K}^{-1}$.

\section{Microscopic Model of Gases}

\section{Kinetic Model of Ideal Gases}

\subsection{Temperature}

\subsection{Ideal vs. Real Gases}

\end{document}