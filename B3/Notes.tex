\documentclass[a4paper,12pt]{article}
\usepackage{setspace}
\usepackage{sectsty}
\usepackage{siunitx}
\usepackage{graphicx}
\usepackage[a4paper, total={3in, 9in}, textwidth=16cm,bottom=1in,top=1.4in]{geometry}
\usepackage[dvipsnames]{xcolor}
\usepackage{amsmath}
\usepackage{esvect}
\usepackage{soul}
\usepackage{amsthm}
\usepackage{hyperref}
\usepackage{longtable}
\usepackage{float}
\usepackage{amssymb}
\usepackage{outlines}
\usepackage{caption}
\usepackage{fancyvrb}
\usepackage{subcaption}
\usepackage{esdiff}
\usepackage{dirtytalk}
\usepackage{colortbl}
\usepackage{booktabs}
\usepackage{setspace}
\usepackage{mathtools}
\usepackage{tikz,pgfplots}
\usepackage[most]{tcolorbox}
\usetikzlibrary{positioning,decorations.markings,arrows.meta,angles,quotes}
\DeclarePairedDelimiter{\ceil}{\lceil}{\rceil}
\newtheorem{lemma}{Lemma}
\newtheorem{proposition}{Proposition}
\newtheorem{remark}{Remark}
\newtheorem{observation}{Observation}
\doublespacing
\let\oldsection\section
\renewcommand\section{\clearpage\oldsection}
\newcommand{\RNum}[1]{\uppercase\expandafter{\romannumeral #1\relax}}
\let\oldsi\si
\renewcommand{\si}[1]{\oldsi[per-mode=reciprocal-positive-first]{#1}}
\usepackage{enumitem}
\newcommand{\subtitle}[1]{%
  \posttitle{%
    \par\end{center}
    \begin{center}\large#1\end{center}
    \vskip0.5em}%
}
\newcommand{\degsym}{^{\circ}}
\newcommand{\eqor}{\quad \text{or} \quad}
\newcommand{\Mod}[1]{\ (\mathrm{mod}\ #1)}
\usepackage{hyperref}
\hypersetup{
  colorlinks=true,
  linkcolor = blue
}
\newcommand{\lb}{\\[8pt]}
\newenvironment*{cell}[1][]{\begin{tabular}[c]{@{}c@{}}}{\end{tabular}}
\newcommand{\img}[4]{\begin{center}
  \begin{figure}[H]
    \centering
    \includegraphics[width=#2\textwidth]{#1}
    \caption{#3}
    \label{fig:#4}
  \end{figure}
\end{center}}
\parindent=0pt
\usepackage{fancyhdr}
\fancyfoot{}
\fancypagestyle{fancy}{\fancyfoot[R]{\vspace*{1.5\baselineskip}\thepage}}
\renewcommand{\contentsname}{Table of Contents}
\newcommand{\angled}[1]{\langle{#1}\rangle}
\newcommand{\paren}[1]{\left(#1\right)}
\newcommand{\sqb}[1]{\left[#1\right]}
\newcommand{\coord}[3]{\angled{#1,\, #2,\, #3}}
\newcommand{\pair}[2]{\paren{#1,\, #2}}
\newcommand{\atom}[3]{{}^{#1}_{#2}\text{#3}}
\usepackage[
  noabbrev,
  capitalise,
  nameinlink,
]{cleveref}

\crefname{lemma}{Lemma}{Lemmas}
\crefname{proposition}{Proposition}{Propositions}
\crefname{remark}{Remark}{Remarks}
\crefname{observation}{Observation}{Observations}

\newtcolorbox[auto counter]{prob}[2][]{fonttitle=\bfseries, title=\strut Problem~\thetcbcounter: #2,#1,colback=Orchid!5!white,colframe=Orchid!75!black,top=5mm,bottom=5mm}

\newtcolorbox[auto counter]{rem}[1][]{fonttitle=\bfseries, title=\strut Remark.~\thetcbcounter,colback=purple!5!white,colframe=purple!65!gray,top=5mm,bottom=5mm}

\newtcolorbox[auto counter]{defin}[1][]{fonttitle=\bfseries, title=\strut Definition.~\thetcbcounter,colback=black!5!white,colframe=black!65!gray,top=5mm,bottom=5mm}

\newtcolorbox[auto counter]{obs}[1][]{fonttitle=\bfseries, title=\strut Observation.~\thetcbcounter,colback=RedViolet!5!white,colframe=RedViolet!65!gray,top=5mm,bottom=5mm}

\newtcolorbox[auto counter]{lem}[1][]{fonttitle=\bfseries, title=\strut Lemma.~\thetcbcounter,colback=Maroon!5!white,colframe=Maroon!65!gray,top=5mm,bottom=5mm}

\newtcolorbox[auto counter]{prop}[1][]{fonttitle=\bfseries, title=\strut Proposition.~\thetcbcounter,colback=RedOrange!5!white,colframe=RedOrange!65!gray,top=5mm,bottom=5mm}

\newtcolorbox[auto counter]{hint}[1][]{fonttitle=\bfseries, title=\strut Hint.~\thetcbcounter,colback=OliveGreen!5!white,colframe=OliveGreen!75!gray,top=5mm,bottom=5mm}

\newcommand{\assref}[1]{\textcolor{orange!100!black!90}{assumption #1}}

\setlength{\belowcaptionskip}{-20pt}
\begin{document}


\pagenumbering{arabic}
\pagestyle{fancy}


\begin{titlepage}
  \begin{center}

    \vspace*{8cm}
    \textbf{\Large {IB Physics Topic B3 Gas Laws; SL \& HL}} \\
    \vspace*{1cm}
    \large{By timthedev07, M25 Cohort}

  \end{center}
\end{titlepage}

\pagebreak
\tableofcontents
\pagebreak

\clearpage
\setcounter{page}{1}
\addtocontents{toc}{\protect\thispagestyle{empty}}

\section{Pressure}

Pressure is defined as the force per unit area.
\begin{equation}\label{eq:pressure_def}
  P = \frac{F}{A}
\end{equation}
\begin{itemize}
  \item \textbf{Solid}: The pressure due to the weight $W$ over an area $A$ is given by
        \begin{equation}\label{eq:solid_pressure}
          P = \frac{W}{A}
        \end{equation}
  \item \textbf{Liquid}: The pressure in a liquid at a depth $h$ is given by
        \begin{equation}\label{eq:liquid_pressure}
          P = \rho gh
        \end{equation}
        where $\rho$ is the density of the liquid, $g$ is the acceleration due to gravity, and $h$ is the depth.
  \item \textbf{Gas}: Will be discussed later.
\end{itemize}

\subsection{Avogadro's Number and the Mole}

The \textbf{mole} is the SI unit for the amount of substance. It has been historically defined as the number of atoms in approximately 12 grams of carbon-12. This quantity is known as the Avogadro number, $N_A$,

$$N_A = 6.022 \times 10^{23} \text{ mol}^{-1}$$

\say{Every mol is $N_A$ things}. For instance, 3 mol of electrons is simply the quantity of $3N_A$ electrons.


\subsubsection{Molar Mass}

\textbf{Molar mass} is the mass of one mole of a substance, typically given in grams per mole (g/mol). The molar mass of a substance is numerically equal to the atomic mass of the substance in atomic mass units (u). E.g. water has a molar mass of $18.015 \text{ g/mol} \equiv 18.015 \text{ u}$, and the mass of a single water molecule is $\dfrac{18.015}{N_A}$.

\section{Gas Laws}

\subsection{Boyle's Law}

This states that the pressure of a gas is inversely proportional to its volume at constant temperature. Mathematically, this is
$$P \propto \frac{1}{V} \eqor P_1V_1 = P_2V_2$$

\img{boyle.png}{0.9}{Two graphs that arise from Boyle's Law}{boyle}
\begin{itemize}
  \item Every one of the curves in (a) is an isothermal curve.
  \item In the second graph
        \begin{enumerate}
          \item For a higher temperature and the same mass of gas, the curve is steeper but still linear
          \item For a higher mass of gas and the same temperature, the curve would also be steeper.
        \end{enumerate}
\end{itemize}

\subsection{Charles' Law}

This states that the \textbf{volume} of a gas is \textbf{directly proportional} to its \textbf{temperature} at \textbf{constant pressure}. Mathematically, this is
$$V\propto T \eqor \frac{V_1}{T_1} = \frac{V_2}{T_2}$$

\subsection{Gay-Lussac's Law}

This states that the \textbf{pressure} of a gas is \textbf{directly proportional} to its \textbf{temperature} at \textbf{constant volume}. Mathematically, this is
$$P \propto T  \eqor \frac{P_1}{T_1} = \frac{P_2}{T_2}$$

\subsection{Avogadro's Law}

This states that the \textbf{volume} of a gas is \textbf{directly proportional} to the \textbf{quantity (in mol)} of the gas at \textbf{constant temperature and pressure}. Mathematically, this is
$$V \propto n \eqor \frac{V_1}{n_1} = \frac{V_2}{n_2}$$

\pagebreak

\subsection{The Ideal Gas Law}
Derived from the previous laws; they combine to give
$$PV = nRT \quad \text{ or } \quad R = \frac{PV}{nT}$$
where $R$ is the ideal gas constant, $8.31 \text{ J mol}^{-1} \text{ K}^{-1}$.\lb
The relation can also be defined in terms of the \textbf{number of molecules} $N$ and the Boltzmann constant $k$ as
$$\frac{PV}{NT} = k_B \quad \text{ or } \quad pV = Nk_BT$$
where $k_B = \dfrac{R}{N_A}$ has value $1.38 \times 10^{-23} \text{ J K}^{-1}$.

\subsubsection{Proportionality Form}
In calculations, the following form is also often useful
$$\frac{P_1V_1}{N_1T_1} = \frac{P_2V_2}{N_2T_2}$$

\section{Brownian Motion}

\begin{itemize}
  \item \textbf{Definition}: Brownian motion refers to the unpredictable and irregular movement of microscopic particles suspended in a fluid (liquid or gas) caused by collisions with molecules of the fluid.
  \item \textbf{Collisions}: Brownian motion occurs because particles are constantly bombarded by surrounding molecules, whose motion is due to thermal energy. These molecular collisions are random in both direction and magnitude.
  \item \textbf{Temperature dependence}: The motion increases with temperature, as higher temperatures lead to more energetic collisions between fluid molecules and the suspended particles.
  \item \textbf{Continuous motion}: Any fluid above the temperature of absolute zero will exhibit Brownian motion, as the fluid molecules are always in motion because of their KE.
\end{itemize}

\section{Kinetic Model of Ideal Gases}

\subsection{Assumptions of the Model}

\begin{enumerate}
  \item All gas molecules are identical.
  \item Brownian motion constantly occurs
  \item The total volume of the gas is negligible compared to the volume of the container.
  \item The molecules collide elastically with each other and the walls of the container.
  \item The internal energy of the gas is entirely kinetic and does not include potential energy; intramolecular forces between the particles and the walls are negligible except during collisions.
  \item The time of collision is negligible compared to the time between collisions.
  \item External forces such as gravity are ignored.
\end{enumerate}

\subsection{The Model}

Consider a cubic container with length $L$ and volume $L^3$.\lb
Also consider a single particle with mass $m$ and $x$-velocity $v_x$ that hits the wall of the container at a right angle. The \textbf{impulse} is $-2mv_x$, and the \textit{average} force exerted on the wall is $F_x = \dfrac{\Delta p}{\Delta t} = -\dfrac{2mv_x}{T}$, where $T$ is the time taken for the collision and the particle to bounce back to the same contact point. This uses \assref{4}. Throughout $T$, the cube has traveled twice the length of the cube, so $T = \dfrac{2L}{v_x}$.\lb
The average force along the $x$-axis is then $$F_x = -\frac{2mv_x}{2L/v_x} = -\frac{mv_x^2}{L}$$
This analysis applies to any of the three components ($x$, $y$, or $z$), and if there are $N$ particles in the container, then, this analysis also applies to any of the $N$ particles.\lb
Now, we return to considering the $x$-dimension, without loss of generality. Let $\{v_{x_1}, v_{x_2}, \dotsc, v_{x_N}\}$ be the $x$-velocities of the particles, the total $x$-force is given by
\begin{align*}
  F_x = F_{x_1} + F_{x_2} + \dotsb + F_{x_N} & = \frac{m(v_{x_1}^2 + v_{x_2}^2 + \dotsb + v_{x_N}^2)}{L} \\
\end{align*}
We then take the \textbf{mean square speed} of the $x$-components across the $N$ particles $$\overline{v_x^2} = \frac{v_{x_1}^2 + v_{x_2}^2 + \dotsb + v_{x_N}^2}{N}$$
Hence, we obtain that the average force is given by $$\overline{F_x} = \frac{Nm}{L}\times\overline{v_x^2}$$
Finally, we now combine the three dimensions:
\begin{enumerate}
  \item The magnitude of a 3D vector $\vv{v}$ is given by $v = \sqrt{v_x^2 + v_y^2 + v_z^2}$.
  \item We can apply this to the mean square speed to get $\overline{v^2} = \overline{v_x^2} + \overline{v_y^2} + \overline{v_z^2}$; this represents the \textbf{mean square speed of the molecules}.
\end{enumerate}
We now make a further approximation based on the idea that the motion is random:
\begin{enumerate}
  \item The motion is said to be \textbf{isotropic}; there is no preferred direction
  \item We can then take $v_x^2 = v_y^2 = v_z^2$, this gives that $\overline{v^2} = 3\overline{v^2_x}$ and equivalently $\overline{v^2_x} = \dfrac{1}{3}\overline{v^2}$
\end{enumerate}
Substitution of this gives:
$$\overline{F} = \frac{Nm}{L}\times\frac{1}{3}\overline{v^2}$$
Since pressure is defined as the force per unit area, we divide by $L^2$
$$P = \frac{Nm}{3L^3}\times\overline{v^2} = \frac{Nm}{3V}\times\overline{v^2}$$
Also notice that $N \times m$ is the number of molecules $\times$ mass of each molecule; this is exactly the total mass of the gas, $M$. Then, $\dfrac{M}{V}$ is simply the density of the gas. Hence, we obtain
$$P = \frac{1}{3}\rho\overline{v^2}$$

\subsection{Temperature}

\subsection{Ideal vs. Real Gases}

\end{document}