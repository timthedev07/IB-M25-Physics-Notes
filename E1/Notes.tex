\documentclass[a4paper,12pt]{article}
\usepackage{setspace}
\usepackage{sectsty}
\usepackage{siunitx}
\usepackage{graphicx}
\usepackage[a4paper, total={3in, 9in}, textwidth=16cm,bottom=1in,top=1.4in]{geometry}
\usepackage[dvipsnames]{xcolor}
\usepackage{amsmath}
\usepackage{esvect}
\usepackage{soul}
\usepackage{amsthm}
\usepackage{svg}
\usepackage{hyperref}
\usepackage{float}
\usepackage{amssymb}
\usepackage{outlines}
\usepackage{caption}
\usepackage{fancyvrb}
\usepackage{subcaption}
\usepackage{esdiff}
\usepackage{setspace}
\usepackage{mathtools}
\usepackage{tikz,pgfplots}
\usepackage[most]{tcolorbox}
\usetikzlibrary{positioning,decorations.markings,calc}
\DeclarePairedDelimiter{\ceil}{\lceil}{\rceil}
\newtheorem{lemma}{Lemma}
\newtheorem{proposition}{Proposition}
\newtheorem{remark}{Remark}
\newtheorem{observation}{Observation}
\doublespacing
\let\oldsection\section
\renewcommand\section{\clearpage\oldsection}
\newcommand{\RNum}[1]{\uppercase\expandafter{\romannumeral #1\relax}}
\let\oldsi\si
\renewcommand{\si}[1]{\oldsi[per-mode=reciprocal-positive-first]{#1}}
\usepackage{enumitem}
\newcommand{\subtitle}[1]{%
  \posttitle{%
    \par\end{center}
    \begin{center}\large#1\end{center}
    \vskip0.5em}%
}
\newcommand{\degsym}{^{\circ}}
\newcommand{\Mod}[1]{\ (\mathrm{mod}\ #1)}
\usepackage{hyperref}
\hypersetup{
  colorlinks=true,
  linkcolor = blue
}
\newcommand{\lb}{\\[8pt]}
\newenvironment*{cell}[1][]{\begin{tabular}[c]{@{}c@{}}}{\end{tabular}}
\newcommand{\img}[4]{\begin{center}
  \begin{figure}[H]
    \centering
    \includegraphics[width=#2\textwidth]{#1}
    \caption{#3}
    \label{fig:#4}
  \end{figure}
\end{center}}
\parindent=0pt
\newcommand{\doubleimg}[4]{\begin{center}
  \begin{figure}[H]
    \centering
    \begin{subfigure}{.45\textwidth}
      \centering
      \includegraphics[width=1\linewidth]{#1}
      \caption{#2}
      \label{fig:sub1}
    \end{subfigure}
    \begin{subfigure}{.45\textwidth}
      \centering
      \includegraphics[width=1\linewidth]{#3}
      \caption{#4}
      \label{fig:sub2}
    \end{subfigure}
  \end{figure}
\end{center}}
\usepackage{fancyhdr}
\fancyfoot{}
\newcommand{\vect}[3]{\begin{bmatrix}
  #1 \\
  #2 \\
  #3
\end{bmatrix}}
\fancypagestyle{fancy}{\fancyfoot[R]{\vspace*{1.5\baselineskip}\thepage}}
\renewcommand{\contentsname}{Table of Contents}
\newcommand{\angled}[1]{\langle{#1}\rangle}
\newcommand{\paren}[1]{\left(#1\right)}
\newcommand{\sqb}[1]{\left[#1\right]}
\newcommand{\coord}[3]{\angled{#1,\, #2,\, #3}}
\newcommand{\pair}[2]{\paren{#1,\, #2}}
\usepackage[
  noabbrev,
  capitalise,
  nameinlink,
]{cleveref}
\crefname{lemma}{Lemma}{Lemmas}
\crefname{proposition}{Proposition}{Propositions}
\crefname{remark}{Remark}{Remarks}
\crefname{observation}{Observation}{Observations}

\newtcolorbox[auto counter]{prob}[2][]{fonttitle=\bfseries, title=\strut Problem~\thetcbcounter: #2,#1,colback=Orchid!5!white,colframe=Orchid!75!black,top=5mm,bottom=5mm}

\newtcolorbox[auto counter]{rem}[1][]{fonttitle=\bfseries, title=\strut Remark.~\thetcbcounter,colback=purple!5!white,colframe=purple!65!gray,top=5mm,bottom=5mm}

\newtcolorbox[auto counter]{defin}[1][]{fonttitle=\bfseries, title=\strut Definition.~\thetcbcounter,colback=black!5!white,colframe=black!65!gray,top=5mm,bottom=5mm}

\newtcolorbox[auto counter]{obs}[1][]{fonttitle=\bfseries, title=\strut Observation.~\thetcbcounter,colback=RedViolet!5!white,colframe=RedViolet!65!gray,top=5mm,bottom=5mm}

\newtcolorbox[auto counter]{lem}[1][]{fonttitle=\bfseries, title=\strut Lemma.~\thetcbcounter,colback=Maroon!5!white,colframe=Maroon!65!gray,top=5mm,bottom=5mm}

\newtcolorbox[auto counter]{prop}[1][]{fonttitle=\bfseries, title=\strut Proposition.~\thetcbcounter,colback=RedOrange!5!white,colframe=RedOrange!65!gray,top=5mm,bottom=5mm}

\newtcolorbox[auto counter]{hint}[1][]{fonttitle=\bfseries, title=\strut Hint.~\thetcbcounter,colback=OliveGreen!5!white,colframe=OliveGreen!75!gray,top=5mm,bottom=5mm}

\setlength{\belowcaptionskip}{-20pt}
\begin{document}


\pagenumbering{arabic}
\pagestyle{fancy}


\begin{titlepage}
  \begin{center}

    \vspace*{8cm}
    \textbf{\Large {IB Physics Topic E1 Atomic Structure; SL \& HL}} \\
    \vspace*{1cm}
    \large{By timthedev07, M25 Cohort}


  \end{center}
\end{titlepage}

\pagebreak
\tableofcontents
\pagebreak

\clearpage
\setcounter{page}{1}
\addtocontents{toc}{\protect\thispagestyle{empty}}

\section{Nuclear Notation and Structure}
\begin{itemize}
  \item An atom has 0 net charge; the number of protons equals the number of electrons.
  \item An anion is negatively charged (with an excess of electrons)
  \item A cation is positively charged (with a deficit of electrons)
\end{itemize}

\img{atomicNotation.png}{0.4}{The atomic notation}{atomicnotation}
\begin{itemize}
  \item Let $p = $ number of protons, $n = $ number of neutrons, $e = $ number of electrons.
  \item The nucleon number is the number of protons and neutrons, i.e. $A = p + n$
  \item The proton number is $Z = p$
  \item From this representation, one can work out the neutron number as $n = A - Z$
  \item In this notation,
        \begin{itemize}
          \item proton $\to$ $^1_1p$
          \item neutron $\to$ $^1_0n$
          \item electron $\to$ $^0_{-1}e$
        \end{itemize}
\end{itemize}

\pagebreak

\section{The Plum Pudding Model}
This was first proposed by J.J. Thomson in 1897. It was based on the idea that the atom was a sphere of positive charge with electrons embedded in it, with an even distribution and uniform density.

\img{plumpudding.png}{0.4}{The plum pudding model}{plumpudding}

\section{The Geiger-Marsden Experiment}
An experiment carried out in 1911 to determine the structure of an atom, this disproved the plum pudding model.\lb
The experiment is directed by Rutherford but carried out by Geiger and Marsden. The experiment involved firing alpha particles at a thin gold foil and observing the scattering pattern. The gold foil is made thin so that the alpha particles are not subject to the interference of many atoms --- in fact, scientists believed that the thin foil was only two lines of atoms thick.

\img{goldfoilexpt.png}{0.7}{The Geiger-Marsden-Rutherford experiment}{goldfoil}

If the plum pudding model were correct, only one of the following would have happened:
\begin{itemize}
  \item Case 1: if the gold atoms have \textbf{low density}, all the alpha particles would have effortlessly passed through the foil with minimal deflection.
  \item Case 2: if the gold atoms have \textbf{high density}, all the alpha particles would have been bounded back or deflected at large angles.
\end{itemize}

\pagebreak

What actually happened was that most of the alpha particles passed through the foil with minimal deflection, but some were deflected at large angles, and a few were bounded back. This offers the following implications:
\begin{itemize}
  \item most of the atom is a small and dense region
  \item the atom contains small dense regions of positive electric charge
\end{itemize}

\subsection{Deviations from the Scattering}

\begin{defin}
  The distance of closest approach of an alpha particle is the minimum distance it gets to the center of the nucleus of an atom before being repelled by the electrostatic (Coulomb) force.

  At high initial KE of the alpha particle $E_\alpha$, the closest approach distance $r_c$ becomes small.
\end{defin}

For alpha particles with initial energy greater than 28 $\si{\mega\eV}$, the scattering pattern observed by Geiger and Marsden was no longer true. The pattern assumed that the alpha particles only interact through electrostatic repulsion. Beyond that limit, the alpha particles will be close enough to the gold nucleus to also interact with the strong nuclear force. This then provides \textbf{evidence for the strong nuclear force}.\lb

If $E_{\alpha, \max} \approx 28\si{\mega\eV}$ is the maximum energy for which the scattering pattern still obeys the Rutherford model, then, the estimate gives the smallest $r_c$ at which other nuclear forces to not operate --- this is the effective size of a nucleus.

\section{Distance of Closest Approach and Energy}

\begin{enumerate}
  \item As an alpha particle approaches the nucleus, WLOG, to the right, it will start to decelerate due to the repulsive electrostatic force and start to lose its kinetic energy.
  \item At some point, the alpha particle will come to a stop; this is when all of its kinetic energy has been converted to electric potential energy. Therefore
        $$E_\alpha = \frac{1}{2}m_\alpha v_\alpha^2 = k\frac{q_\alpha Q_\text{gold}}{r_c}$$
        \img{doca.png}{0.9}{The distance of closest approach}{doca}
        \begin{itemize}
          \item The charge of the gold nucleus is $Q_\text{gold} = Z\times e$, where $Z$ is the atomic number of gold and $e$ is the elementary charge.
          \item The charge of the alpha particle is $q_\alpha = 2e$.
        \end{itemize}
  \item Rearranging give
        \begin{align*}
          r_c & = \frac{kq_\alpha Q_\text{gold}}{E_\alpha} \\
              & = \frac{2kZe^2}{E_\alpha}
        \end{align*}
\end{enumerate}

\section{Nuclear Density}

The volume of an atom is directly proportional to the nucleon number of the atom
$$V \propto A$$
assuming the volume is a sphere, we then have that
$$R^3 \propto A \iff R \propto A^{\frac{1}{3}}$$
The constant of proportionality is the Fermi radius $R_0 = 1.2 \times 10^{-15}\si{\meter}$, and thus
$$R = R_0\sqrt[3]{A}$$

We can form an expression for the volume:
$$V_{\text{nucleus}} = \frac{4}{3}\pi A R_0^3$$
Assuming a proton and a neutron have roughly the same mass of $u = 1.66 \times 10^{-27}\si{\kilo\gram}$ (this is the \textbf{unified atomic mass unit}), we can find the total mass of the nucleus:
$$M_{\text{nucleus}} = A \times u$$
The nuclear density is hence given by
$$\rho_{\text{nucleus}} = \frac{M_{\text{nucleus}}}{V_{\text{nucleus}}} = \frac{A \times u}{\frac{4}{3}\pi A R_0^3} = \frac{3u}{4\pi R_0^3}$$
A quick substitution gives the nuclear density as $\rho_{\text{nucleus}} = 2.4 \times 10^{17}$ $\si{\kilo\gram\per\meter\cubed}$. This is the same for all nuclides.


\section{Emission and Absorption Spectra}

\subsection{Energy Levels of Electrons}

\img{energylevel.jpg}{0.5}{Energy levels of electrons}{energylevel}

Electrons exist at \textbf{discrete energy levels} in an atom; this can be evidenced by the emission and absorption spectra of an atom.
\begin{itemize}
  \item The ground state is the energy level at which the electron normally resides, this is the lowest energy level (max. negative).
  \item There is a limit of maximum energy level, beyond which, the electron is \textbf{ionized} and no longer part of the atom.
  \item When an electron exists at another energy level other than the ground state, it is said to be \textbf{excited}.
  \item An electron can be excited by:
        \begin{itemize}
          \item absorbing a photon of energy
          \item colliding with a nearby particle
        \end{itemize}
\end{itemize}

Electrons do not remain excited for long, they will eventually relax and return to their ground state by emitting a photon of energy. The energy of the photon is equal to the difference in energy levels between the excited state and the ground state.

\subsection{Emitting and Absorbing Photons}

The energy of the photon emitted is given by $$E = hf$$where $h$ is Planck's constant. The frequency of the photon is given by the equation $$f = \frac{c}{\lambda}$$where $c$ is the speed of light and $\lambda$ is the wavelength of the photon.

As previously mentioned, \textbf{the energy levels of electrons are discrete/quantized}, i.e. an electron cannot sit between two energy levels. This means that the energy of the photon emitted/absorbed is also discrete.
\begin{itemize}
  \item If a photon whose energy does not match any of the energy differences between the discrete levels passes through the atom, \textbf{it will not be absorbed}.
  \item If we plot the wavelength of the emitted photons, we get an emission spectrum with a \textbf{black background and colored lines}.
\end{itemize}

\subsubsection{Photons}

Discrete packets of energy. The energy $E$ is given by $$E = hf$$where $h$ is Planck's constant. I.e., the energy in a photon is proportional to its frequency.

\pagebreak

\subsection{The Spectra}

There is a one-to-one correspondence between every element and a spectrum --- each element has a unique spectrum and each spectrum can uniquely determine an element. In other words, the spectrum of an element is its fingerprint that provides information about the chemical composition of an element.

\subsubsection{The Absorption Spectrum}

White light contains all wavelengths of light, and so it will occupy the entire spectrum.

\img{visiblelight.png}{0.4}{The visible light spectrum}{visiblelight}

If we shine the light through a gas of the element that we are studying, the gas will absorb photons of specific energy levels and hence wavelengths. Since they are observed, they will be missing from the observed spectrum.

\img{spectracompare.jpg}{0.5}{The emission and absorption spectra}{spectracompare}

\subsection{Caveat with Energy Level Changes}

Each energy change only occurs between a unique pair of levels in an atom, and doesn't occur twice in the same atom. This means that each line on an absorption or emission spectrum is associated with a unique pair of energy levels of the electron.




\section{The Bohr Model}

\subsection{Problem with the Rutherford Model \& Classical Physics}

\begin{itemize}
  \item Classical electromagnetic theory predicted that accelerating charged particles, like electrons in orbit, should lose energy by emitting electromagnetic radiation.
  \item If this were true, the electrons would lose energy, \textbf{spiral into the nucleus}, and atoms would collapse, which contradicted experimental observations, whereby atoms are stable and do not collapse.
\end{itemize}

\subsection{Bohr's Solution}

Bohr's atomic model has the property where electrons orbit the nucleus in \textbf{circular orbits} at fixed radii. These orbits are \textbf{quantized}, i.e. the electron can only exist in certain orbits. These orbits are called \textbf{stationary states}, where the electron does not emit radiation, hence does not lose energy and spiral into the nucleus.\lb
One of the limitations and reasons why this model does not hold is that it is \textbf{only applicable to hydrogen-like atoms}, i.e. atoms with only one electron. It fails to predict the atomic behavior for atoms with more than one electron in orbit.

\subsection{Energy Levels}

The energy levels of the electron in the Bohr model are quantized, i.e. the electron can only exist in certain energy levels. These levels are labeled by the principal quantum number $n \in \mathbb{Z}^+$, where $n = 1$ is the ground state, and $n = 2, 3, 4, \ldots$ are the excited states.

\begin{figure}[H]
  \centering
  \includesvg[width=0.5\textwidth]{n}
\end{figure}

The total (kinetic and electric potential) energy in an electron at the $n$th energy level is given by $$E = -\frac{13.6}{n^2}\text{ } \si{\eV}$$

\pagebreak

\subsection{Quantized Angular Momentum}

The \textbf{de Broglie wavelength} of a charged particle is given by \begin{equation}
  \lambda = \frac{h}{p} = \frac{h}{mv}
\end{equation} where $h$ is Planck's constant and $p$ is the momentum of the particle.

One of the assumptions of Bohr that the angular momentum of an electron in a stationary state is \textbf{quantized} led to the implication that each orbital circumference is an integer multiple of the de Broglie wavelength of the electron "joined up". This means that, at the $n$th level, the orbital circumference is \begin{equation}
  n\lambda = 2\pi r
\end{equation}
Combining (1) and (2), we get the following expression for the angular momentum\begin{equation}
  L = mvr = \frac{nh}{2\pi}
\end{equation} where $m$ is the mass of the electron, $v$ is the velocity of the electron, and $r$ is the radius of the orbit.
This quantization has the following implications
\begin{itemize}
  \item The orbitals must have the nature of standing waves.
  \item Confirms the quantization of energy.
\end{itemize}


\pagebreak

\end{document}