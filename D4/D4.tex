\documentclass[a4paper,12pt]{article}
\usepackage{setspace}
\usepackage{sectsty}
\usepackage{siunitx}
\usepackage{graphicx}
\usepackage[a4paper, total={3in, 9in}, textwidth=16cm,bottom=1in,top=1.4in]{geometry}
\usepackage[dvipsnames]{xcolor}
\usepackage{amsmath}
\usepackage{esvect}
\usepackage{soul}
\usepackage{amsthm}
\usepackage{hyperref}
\usepackage{float}
\usepackage{amssymb}
\usepackage{outlines}
\usepackage{caption}
\usepackage{fancyvrb}
\usepackage{subcaption}
\usepackage{esdiff}
\usepackage{setspace}
\usepackage{mathtools}
\usepackage{tikz,pgfplots}
\usepackage[most]{tcolorbox}
\usepackage{draftwatermark}
\usepackage{helvet}
\renewcommand{\familydefault}{\sfdefault}
\SetWatermarkText{timthedev07}
\SetWatermarkScale{4}
\SetWatermarkColor[gray]{0.97}
\usetikzlibrary{positioning,decorations.markings,calc}
\DeclarePairedDelimiter{\ceil}{\lceil}{\rceil}
\newtheorem{lemma}{Lemma}
\newtheorem{proposition}{Proposition}
\newtheorem{remark}{Remark}
\newtheorem{observation}{Observation}
\doublespacing
\let\oldsection\section
\renewcommand\section{\clearpage\oldsection}
\newcommand{\RNum}[1]{\uppercase\expandafter{\romannumeral #1\relax}}
\let\oldsi\si
\renewcommand{\si}[1]{\oldsi[per-mode=reciprocal-positive-first]{#1}}
\usepackage{enumitem}
\newcommand{\subtitle}[1]{%
  \posttitle{%
    \par\end{center}
    \begin{center}\large#1\end{center}
    \vskip0.5em}%
}
\newcommand{\degsym}{^{\circ}}
\newcommand{\Mod}[1]{\ (\mathrm{mod}\ #1)}
\usepackage{hyperref}
\hypersetup{
  colorlinks=true,
  linkcolor = blue
}
\newcommand{\lb}{\\[8pt]}
\newenvironment*{cell}[1][]{\begin{tabular}[c]{@{}c@{}}}{\end{tabular}}
\newcommand{\img}[4]{\begin{center}
  \begin{figure}[H]
    \centering
    \includegraphics[width=#2\textwidth]{#1}
    \caption{#3}
    \label{fig:#4}
  \end{figure}
\end{center}}
\parindent=0pt
\usepackage{fancyhdr}
\fancyfoot{}
\newcommand{\vect}[3]{\begin{bmatrix}
  #1 \\
  #2 \\
  #3
\end{bmatrix}}
\fancypagestyle{fancy}{\fancyfoot[R]{\vspace*{1.5\baselineskip}\thepage}}
\renewcommand{\contentsname}{Table of Contents}
\newcommand{\angled}[1]{\langle{#1}\rangle}
\newcommand{\paren}[1]{\left(#1\right)}
\newcommand{\sqb}[1]{\left[#1\right]}
\newcommand{\coord}[3]{\angled{#1,\, #2,\, #3}}
\newcommand{\pair}[2]{\paren{#1,\, #2}}
\usepackage[
  noabbrev,
  capitalise,
  nameinlink,
]{cleveref}
\crefname{lemma}{Lemma}{Lemmas}
\crefname{proposition}{Proposition}{Propositions}
\crefname{remark}{Remark}{Remarks}
\crefname{observation}{Observation}{Observations}

\newtcolorbox[auto counter]{defin}[1][]{fonttitle=\bfseries, title=\strut Definition.~\thetcbcounter,colback=black!5!white,colframe=black!65!gray,top=5mm,bottom=5mm}

\newtcolorbox[auto counter]{obs}[1][]{fonttitle=\bfseries, title=\strut Observation.~\thetcbcounter,colback=RedViolet!5!white,colframe=RedViolet!65!gray,top=5mm,bottom=5mm}

\setlength{\belowcaptionskip}{-20pt}

\begin{document}


\pagenumbering{arabic}
\pagestyle{fancy}


\begin{titlepage}
  \begin{center}

    \vspace*{8cm}
    \textbf{\Large {IB Physics Topic D4 Electromagnetic Induction; HL}} \\
    \vspace*{1cm}
    \large{By timthedev07, M25 Cohort}


  \end{center}
\end{titlepage}

\pagebreak
\tableofcontents
\pagebreak

\clearpage
\setcounter{page}{1}
\addtocontents{toc}{\protect\thispagestyle{empty}}

\section{Fleming's Left and Right Hand Rules}

The left-hand rule is for the motor effect; the right-hand equivalent is for the generator effect.

Induction and motor effects are intertwined and must co-occur. Consider a bar moving through a magnetic field. The motion would induce a current in the rod, which is explained by the left-hand rule that suggests that electrons feel a force and thus there is a current. The induced current would lead to a motor effect, creating a force that opposes the bar's motion. Thus there is a deceleration, if the system is isolated.

\section{The Generator Effect}

When there is a relative motion between a conductor and a magnetic field, a current/$\varepsilon$ is induced in the conductor. Example scenarios include:
\begin{itemize}
  \item Moving a magnet away from or into a coil.
  \item Moving a bar of conductor in a magnetic field
  \item Moving a charge in a magnetic field
\end{itemize}

\subsection{Lenz's Law}

The induced magnetic current in a coil by inserting a magnet into the coil must oppose the motion creating the current. In other words, an insertion must be north-to-north or south-to-south; conversely, if we are pulling the magnet away from the coil, the facing poles should be opposite and attractive.\lb
The induced current is continuous only if there is a complete circuit for charge to flow.\lb
This effect law is explained by the \hl{conservation of energy}. Consider dropping a bar magnet vertically through a coil: The magnetic force created by cutting field lines should oppose the falling motion of the magnet both at the top and the bottom of the coil to decelerate/resist its downward motion, otherwise, we would have a system where the energy output is greater than the energy input, thereby creating energy. This is impossible and disobeys the conservation of energy.
\pagebreak

\subsection{Right-hand Grip}

This rule determines where the north-pole of a solenoid/coil is. The direction in which the fingers curl is the direction of current, and the thumb points to the north-pole.

\img{righthandgrip.jpg}{0.65}{Right Hand Grip}{rhgrip}
\pagebreak

\subsection{Explaining the Generator Effect}

Consider a conducting bar moving through a magnetic field.
\img{bar.png}{0.6}{A conductor moving in a uniform magnetic directed into the page}{bar}
Using Flemming's left-hand rule to analyze \textbf{the motion of the electrons}, we can see that there is a (conventional) current to the left, since the electrons are pushed to the right. There is now a p.d. between L and R, with L at a higher potential.\lb
As the negative charges accumulate, another force, opposing the force pushing the electrons to R, arises --- the electric force. Like charges repel, and hence, as the electrons accumulate, the electric force increases, opposing the motion of the electrons. Eventually, the two forces balance out.
\begin{itemize}
  \item The force pushing the electrons to R is the magnetic force, given by $F_B = Bqv$
  \item The opposing electric force is given by $F_e = Eq$, where $V = \varepsilon$ the induced \textbf{emf}; by $E = \dfrac{V}{d}$, we have $F_e = \dfrac{\varepsilon q}{d}$, where $d$ is, in fact, the length of the rod. Let's denote that as $l$ and hence $F_e = \dfrac{\varepsilon q}{l}$
  \item Equating the two gives the following, which allows us to find the \textbf{induced emf}
        \begin{align*}
          Bv = \frac{\varepsilon}{l} \\
          \varepsilon = Blv
        \end{align*}
\end{itemize}


\subsection{External Force to Achieve Constant Velocity}

Suppose, we have a conducting bar rolling (without friction) on a pair of parallel conducting rails.

\begin{minipage}{0.5\textwidth}
  \img{rolling1.png}{1}{3D View}{rollingbar1}
\end{minipage}%
\begin{minipage}{0.5\textwidth}
  \img{rolling2.png}{1}{Top View}{rollingbar2}
\end{minipage}

There must be an external force opposing the magnetic force causing the rod to roll to the right. This force is given by $F_{\text{ext}} = BIL$, where $I$ is the current in the rod. This force is given by $F_B=BIL$ because it is the negative equivalent of $F_B$.

If we are considering the rate at which work is done to oppose the motion to the right then it's $F_Bv$ using $P = Fv$ or equivalently $P = I\varepsilon$, using the circuit equation.

\subsection{Deriving the Induced emf Using Energy}

We know $\varepsilon = \frac{W}{Q}$, thus
\bgroup
\addtolength{\jot}{1em}
\begin{align*}
  \varepsilon & = \frac{\text{work done by the magnetic force}}{\text{total charge flowing}} \\
              & = \frac{F_B\times vt}{It}                                                    \\
              & = \frac{BILvt}{It}                                                           \\
  \varepsilon & = Blv
\end{align*}
\egroup

\subsection{Area Swept Out by the Rod}

Consider the area swept out by the rod, which is given by $A = Lvt$ and $\frac{\Delta A}{\Delta t} = Lv$

\img{rollingarea.png}{0.6}{The area swept by the rod}{rollingarea}
we can incorporate this into our derivation of the induced emf and get
\begin{equation}\label{eq:emf_area}
  \varepsilon = B\times \frac{\Delta A}{\Delta t} = B \times \text{ rate of change of area}
\end{equation}

\pagebreak

\section{Magnetic Flux}

The \textbf{magnetic flux} is given by $$\Phi = BA\cos\theta$$where $B$ is the magnetic field strength, $A$ is the area of the coil, and $\theta$ is the angle between the magnetic field and the normal to the surface. The quantity $\Phi$ has unit webers (Wb).

\img{fluxdensity.png}{1}{Magnetic Flux Density}{fluxdensity}

The \textbf{magnetic flux density} is numerically analogous to the electric field strength; both quantities are denoted by $B$. It represents the number of field lines per unit area. Thus, the magnetic flux is the number of field lines passing through a surface of area $A$.\lb
This also allows us to rewrite \cref{eq:emf_area} as $$\varepsilon = \frac{\Delta \Phi}{\Delta t}$$
and we can then define the unit of flux, a weber, as \textbf{the flux that produces an emf of 1 volt per second}.
Also, one tesla is defined as 1 weber per square meter.
$$1\si{\tesla}\equiv 1\si{\weber\per\m\squared}$$

\pagebreak

\subsection{Magnetic Flux Linkage}

Previously, we considered a single rod rolling along two rails. Now, let's consider a coil of $N$ turns, each of area $A$, in a magnetic field. The magnetic flux linkage is given by $$\Lambda = N\Phi = BAN$$
the unit of this quantity is also the weber, or weber-turns.\lb
In this case, the induced emf is given by $$\varepsilon = \frac{\Delta \Lambda}{\Delta t} = N\frac{\Delta \Phi}{\Delta t} = N\frac{B\Delta A}{\Delta t}$$

\subsection{Faraday's Law}

Faraday's law states that the induced emf is directly proportional to the rate of change of magnetic flux linkage. This is given by $$\varepsilon = -N\frac{\Delta \Phi}{\Delta t}$$
which is the Neumann's equation that includes both the ideas of Lenz and Faraday.\lb
It suggests that the magnetic flux can be changed by changing at least one of the following quantities:
\begin{itemize}
  \item $\dfrac{\Delta A}{\Delta t}$
  \item $\dfrac{\Delta \cos \theta}{\Delta t}$
  \item $\dfrac{\Delta B}{\Delta t}$
\end{itemize}

\section{Relative Motion Between Coil \& Field}

\subsection{Coil Remains Within a Uniform Field}

This is when a coil moves at a \textbf{constant speed} from one position to another \textbf{completely within a uniform magnetic field}. In this case, there is \textbf{no change in flux linkage}, because the same number of field lines are being cut on opposite sides of the coil, where the current is in opposite directions.

\subsection{Coil Moves Out of A Uniform Field}

When a coil of $N$ turns moves from a position where the flux is $\Phi$ to a position where the flux is 0, the change in flux linkage is $-N\Phi$, and thus, the induced emf is given by $$\varepsilon = -\frac{N\Phi}{\Delta t}$$

\subsection{Rotation of a Coil in a Uniform Field}

When the coil is rotated by $180\degsym$ in a uniform field, the change in flux linkage is $2N\Phi$, and thus, the induced emf is given by $$\varepsilon = -\frac{2N\Phi}{\Delta t}$$because the field lines reverse their direction.

\pagebreak

\subsection{Changing Magnetic Field}

The coil remains stationary, but the magnetic field cutting through it changes. In this case, the induced emf is given by $$\varepsilon = -NA\frac{\Delta B}{\Delta t}$$We do not need to know the exact field strength, only the \textbf{rate of change of the field strength} is sufficient.

There are two graphs that can be drawn.

\begin{minipage}{0.5\textwidth}
  \img{g1.png}{1}{Flux Linkage vs. Time}{g1}
\end{minipage}\begin{minipage}{0.5\textwidth}
  \img{g2.png}{1}{emf vs. Time}{g1}
\end{minipage}

\begin{itemize}
  \item The gradient of the first graph is the emf
  \item The area under the second graph is the total change in flux linkage.
\end{itemize}


\section{Generators}
Key components of an AC generator include:
\begin{itemize}
  \item A coil of wire rotating in a magnetic field
  \item A magnetic field (e.g. from a bar magnet)
  \item Relative movement between the coil and the field
  \item A suitable connection to the static circuit outside the generator
\end{itemize}

\img{generator.jpg}{1}{A simple generator}{generator}

Relationship between emf and flux linkage:

\begin{itemize}
  \item The induced emf is the rate of change of magnetic flux linkage.
  \item Since $\Phi = BA\cos\theta$, the magnetic flux linkage is maximum when $\theta = 0$, e.g. when the coil is perpendicular to the field. However, this is the moment at which the induced emf is 0; consider the taking the partial derivative of $\Phi$ with respect to $\theta$ --- $\varepsilon$ will involve $\sin\theta$.
  \item Conversely, the magnetic flux linkage is minimum when $\theta = 90\degsym$, e.g. when the coil is parallel to the field. This is the moment at which the induced emf is maximum.

\end{itemize}

Effect of increasing the angular speed of the coil:
\begin{itemize}
  \item Frequency (cycles per second) increases
  \item This squashes the graph, increasing the rate of change of linkage with respect to time, thus increasing the peak emf.
\end{itemize}

Other ways of increasing the induced emf
\begin{itemize}
  \item Increasing flux density
  \item Increasing the number of turns on the coil
  \item Increasing the coil area
\end{itemize}

The rule of thumb of the generator is that
\begin{itemize}
  \item When the coil is perpendicular to the field, the emf is 0 and the flux linkage is maximum.
  \item When the coil is parallel to the field, the emf is maximum and the flux linkage is 0.
\end{itemize}

\subsection{Power}

The voltage (induced emf) and current graphs are both sinusoidal, with different vertical scaling factors but the same zeroes. Since power is calculated by $P = IV$, the power graph will be a \textbf{sine squared graph}.
\begin{itemize}
  \item The mean power is given by $V_\text{rms}I_\text{rms} = \dfrac{1}{2}V_{\text{max}}I_{\text{max}}$
  \item $I_{\text{rms}} = \dfrac{I_{\text{max}}}{\sqrt{2}}$ (root mean square)
  \item $V_{\text{rms}} = \dfrac{V_{\text{max}}}{\sqrt{2}}$ (root mean square)
\end{itemize}

\subsection{Effect of Changing the Frequency}

If the frequency of the generator is doubled with no other changes being made, then
\begin{itemize}
  \item The peak emf doubles.
  \item The time period halves.
\end{itemize}

\section{Mutual and Self-Induction}

\subsection{Mutual Induction}

This is the process by which a changing current in one coil induces an emf in another coil nearby. The emf induced is given by $$\varepsilon = -M\frac{\Delta I_1}{\Delta t}$$where $M$ is the mutual inductance of the two coils. The unit of mutual inductance is the henry (H).
\begin{itemize}
  \item When $I_1$ is increasing or decreasing, there is a change in the magnetic field and thus an induced emf.
  \item When $I_1$ reaches a constant level and stops changing, the induced emf is 0.
  \item This is because the presence of an induced emf requires a \textbf{non-zero rate of change in the magnetic flux linkage}.
  \item By Lenz's Law, the induced emf/current \textbf{opposes the change in $I_1$} (not $I_1$). I.e., if the current is increasing, the induced current is in the opposite direction; if the current is decreasing, the induced current is in the same direction.
\end{itemize}
This is the effect behind transformers.

\subsection{Self-Induction}

Occurrence of the phenomenon:
\begin{itemize}
  \item Current starts flowing through a wire or coil.
        As the current increases, a magnetic field grows around the wire.
  \item But when the current changes, the magnetic field changes too. According to Faraday's Law, a changing magnetic field induces a voltage.
  \item This newly induced voltage, however, opposes the very change that created it (by Lenz's Law). It is as if the coil resists the current's attempt to speed up or slow down. This resistance to change is what we call self-induction.
\end{itemize}


\section{Exam Questions}

\subsection{Terminal Velocity}
\img{ex/droppedloop.png}{0.45}{Dropped loop}{droppedloop}
\begin{quote}
  A vertical rectangular loop of conducting wire is dropped in a region of horizontal magnetic field. The diagram shows the loop as it leaves the region of the magnetic field.
\end{quote}
(i) Explain, by reference to Faraday's law of electromagnetic induction, why there is an electromotive force (emf) induced in the loop as it leaves the region of magnetic field.
\begin{itemize}
  \item First simply state the law: \hl{Faraday's states that the induced emf is proportional to the rate of change of magnetic flux linkage}.
  \item Now we must explain why there is a change in the flux linkage $N\Phi = BAN$; we must identify the quantity that is changing that makes the linkage change. In this case when the loop leaves the region, the area $A$ interacting with the field decreases.
\end{itemize}
(ii) Just before the loop is about to completely exit the region of magnetic field, the loop moves with constant terminal speed $v$. The following data is available; find $v$.
\begin{table}[H]
  \centering
  \begin{tabular}{|c|c|}
    \hline
    Mass of loop          & $m = 4.0$g                                       \\
    Resistance of loop    & $R = \SI{25}{\milli\ohm}$                        \\
    Width of loop         & $L = \SI{15}{\centi\metre}$                      \\
    Magnetic flux density & $                         B = \SI{0.80}{\tesla}$ \\
    \hline
  \end{tabular}
\end{table}

\begin{enumerate}
  \item The terminal velocity is reached when forces are balanced. Let's identify the two opposing force that must be equal in magnitude in this case:
        \begin{itemize}
          \item The downward force is due to gravity, $w = mg$.
          \item The upward force is due to the magnetic field, $F = BIL$.
        \end{itemize}
  \item Let us now list the known quantities
        \begin{itemize}
          \item $w = mg = 4 \times 10^{-3} \times 9.8$
          \item $BL = 0.8 \times 0.15$
        \end{itemize}
  \item Now we must bring the velocity into this equation somehow, which hints at the use of the formula $\varepsilon = Blv$. However, the $emf$ can be expressed as $\varepsilon = IR$, and thus $Blv = IR$.
  \item In the equation $mg = BIL$, $I$ is the unknown quantity that we do not desire to find, thus we proceed by making the substitution $I = \dfrac{Blv}{R}$.
  \item Combining everything together, we have
        \begin{align*}
          mg & = BL(\frac{Blv}{R})  \\
          v  & = \frac{mgR}{B^2L^2}
        \end{align*}
  \item This will allow us to compute the desired value of $v$.
\end{enumerate}

\pagebreak

\subsection{Induction and Current}

Two coils of wire are wound around an iron cylinder. One coil is connected in a circuit with a cell and a switch that is initially closed. The other coil is connected to an ammeter. The switch is opened at time $t_0$.
\img{ex/gencur.png}{0.45}{Diagram}{gencur}

What is the ammeter reading before $t_0$, and what is the ammeter reading after $t_0$?
\begin{enumerate}
  \item First we must recall that current arises from induced emf, which in turn results from a change in magnetic flux linkage.
  \item Before the switch was opened, the magnetic field was constant and thus there was no change in flux linkage. Thus, the induced emf and consequently the ammeter reading was 0.
  \item At the moment the switch is opened, the current drops to 0, and there is now a change in flux linkage. This change in flux linkage induces an emf, and thus a current in the second coil. This then falls back to 0 as the flux linkage eventually stabilizes at 0.
\end{enumerate}

\pagebreak

\subsection{Mutual Induction}

Two conducting rings, A and B, have their centres on the same line. The planes of A and B
are parallel. There is a constant clockwise current in A. Ring A is stationary and ring B moves
towards ring A at a constant speed.

\img{ex/mutual.png}{0.45}{Diagram}{mutual}

(a) Outline why the magnetic flux in ring B increases.
\begin{itemize}
  \item Magnetic flux is given by $\Phi = BA$.
  \item In this case, the quantity of the field strength $B$ is increasing, because the rings are \hl{cutting more and more field lines}.
\end{itemize}
(b) State the direction of the induced current in ring B.
\begin{itemize}
  \item By Lenz's law, the induced current will be in the direction that opposes the change in the magnetic field.
  \item Since the field is increasing, the induced current will be in the opposite direction to the current in ring A.
  \item Thus, the induced current will be counterclockwise.
\end{itemize}
\pagebreak
(c) The graph shows how the magnetic flux in ring B varies with time.
\img{ex/mutual1.png}{0.45}{Graph}{mutual1}
Discuss the variation with time of the induced current in ring B.
\begin{itemize}
  \item First things first, when the question talks about the induced emf, stick Faraday's Law in your answer before anything.
        \begin{enumerate}
          \item The \textbf{rate of change} of magnetic flux in B increases (any graph that is not a straight line has a changing gradient)
          \item so, by Faraday's Law (stated at the beginning), the induced current will increase (in the direction that opposes the motion creating it) because resistance of ring is constant
        \end{enumerate}
\end{itemize}
(d) Outline why work must be done on ring B as it moves towards ring A at a constant speed.
\begin{itemize}
  \item The current induced in B gives rise to a magnetic field opposing that of A (by Lenz's) and thus an opposing force repelling B.
  \item Work must be done to move B in the opposite direction to this force such that the net force becomes 0 to achieve a constant speed.
\end{itemize}

\pagebreak

\subsection{Misc \#1}

A geophone is an instrument designed to measure the movement of ground rocks. When the ground moves, the magnet-spring system oscillates relative to the coil.
An emf is generated in the coil. The magnitude of this emf is proportional to the speed of the magnet relative to the coil.

\img{ex/1.png}{0.45}{Diagram}{geophone}

\begin{enumerate}[label=(\alph*)]
  \item \begin{enumerate}[label=(\roman*)]
          \item State the movement direction for which the geophone has its greatest sensitivity.
                \begin{itemize}
                  \item Vertical direction / parallel to springs
                \end{itemize}
          \item Outline how an emf is generated in the coil.
                \begin{enumerate}
                  \item The field of the magnet \hl{moves relative} to the coil
                  \item As field lines \hl{cut} the coil, \hl{forces} act on the initially stationary \hl{electrons} in the wire, and these move to produce an emf.

                \end{enumerate}
          \item Explain why the magnitude of the emf is related to the amplitude of the ground movement.
                \begin{itemize}
                  \item The springs have a natural time period for the oscillation
                  \item A greater amplitude of movement leads to higher magnet speed (with constant time period)
                  \item So field lines cut coil more quickly leading to greater emf
                \end{itemize}
          \item In one particular event, a maximum emf of 65 mV is generated in the geophone. The geophone coil has 150 turns. Calculate the rate of flux change that leads to this emf.
                \begin{align*}
                  \varepsilon                  & = -\frac{N\Delta \Phi}{\Delta t}     \\
                  \frac{\Delta \Phi}{\Delta t} & = -\frac{\varepsilon}{N}             \\
                                               & = -\frac{65\times 10^{-3}}{150}      \\
                                               & = \SI{0.43}{\milli\weber\per\second}
                \end{align*}
          \item Suggest two changes to the system that will make the
                geophone more sensitive.
                \begin{itemize}
                  \item Increase number of turns in coil; because more flux cutting per cycle
                  \item Increase field strength of magnet; so that there are more field lines
                  \item Change mass-spring system so that time period decreases; so magnet will be moving faster for given amplitude of movement
                \end{itemize}
        \end{enumerate}
\end{enumerate}

\pagebreak

\subsection{Quick-Fire MCQ \#1}

A coil X is connected to a cell and a switch that is initially open. Coil Y has its plane parallel to X. X and Y have a common axis.

\img{ex/2.png}{0.45}{Diagram}{coilxy}

When the switch is closed a force $F$ acts on Y due to X.
What is the variation with time of $F$ and what is the direction of F?

\img{ex/3.png}{0.7}{Options}{Options}

\begin{itemize}
  \item The direction is quite straightforward; X will initially induce a current in Y and by Lenz's law, the induced current in Y will be in the opposite direction to the current in X. Loops with opposite current will repel each other, and thus the force will be away from X. This \textcolor{red}{eliminates options A and C}.
  \item As with the variation of $F$:
        \begin{enumerate}
          \item Initially, the current in X increases rapidly, causing a large rate of change of flux, which induces a strong current in Y.
          \item Since only a changing magnetic field induces an emf, the induced emf in Y decreases as the rate of change of flux approaches zero.
          \item Eventually, when the current in X becomes constant, the induced current in Y becomes zero, meaning no force remains.
        \end{enumerate}
  \item This \textcolor{ForestGreen}{gives us option D}.

\end{itemize}

\pagebreak

\subsection{In, Through, and Out}


A square loop of wire of width $w$ is pulled at a constant velocity $v$ through a magnetic field of width $2w$.

\img{ex/4.png}{0.6}{Diagram}{loop}

Which of the following shows the variation of current $I$ in the loop with time $t$?

\img{ex/5.png}{0.7}{Options}{Options}

Child's play.
\begin{enumerate}
  \item The current is analogous to the induced emf.
  \item When the coil moves through the field with no parts of it out of the field, there is no change in flux linkage, and thus no induced emf. This is because the cutting at the front is compensated and cancelled out by the cutting at the back (at which the current is in the opposite direction). Hence, the current is 0. Already, \textcolor{red}{eliminating options A and D}.
  \item Finally, the emf before entering and after exiting the field are in opposite directions, therefore giving us \textcolor{ForestGreen}{option C}.
\end{enumerate}

\pagebreak

\subsection{Falling and Lenz's Law}

A conducting ring is dropped from rest from above the ground. As it falls the ring passes through a region of uniform horizontal magnetic field. Air resistance is negligible.

\img{ex/6.png}{0.25}{Diagram}{ring}

Which is correct about the acceleration $a$ of the ring as it enters and as it leaves the region of magnetic field?

\img{ex/7.png}{0.5}{Options}{Options}

\begin{itemize}
  \item As the ring enters the field, the field is pushing the ring to the top to oppose its downward motion, hence the acceleration is less than $g$.
  \item As the ring exits the field, the field is pulling the ring upwards to oppose its upward motion below the field, hence the acceleration is still less than $g$.
  \item Hence, \textcolor{ForestGreen}{option D} is the correct answer.
\end{itemize}

\pagebreak

\subsection{emf and Velocity}

Wire XY moves perpendicular to and through a magnetic field. The graph shows the variation with time of the displacement of XY.

\img{ex/8.png}{0.25}{Graph}{displacement}

What is the graph of the electromotive force (emf) induced across XY?

\img{ex/9.png}{0.55}{Options}{Options}

\begin{itemize}
  \item Simply invoke $\varepsilon = Blv$.
  \item At no stage is the velocity 0 (i.e. horizontal line on the graph), and thus the induced emf is always positive. This \textcolor{red}{eliminates options A and B}.
  \item The curved bits have lower gradients, hence lower velocities, and thus smaller emfs. This gives \textcolor{ForestGreen}{option C} as the correct answer.
\end{itemize}

\pagebreak

\subsection{emf and Flux Linkage Graphically}

The graph shows the variation of magnetic flux $\Phi$ in a coil with time $t$.

\img{ex/10.png}{0.55}{Graph}{flux}

What represents the variation with time of the induced emf $\varepsilon$ across the coil?

\img{ex/11.png}{0.55}{Options}{emf}

\begin{enumerate}
  \item Recognise this is a cosine graph.
  \item We know that the emf is the negative derivative function of the flux linkage. This means $$\varepsilon = -\diff{\Phi}{t} = -\diff{\cos t}{t} = \sin t$$
  \item Hence, \textcolor{ForestGreen}{option A} is the correct answer.
\end{enumerate}

\pagebreak

\subsection{Flemming's LHR in Disguise}

The coil of a direct current electric motor is turning with a period $T$. At $t = 0$, the coil is in the position shown in the diagram. Assume the magnetic field is uniform across the coil.

\img{ex/12.png}{0.75}{}{motor}

\begin{itemize}
  \item At the position shown, using the left hand rule gives that the force is upwards. Then, doing the same when XY is near the north pole of the magnet, namely at a half turn, we find that the force is downwards (the current is now in the opposite direction as it was in the previous position, since it is d.c. and was flipped). This helps us to eliminate \textcolor{red}{options B and C}, where the force at $t=\frac{T}{2}$ is in the same direction as at $t = 0$.
  \item We know that $F = BIL$. Since we are told that the field is uniform, $B$ remains constant while XY passed through the range that it covers. None of $I$ or $L$ change, and thus the force is constant. This eliminates \textcolor{red}{option D}, where the force is curving.
  \item Ultimately, this leads us to the answer \textcolor{ForestGreen}{option A}.
\end{itemize}

\pagebreak

\subsection{Quick-Fire MCQ \#2}

\img{ex/13.png}{0.75}{}{quickfire}
\begin{itemize}
  \item We immediately \textcolor{red}{eliminate B} because when the magnet is far away from the coil, the flux linkage is 0, and thus the induced emf is 0.
  \item Long story short, at the two ends, the current and hence the emf must be in opposite directions, and thus the induced emf must be 0 at the two ends. This \textcolor{ForestGreen}{gives C}.
\end{itemize}

Here is a more detailed explanation:
\begin{enumerate}
  \item Initially, the flux in the loop is increasing because the magnetic field at the loop is getting larger as the magnet approaches.
  \item The induced current must then \hl{oppose the increase in the flux}.
  \item At the front of the magnet, because field lines go from N to S, the field is actually downwards.
  \item To establish an opposing field (e.g. upwards), we use the right-hand grip rule to find the direction of the induced current, which is anticlockwise seen from above.
  \item In the middle, the flux is constant, so the induced emf is 0.
  \item Moving below the loop, the flux starts to decrease as the magnet moves away.
  \item Our induced current will now want to maintain the flux / \hl{oppose the decrease} by reinforcing the field.
  \item At the tail S, the field lines are going into S and so the field is downwards. To reinforce this field, the current must be reversed, since previously it was creating an upward field.
\end{enumerate}

\pagebreak

\subsection{Misc \#2}

The diagram shows an alternating current generator with a rectangular coil rotating at a constant frequency in a uniform magnetic field.

\img{ex/14.png}{0.75}{Question}{acgen}

\begin{enumerate}[label=(\alph*)]
  \item Explain, by reference to Faraday's law of induction, how an electromotive force (emf) is induced in the coil. \hl{This is a standard framework}
        \begin{itemize}
          \item There is a magnetic flux linkage in the coil as the coil cuts magnetic field
          \item This flux linkage changes as the angle varies/coil rotates
          \item Faraday's law states that
                $$\varepsilon = -\frac{N\Delta \Phi}{\Delta t}$$
                and hence a change in flux linkage induces an emf.
        \end{itemize}
  \item \begin{enumerate}[label=(\roman*)]
          \item The average power output of the generator is $\SI{8.5e5}{\W}$. Calculate the root mean square (rms) value of the generator output current.
                \begin{align*}
                  V_\text{rms} & = \frac{V_\text{peak}}{\sqrt{2}} = \frac{25\times 10^3}{\sqrt{2}} = \SI{1.77e4}{\V} \\
                  I_\text{rms} & = \frac{P}{V_\text{rms}} = \frac{8.5\times 10^5}{1.77\times 10^4} = \SI{48}{\A}
                \end{align*}
        \end{enumerate}
  \item

\end{enumerate}


\end{document}