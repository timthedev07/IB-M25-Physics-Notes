\documentclass[a4paper,12pt]{article}
\usepackage{setspace}
\usepackage{sectsty}
\usepackage{siunitx}
\usepackage{graphicx}
\usepackage[a4paper, total={3in, 9in}, textwidth=16cm,bottom=1in,top=1.4in]{geometry}
\usepackage[dvipsnames]{xcolor}
\usepackage{amsmath}
\usepackage{esvect}
\usepackage{soul}
\usepackage{amsthm}
\usepackage{hyperref}
\usepackage{longtable}
\usepackage{draftwatermark}
\usepackage{float}
\usepackage{amssymb}
\usepackage{outlines}
\usepackage{caption}
\usepackage{fancyvrb}
\usepackage{subcaption}
\usepackage{esdiff}
\usepackage{dirtytalk}
\usepackage{colortbl}
\usepackage{booktabs}
\usepackage{setspace}
\usepackage{mathtools}
\usepackage{tikz,pgfplots}
\usepackage[most]{tcolorbox}
\SetWatermarkText{timthedev07}
\SetWatermarkScale{4}
\SetWatermarkColor[gray]{0.97}
\usetikzlibrary{positioning,decorations.markings,arrows.meta,angles,quotes}
\DeclarePairedDelimiter{\ceil}{\lceil}{\rceil}
\newtheorem{lemma}{Lemma}
\newtheorem{proposition}{Proposition}
\newtheorem{remark}{Remark}
\newtheorem{observation}{Observation}
\doublespacing
\let\oldsection\section
\renewcommand\section{\clearpage\oldsection}
\newcommand{\RNum}[1]{\uppercase\expandafter{\romannumeral #1\relax}}
\let\oldsi\si
\renewcommand{\si}[1]{\oldsi[per-mode=reciprocal-positive-first]{#1}}
\usepackage{enumitem}
\newcommand{\subtitle}[1]{%
  \posttitle{%
    \par\end{center}
    \begin{center}\large#1\end{center}
    \vskip0.5em}%
}
\newcommand{\degsym}{^{\circ}}
\newcommand{\Mod}[1]{\ (\mathrm{mod}\ #1)}
\usepackage{hyperref}
\hypersetup{
  colorlinks=true,
  linkcolor = blue
}
\newcommand{\lb}{\\[8pt]}
\newenvironment*{cell}[1][]{\begin{tabular}[c]{@{}c@{}}}{\end{tabular}}
\newcommand{\img}[4]{\begin{center}
  \begin{figure}[H]
    \centering
    \includegraphics[width=#2\textwidth]{#1}
    \caption{#3}
    \label{fig:#4}
  \end{figure}
\end{center}}
\parindent=0pt
\usepackage{fancyhdr}
\fancyfoot{}
\fancypagestyle{fancy}{\fancyfoot[R]{\vspace*{1.5\baselineskip}\thepage}}
\renewcommand{\contentsname}{Table of Contents}
\newcommand{\angled}[1]{\langle{#1}\rangle}
\newcommand{\paren}[1]{\left(#1\right)}
\newcommand{\sqb}[1]{\left[#1\right]}
\newcommand{\coord}[3]{\angled{#1,\, #2,\, #3}}
\newcommand{\pair}[2]{\paren{#1,\, #2}}
\newcommand{\atom}[3]{{}^{#1}_{#2}\text{#3}}
\usepackage[
  noabbrev,
  capitalise,
  nameinlink,
]{cleveref}

\crefname{lemma}{Lemma}{Lemmas}
\crefname{proposition}{Proposition}{Propositions}
\crefname{remark}{Remark}{Remarks}
\crefname{observation}{Observation}{Observations}

\newtcolorbox[auto counter]{prob}[2][]{fonttitle=\bfseries, title=\strut Problem~\thetcbcounter: #2,#1,colback=Orchid!5!white,colframe=Orchid!75!black,top=5mm,bottom=5mm}

\newtcolorbox[auto counter]{rem}[1][]{fonttitle=\bfseries, title=\strut Remark.~\thetcbcounter,colback=purple!5!white,colframe=purple!65!gray,top=5mm,bottom=5mm}

\newtcolorbox[auto counter]{defin}[1][]{fonttitle=\bfseries, title=\strut Definition.~\thetcbcounter,colback=black!5!white,colframe=black!65!gray,top=5mm,bottom=5mm}

\newtcolorbox[auto counter]{obs}[1][]{fonttitle=\bfseries, title=\strut Observation.~\thetcbcounter,colback=RedViolet!5!white,colframe=RedViolet!65!gray,top=5mm,bottom=5mm}

\newtcolorbox[auto counter]{lem}[1][]{fonttitle=\bfseries, title=\strut Lemma.~\thetcbcounter,colback=Maroon!5!white,colframe=Maroon!65!gray,top=5mm,bottom=5mm}

\newtcolorbox[auto counter]{prop}[1][]{fonttitle=\bfseries, title=\strut Proposition.~\thetcbcounter,colback=RedOrange!5!white,colframe=RedOrange!65!gray,top=5mm,bottom=5mm}

\newtcolorbox[auto counter]{hint}[1][]{fonttitle=\bfseries, title=\strut Hint.~\thetcbcounter,colback=OliveGreen!5!white,colframe=OliveGreen!75!gray,top=5mm,bottom=5mm}

\setlength{\belowcaptionskip}{-20pt}
\begin{document}


\pagenumbering{arabic}
\pagestyle{fancy}


\begin{titlepage}
  \begin{center}

    \vspace*{8cm}
    \textbf{\Large {IB Physics Topic B2 The Greenhouse Effect; SL \& HL}} \\
    \vspace*{1cm}
    \large{By timthedev07, M25 Cohort}

  \end{center}
\end{titlepage}

\pagebreak
\tableofcontents
\pagebreak

\clearpage
\setcounter{page}{1}
\addtocontents{toc}{\protect\thispagestyle{empty}}

\section{Gray Bodies and Emissivity}

A gray body is one that emits less energy than a perfect black body, this is the case for most real-world objects. To encapsulate this reduction, the previously-mentioned Stefan-Boltzmann law is modified to include a factor $\varepsilon$ which is the emissivity of the body.
\begin{itemize}
  \item The emissivity of a body is a measure of how well it emits radiation compared to a black body.
  \item Formally, the emissivity is defined as the ratio of the energy radiated by a body to the energy radiated by a black body at the same temperature.
        $$\varepsilon = \frac{\text{power emitted by the object}}{\text{power emitted by a black body at the same temperature}}$$
        \begin{enumerate}
          \item for an ideal black body, $\varepsilon = 1$
          \item for a total reflector, $\varepsilon = 0$
        \end{enumerate}
\end{itemize}

The Stefan-Boltzmann law for gray bodies is given by
\begin{equation}\label{eq:stefan-boltzmann-law-gray}
  P = \varepsilon \sigma A T^4
\end{equation}
equivalently
$$I = \varepsilon \sigma T^4$$
where $I$ is the intensity of radiation emitted by the body.\lb
Note that, if one were to calculate the \textbf{net power} radiated by a body in a surrounding of temperature $T_0$, one would have to discount the power absorbed by the body from the surroundings, because while the body emits radiation, it also absorbs radiation from the surroundings. This may also be referred to as the net power exchanged between the body and its surroundings.
$$\Delta P = \varepsilon \sigma A \paren{T^4 - T_0^4}$$

\section{The Solar Constant}

The solar constant is defined as

\begin{center}
  the total intensity of solar radiation, across all wavelengths, that reaches a surface perpendicular to the line connecting the centers of the Earth and Sun at Earth's average distance from the Sun, taken at the top of the atmosphere
\end{center}

\img{solar-constant.jpg}{1}{The Solar Constant}{solar-constant}

This constant has a value of about $\SI{1360}{\watt\per\meter\squared}$. However, this value varies periodically because
\begin{itemize}
  \item The output of the Sun varies by about $\SI{0.1}{\percent}$ over an 11-year cycle
  \item The Earth's orbit is slightly elliptical
  \item There are other longer periodic variations
\end{itemize}

S is calculated as
$$S = \frac{L}{4\pi d^2} = \varepsilon \sigma T^4$$
where $L$ is the luminosity of the Sun, $d$ is the distance from the Sun to the Earth. In other words, the solar constant for a particular planet $d'$ away from the Sun, in terms of $S$, is given by
\begin{equation}\label{eq:solar-constant}
  S' = S \paren{\frac{d}{d'}}^2
\end{equation}

\section{The Atmosphere}

\img{atm_rad.png}{0.5}{The Atmosphere}{atm-rad}

The radiation from the Sun that actually falls onto the Earth has an area of $\pi R^2$, where $R$ is the radius of the Earth. The atmosphere then distributes this energy over the entire surface of the Earth, which has an area of $4\pi R^2$. Thus, \textbf{the average intensity} of solar radiation \textbf{on the entire surface} is $\dfrac{1}{4}S$.\lb
In reality, this theoretical prediction is higher than the practical value of the intensity on the surface --- the atmosphere also partially absorbs as well as reflects the radiation.\lb
However, since the ground is not an idea black body, it will reflect off some of the incident radiation. The fraction of the total radiation reflected is called the \textbf{albedo}.

\pagebreak

\subsection{Albedo}

The \textit{albedo} of a surface is the \textbf{fraction of the total radiation incident on the surface that is reflected}. It is a ratio and hence a dimensionless quantity between 0 and 1. The albedo of the Earth is about 0.3, which means that about 30\% of the radiation incident on the Earth is reflected back into space. Again, this value varies depending on the latitude, the surface, and the time of year.
It is formally defined as
$$\alpha = \frac{\text{total reflected power}}{\text{total incident power}}$$
The \say{coefficient of absorption}, on the other hand, is $1 - \alpha$; it shows the portion of the radiation that is absorbed by the surface.
\hl{It is important to note that the intensity emitted = intensity absorbed.}


\section{The Greenhouse Effect}

The Earth and the Moon have roughly the same distance from the Sun, but the Earth is warmer than the Moon. This is because the Earth has an atmosphere.
\begin{enumerate}
  \item There are certain types of gases, e.g. CO$_2$, H$_2$O (water vapor), CH$_4$ (methane), and N$_2$O (nitrous oxide), that naturally occur in the atmosphere.
  \item Others, such as Ozone, are partially natural and partially artificial; they contribute to the greenhouse effect too.
  \item Collectively, they are the \textbf{greenhouse gases} --- they are inert to photons at the frequency of visible light, but they interact with and absorb infrared radiation.
\end{enumerate}
The origins of the greenhouse effect have two categories --- artificial and natural. The natural greenhouse effect is due to the naturally occurring levels of certain gases; the artificial greenhouse effect is due to human activities that increase the levels of these gases --- this is referred to as the \textbf{enhanced greenhouse effect}.

\subsection{The Temperature Equilibrium}

\begin{enumerate}
  \item The atmosphere absorbs most of the infrared and ultraviolet radiation from the Sun, but allows most of the visible light to pass through. The surface then absorbs this visible light, increasing in temperature.
  \item Since the Earth is above 0 K, it \textbf{re-radiates} back at the atmosphere, at a much lower frequency (longer wavelength) than the radiation it absorbed from the Sun.
  \item Just as when the radiation came from the Sun, the atmosphere absorbs parts of the radiation from Earth in the infrared part of the spectrum, and re-radiates it arbitrarily in all directions.
  \item Some are returned to the surface of the Earth, which increases the temperature of the Earth. This means that a portion of the energy is trapped.
\end{enumerate}

The whole system is in an \textbf{equilibrium}, where
\begin{equation}\label{eq:energy-balance}
  \text{incident energy from the Sun} = \text{energy radiated back into the space}
\end{equation}
The \say{Earth} refers to the entirety of the Earth-atmosphere system.
The \textit{enhanced greenhouse effect} causes an increase in the level of energy retained in the system, which then requires an increase in the energy emission of the Earth to \textbf{maintain the equilibrium}. Per the \textit{Stefan-Boltzmann law}, this means that the Earth must \textbf{increase in temperature} until reaching a new and higher equilibrium point.

\subsection{Energy Absorption}

This subsection explores why greenhouse gases absorb energy. Both ultraviolet and infrared can be absorbed, but in different ways:
\begin{enumerate}
  \item Ultraviolet radiation has \textbf{high energy} and can \textbf{break molecular bonds} in gases.
        \begin{enumerate}
          \item For example, when oxygen molecules absorb UV light, they break into individual oxygen atoms, leading to the \textbf{formation of ozone} (O$_3$).
          \item This process is a reaction in the upper atmosphere that protects us from harmful UV radiation by absorbing it.
        \end{enumerate}
  \item Infrared radiation has lower energy compared to UV, so it can't break chemical bonds. Instead, it interacts with the \textbf{vibrational modes of greenhouse gas} molecules like carbon dioxide.
        \begin{enumerate}
          \item Each gas molecule has specific vibrational modes that are like the natural frequency of the molecule.
          \item \textbf{Resonance} occurs when the frequency of incoming IR radiation matches the natural frequency of the vibrational modes in the molecule. This causes the molecule to absorb energy and vibrate more intensely.
        \end{enumerate}
\end{enumerate}

\subsection{Modeling Climate Balance}

\subsubsection{Naive Model}

\begin{minipage}{0.35\textwidth}
  \img{naive.png}{1}{Naive Model}{naive}
\end{minipage}%
\hspace{0.05\textwidth}
\begin{minipage}{0.6\textwidth}
  The initial naive model uses the assumption that the atmosphere does not absorb any radiation emitted by the Earth.
  \begin{enumerate}
    \item We start by assuming that the Earth behaves like a black body and that it emits radiation with an intensity of $\SI{238}{\watt\per\meter\squared}$.
    \item The Earth emits radiation at a temperature of 255 K in this case.
  \end{enumerate}
\end{minipage}
\begin{enumerate}
  \setcounter{enumi}{2}
  \item The transmittance (the extent of transparency) is $\SI{100}{\percent}$ for all wavelengths, so the Earth's radiation completely passes through the atmosphere.
\end{enumerate}

\pagebreak

\subsubsection{Revised Model}

\begin{minipage}{0.6\textwidth}
  In reality, the atmosphere absorb the infrared and ultraviolet parts of the spectrum.
  \begin{enumerate}
    \item The absorbed area is shown by the regions where the transmittance is $\SI{0}{\percent}$.
    \item Ultraviolet is \textbf{below $\SI{300}{\nano\meter}$}, and infrared is \textbf{above $\SI{700}{\nano\meter}$}. The left-out region in the middle of the two is the visible light spectrum.
    \item The atmosphere absorbs but then re-radiates it back to the Earth, \textbf{decreasing the level of energy going back into the space}.
    \item This \textbf{breaks the equilibrium}; as the incoming energy is now greater, the Earth must increase in temperature to maintain the equilibrium.
  \end{enumerate}
\end{minipage}%
\hspace{0.05\textwidth}
\begin{minipage}{0.35\textwidth}
  \img{notnaive.png}{1}{Corrected Model}{corrected}
\end{minipage}
\lb
It must be noted that, for each of UV and IR, there also exist specific wavelengths within themselves that are absorbed by different gases. For example, for infrared, the transmittance graph is as follows
\img{irtransmittance} {0.8} {Infrared Transmittance} {irtransmittance}


\section{Earth Energy Balance}

\img{systemsummary.png}{1}{Earth Energy Balance}{systemsummary}

The above graph provides a simplified visualization of the energy balance system. The numbers do not matter.

\pagebreak

\subsection{Global Warming}

For the exam, it is important to remember the following greenhouse gases; their respective symbols are CO$_2$, CH$_4$, and N$_2$O. \hl{Their main role in the greenhouse effect is to absorb outgoing radiation emitted by the Earth.}

\img{env.png}{0.8}{Global Warming}{env}

Analysis of Antarctic ice cores has shown that the concentration of CO$_2$ has never been this high in the last 800,000 years.

\img{icecore.png}{0.7}{Ice Core Analysis}{icecore}

The increasing concentration of greenhouse gases in the atmosphere causes more radiation from the Earth's surface to be absorbed by the atmosphere, which in term leads to an increased intensity of the radiation emitted back to the Earth.

\pagebreak

\subsubsection{Consequences that Further The Warming}

Global warming leads to a series of consequences that themselves will further the warming:
\begin{itemize}
  \item The melting of ice caps and glaciers will \textbf{decrease the Earth's albedo}, which will increase the amount of radiation absorbed by the Earth.
  \item Higher seawater temperatures will reduce the solubility of CO$_2$ in the water, which will increase the concentration of CO$_2$ in the atmosphere.
\end{itemize}

\section{Exam Questions}

\subsection{Re-radiating Infrared}

The average temperature of ocean surface water is 289 K. Oceans behave as black bodies.
\begin{enumerate}[label=(\alph*)]
  \item Show that the intensity radiated by the oceans is about 400 $\si{\W\per\m\squared}$

        \begin{align*}
          \varepsilon & = 1                                                                                       \\
          I           & = \varepsilon \sigma T^4 = 1 \times \sigma \times 289^4 \approx 400 \si{\W\per\m\squared}
        \end{align*}
  \item Explain why some of this radiation is returned to the oceans from the atmosphere. [3]
        \begin{itemize}
          \item Most of the emitted radiation is in the infrared part of the spectrum, this is because $$\lambda = \frac{b}{T} = \frac{2.9 \times 10^{-3}}{289} \approx 10 \times 10^{-6} \si{\m} = 10 \si{\micro\m}$$
          \item This radiation is absorbed by greenhouse gases in the atmosphere such as CO2 or methane.
          \item The gas then re-emits/re-radiates this radiation ...
          \item ... partly back towards oceans and partly into other arbitrary directions.
        \end{itemize}
\end{enumerate}

\pagebreak

\subsection{M19 SL Paper 2 TZ1 Q6}

The Moon has no atmosphere and orbits the Earth. The diagram shows the Moon with rays
of light from the Sun that are incident at 90° to the axis of rotation of the Moon.

\img{ex/1.png}{0.7}{Moon}{moon}

\begin{enumerate}[label=(\alph*)]
  \item \begin{enumerate}[label=(\roman*)]
          \item A black body is on the Moon’s surface at point A. Show that the maximum
                temperature that this body can reach is 400 K. Assume that the Earth and the
                Moon are the same distance from the Sun. [2]
                \begin{align*}
                  T^4 & = \frac{S}{\varepsilon \sigma} = \frac{1360}{1 \times 5.67 \times 10^{-8}} \\
                  T   & \approx 400 \si{\K}
                \end{align*}
          \item Another black body is on the Moon's surface at point B. Outline, without calculation, why the maximum temperature of the black body at point B is less than at point A.

                \textcolor{ForestGreen}{Answer:} \begin{enumerate}
                  \item Energy/Power/Intensity lower at B, since it is further away from the Sun.

                  \item Since $T^4 \propto I$, the temperature at B will be lower.
                \end{enumerate}
        \end{enumerate}
  \item The albedo of the Earth's atmosphere is 0.28. Outline why the maximum temperature of a black body on the Earth when the Sun is overhead is less than that at point A on the Moon.
        \begin{itemize}
          \item Because a portion of the radiation from the Sun is reflected back into space by the atmosphere; this portion is about 28\%.
        \end{itemize}
  \item The Moon orbits the Earth in a circular path.
        Outline why
        \begin{enumerate}[label=(\roman*)]
          \item a force acts on the Moon.
                \begin{itemize}
                  \item There is an attractive gravitational force between the planet and the Moon.
                \end{itemize}
          \item this force does no work on the Moon.
                \begin{itemize}
                  \item The force and the velocity are at 90$\degsym$ to each other and there is no change in GPE of the moon.
                \end{itemize}
        \end{enumerate}
\end{enumerate}

\pagebreak

\subsection{Radiation Processes in the Atmosphere}

What is the average intensity radiated by the atmosphere towards the surface?
\img{ex/2.png}{0.7}{Radiation Processes in the Atmosphere}{rad}
\begin{itemize}
  \item We must first identify the quantity we are interested in determining --- this is the right-most unlabeled arrow.
  \item It is important to realize that the \hl{intensity emitted by the surface must equal the intensity absorbed by the surface}. This is the most crucial step in any question of this kind.
  \item The total emitted intensity is 390 $\si{\W\per\m\squared}$, and the total absorbed intensity is 240 + our desired quantity.
  \item Baby maths -- the answer is 150.
\end{itemize}

\pagebreak

\subsection{Titan and the Sun}

Titan is a moon of Saturn. The Titan-Sun distance is 9.3 times greater than the Earth-Sun distance.
\begin{enumerate}[label=(\alph*)]
  \item Show that the intensity of the solar radiation at the location of Titan is 16 $\si{\W\per\m\squared}$.
        \begin{itemize}
          \item We invoke the equation \cref{eq:solar-constant} with $\dfrac{d}{d'} = \dfrac{1}{9.3}$ to obtain
                $$S' = 1360 \times \paren{\frac{1}{9.3}}^2 = 15.72... \approx 16 \si{\W\per\m\squared}$$
        \end{itemize}
  \item Titan has an atmosphere of nitrogen. The albedo of the atmosphere is 0.22. The surface of Titan may be assumed to be a black body. Explain why the \textbf{average} intensity of solar radiation \textbf{absorbed} by the whole surface of Titan is 3.1$\si{\W\per\m\squared}$.
        \begin{itemize}
          \item An albedo of 0.22 means that 22\% of the radiation is reflected back into space and only 0.78 enter the atmosphere. Also, because the area of contact with radiation is $\pi R^2$ and the total surface area is $4\pi R^2$, the average intensity across the whole Moon is $\dfrac{1}{4}$ of the previously calculated value.
                \begin{align*}
                  0.78 \times 16 \times \frac{1}{4} = 3.12 \approx 3.1 \si{\W\per\m\squared}
                \end{align*}
        \end{itemize}
  \item Show that the equilibrium surface temperature of Titan is about 90K.
        \begin{align*}
          T & = \sqrt[4]{\frac{S}{\varepsilon \sigma}} = \sqrt[4]{\frac{3.1}{1 \times 5.67 \times 10^{-8}}} = 85.98... \approx 90 \si{\K} \\
        \end{align*}
\end{enumerate}

\pagebreak

\subsection{Emissivity Ratio Question}

Two surfaces X and Y emit radiation of the same surface intensity. X emits a radiation of peak wavelength twice that of Y.

What is $\dfrac{\text{emissivity of X}}{\text{emissivity of Y}}$?

\begin{itemize}
  \item We use Wien's Law $$\lambda \propto \frac{1}{T}$$
        to deduce that the temperature of X is half that of Y.
  \item We then use the Stefan-Boltzmann law for intensity $$I = \varepsilon \sigma T^4 \implies \varepsilon \propto \frac{1}{T^4}$$
        joined with the previous proportionality to obtain
        $$\varepsilon \propto \frac{1}{T^4} \propto \lambda^4$$
  \item Hence, a factor multiplied to the wavelength is raised to the power of 4 when applied to the emissivity. Hence the answer is 16.
\end{itemize}

\pagebreak

\vspace*{10cm}

\begin{center}
  \textbf{\Huge{Somebody stop global warming...}}
\end{center}

\end{document}