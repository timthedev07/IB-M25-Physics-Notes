\documentclass[a4paper,12pt]{article}
\usepackage{setspace}
\usepackage{sectsty}
\usepackage{siunitx}
\usepackage{graphicx}
\usepackage[a4paper, total={3in, 9in}, textwidth=16cm,bottom=1in,top=1.4in]{geometry}
\usepackage[dvipsnames,table]{xcolor}
\usepackage{amsmath}
\usepackage{esvect}
\usepackage{soul}
\usepackage{amsthm}
\usepackage{hyperref}
\usepackage{float}
\usepackage{amssymb}
\usepackage{outlines}
\usepackage{caption}
\usepackage{makecell}
\usepackage{fancyvrb}
\usepackage{subcaption}
\usepackage{esdiff}
\usepackage{setspace}
\usepackage{dirtytalk}
\usepackage{mathtools}
\usepackage{tikz,pgfplots}
\usepackage{dirtytalk}
\usepackage{draftwatermark}
\usepackage[most]{tcolorbox}
\SetWatermarkText{timthedev07}
\SetWatermarkScale{4}
\SetWatermarkColor[gray]{0.97}
\usetikzlibrary{positioning,decorations.markings,calc}
\DeclarePairedDelimiter{\ceil}{\lceil}{\rceil}
\newtheorem{lemma}{Lemma}
\newtheorem{proposition}{Proposition}
\newtheorem{remark}{Remark}
\newtheorem{observation}{Observation}
\doublespacing
\let\oldsection\section
\renewcommand\section{\clearpage\oldsection}
\newcommand{\RNum}[1]{\uppercase\expandafter{\romannumeral #1\relax}}
\let\oldsi\si
\renewcommand{\si}[1]{\oldsi[per-mode=reciprocal-positive-first]{#1}}
\usepackage{enumitem}
\newcommand{\subtitle}[1]{%
  \posttitle{%
    \par\end{center}
    \begin{center}\large#1\end{center}
    \vskip0.5em}%
}
\newcommand{\degsym}{^{\circ}}
\newcommand{\Mod}[1]{\ (\mathrm{mod}\ #1)}
\usepackage{hyperref}
\hypersetup{
  colorlinks=true,
  linkcolor = blue
}
\newcommand{\lb}{\\[8pt]}
\newenvironment*{cell}[1][]{\begin{tabular}[c]{@{}c@{}}}{\end{tabular}}
\newcommand{\img}[4]{\begin{center}
  \begin{figure}[H]
    \centering
    \includegraphics[width=#2\textwidth]{#1}
    \caption{#3}
    \label{fig:#4}
  \end{figure}
\end{center}}
\parindent=0pt
\usepackage{fancyhdr}
\fancyfoot{}
\newcommand{\vect}[3]{\begin{bmatrix}
  #1 \\
  #2 \\
  #3
\end{bmatrix}}
\fancypagestyle{fancy}{\fancyfoot[R]{\vspace*{1.5\baselineskip}\thepage}}
\renewcommand{\contentsname}{Table of Contents}
\newcommand{\angled}[1]{\langle{#1}\rangle}
\newcommand{\paren}[1]{\left(#1\right)}
\newcommand{\sqb}[1]{\left[#1\right]}
\newcommand{\coord}[3]{\angled{#1,\, #2,\, #3}}
\newcommand{\pair}[2]{\paren{#1,\, #2}}
\usepackage[
  noabbrev,
  capitalise,
  nameinlink,
]{cleveref}
\crefname{lemma}{Lemma}{Lemmas}
\crefname{proposition}{Proposition}{Propositions}
\crefname{remark}{Remark}{Remarks}
\crefname{observation}{Observation}{Observations}

\newtcolorbox[auto counter]{defin}[1][]{fonttitle=\bfseries, title=\strut Definition.~\thetcbcounter,colback=black!5!white,colframe=black!65!gray,top=5mm,bottom=5mm}

\newtcolorbox[auto counter]{obs}[1][]{fonttitle=\bfseries, title=\strut Observation.~\thetcbcounter,colback=RedViolet!5!white,colframe=RedViolet!65!gray,top=5mm,bottom=5mm}

\setlength{\belowcaptionskip}{-20pt}

\begin{document}


\pagenumbering{arabic}
\pagestyle{fancy}


\begin{titlepage}
  \begin{center}

    \vspace*{8cm}
    \textbf{\Large {IB Physics Topic C1 S.H.M; SL \& HL}} \\
    \vspace*{1cm}
    \large{By timthedev07, M25 Cohort}


  \end{center}
\end{titlepage}

\pagebreak
\tableofcontents
\pagebreak

\clearpage
\setcounter{page}{1}
\addtocontents{toc}{\protect\thispagestyle{empty}}

\section{Isochronous Oscillations}

This refers to periodic oscillations that maintain a constant frequency regardless of changes in amplitude. In reality, the amplitude of the motion will gradually drop because of energy losses; but isochronous oscillations will maintain a constant frequency.

\section{Defining Periodic Motion}
No matter which system we are looking at, be it the spring-mass or pendulum, the displacement, velocity, and acceleration of the system will be sin/cos functions of time. The following displacement-time diagram shows important properties of an s.h.m. system.
\img{graphlabeledintro.png}{0.8}{Labeled s.h.m. system}{graphlabeledintro}
\begin{itemize}
  \item The displacement is simply the distance from the equilibrium position, and the amplitude is the maximum displacement from the equilibrium position.
  \item The equilibrium position is the position where the system is naturally at rest (this is also where the acceleration is 0 and velocity is maximum when the system is oscillating).
  \item The period is the time taken for the system to complete one full cycle/oscillation.
  \item The frequency --- number of cycles per second, is given by
        $$f = \frac{1}{T}$$
        where $T$ is the period.
\end{itemize}


\section{S.H.M. Basic Equations}

S.h.m is a type of isochronous oscillatory motion where \hl{the force acting on the oscillator is directly proportional to its displacement} from a central equilibrium position and \hl{is directed toward that position}. The constant of proportionality is the square of the angular frequency, namely $\omega^2$. The equation of motion for s.h.m. is given by
\begin{equation}
  a = -\omega^2 x
\end{equation}
This means that the acceleration/displacement graph will be a negative-slope straight line through the origin.
\img{shmrequirement.png}{0.4}{Force-displacement graph for s.h.m.}{shmrequirement}
The angular frequency $\omega$ is given by
$$\omega = \frac{2\pi}{T} = 2\pi f$$
where $f$ is the frequency of the oscillator. Although angular frequency can take on the unit $\si{\radian\per\second}$, the radians are often omitted since it is a unitless ratio --- it is simply $2\pi$ times the frequency, which has unit $\si{\hertz}\equiv \si{\per\s}$.

There are two systems we study at the IB level: the spring-mass system and the pendulum system. We will examine each one separately.

\pagebreak

\subsection{Spring Mass}

Consider a mass $m$ attached to a spring with spring constant $k$, moving on a frictionless surface along the horizontal axis. Initially, it is at rest on the surface and the spring is relaxed. We pull the mass $x_0$ away from the initial position --- this will be the amplitude of the subsequent s.h.m.

\hspace*{0.1\textwidth}
\begin{minipage}{0.3\textwidth}
  \img{springmass1.png}{1}{Spring-mass system at equilibrium}{springmass1}
\end{minipage}\hspace*{0.1\textwidth}%
\begin{minipage}{0.4\textwidth}
  \img{springmass2.png}{1}{Spring-mass system displaced}{springmass2}
\end{minipage}\hspace*{0.1\textwidth}\lb

Using \hl{Hooke's Law} on the oscillator, we obtain that the restoring force is
$$F = ma = -kx$$
where $k$ is the spring constant. Comparing coefficients with the s.h.m. equation, we get that the angular frequency is given by
$$\omega = \sqrt{\frac{k}{m}}$$
Similarly, the period and frequency are given by
$$T = 2\pi\sqrt{\frac{m}{k}}\quad\text{ and }\quad f = \frac{1}{2\pi}\sqrt{\frac{k}{m}}$$
My advice is forget about these two, as they look similar to each other and to $\omega$, and so it's easy to confuse them.

\pagebreak

\subsection{Pendulum}

Consider a pendulum of length $l$ and mass $m$ swinging in a gravitational field of strength $g$. In this case, the displacement will be the arc length that the mass makes with the origin at any point during oscillation.
\img{pendulumlabeled.png}{0.6}{Labeled pendulum system}{pendulumlabeled}
It must be noted that the pendulum obeys the s.h.m only for small angles of $\theta$ (typically less than 10 degrees).\lb
In this case, the restoring acceleration is $mg\sin\theta$; using small angle approximation and the fact that $l$ is the radius of the arc $x$ subtended by the angle $\theta$, we get that the angular frequency is given by
\begin{equation}
  \omega = \sqrt{\frac{g}{l}}
\end{equation}

\section{S.H.M. Equations --- Circular Motion Form}
As mentioned earlier, the displacement, velocity, and acceleration of the system will be sin/cos functions of time. Let us now transform this oscillatory system into a circular motion system to obtain information about these quantities.\lb
Firstly, we must know that the trajectory of a circle with radius $r$ centered at the origin on the Cartesian plane with axes y/x can be expressed in a parametric form as follows
$$x = r\cos\theta\quad\text{ and }\quad y = r\sin\theta$$
where $r$ is the radius of the circle and $\theta = \omega t$ is the so called \say{phase angle}. We can use this to transform the s.h.m. system into a circular motion system. The following diagram shows the transformation.
\img{circ.png}{0.6}{Transformation of s.h.m. to circular motion}{circ}
From \cref{fig:circ}, notice that the peak height of the trig. function, which is the amplitude, maps to the radius of the circular motion. This means that $r = x_0$. The $r\cos \theta$ is for s.h.m. beginning at an extreme position, and inversely, $r\sin \theta$ is for s.h.m. beginning at the equilibrium position. In summary, the displacement is either $x_0\cos(\omega t)$ or $x_0\sin(\omega t)$, depending on the initial position of the oscillator. The respective expressions for velocity and acceleration are of a similar form and can be trivially found by differentiation. The table on the next page puts everything together.\pagebreak
\begin{table}[H]
  \centering
  \def\arraystretch{1.2}
  \begin{tabular}{|c | c |}
    \hline \rowcolor{Blue!25} Starts at extreme & Starts at equilibrium             \\
    \hline
    $x = x_0\cos(\omega t)$                     & $x = x_0\sin(\omega t)$           \\ \hline
    $v = -\omega x_0\sin(\omega t)$             & $v = \omega x_0\cos(\omega t)$    \\ \hline
    $a = -\omega^2 x_0\cos(\omega t)$           & $a = -\omega^2 x_0\sin(\omega t)$ \\\hline
  \end{tabular}
\end{table}
Notice that the above forms are all used when we are given the time; there are alternative forms for both velocity and acceleration to be used when we are given the displacement.
\begin{itemize}
  \item $v = \pm \omega \sqrt{x_0^2 - x^2}$
  \item $a = -\omega^2 x$
\end{itemize}

\section{S.H.M. Energy Equations}

\begin{itemize}
  \item When the oscillator passes through the equilibrium position, it has maximum speed and thus maximum kinetic energy (all of the initial energy is transferrd into KE).
  \item At the extreme positions, the oscillator has maximum potential (either gravitational or elastic) energy and zero kinetic energy.
\end{itemize}

The following diagram summarizes the energy changes in an oscillator starting from the equilibrium position.
\img{energy1.png}{0.7}{Energy changes in an oscillator}{energy}
Recall that we have an expression for the speed of the oscillator, we can thus find an expression for the kinetic energy at any displacement $x$:
\begin{equation}
  E_K = \frac{1}{2}m\omega^2\paren{x_0^2 - x^2}
\end{equation}
Also, the total energy is given by the KE at the fastest point, i.e. at $x = 0$:
\begin{equation}
  \sum E = \frac{1}{2}m\omega^2x_0^2
\end{equation}
We can now obtain the potential energy:
\begin{equation}
  E_P = \sum E - E_K = \frac{1}{2}m\omega^2x^2
\end{equation}

\begin{minipage}[t]{0.5\textwidth}
  On the right is the graph of the variation of KE and PE with displacement. Notice that the sum of the two is constant, which is the total energy of the system.\lb
  The point at which KE = PE is slightly off half of the amplitude.
\end{minipage}\hspace*{0.1\textwidth}%
\begin{minipage}[t]{0.4\textwidth}
  \img{rel.png}{1}{Energy changes in an oscillator}{rel}
\end{minipage}


\section{Phase Difference}

Apart from the extreme and the equilibrium positions, the oscillator can be at any position in between. To account for this possibility, we can add in a phase difference $\phi$ to the displacement equation.
$$x = x_0\sin(\omega t + \phi)$$
Mathematically speaking, this is permitted because it is in fact the general solution to the s.h.m. second order differential equation.\lb
Consider the following graph:
\img{shmphase.png}{0.6}{Phase difference in s.h.m.}{shmphase}
The red curve is actually ahead of the blue one, although it appears to be behind. Consider the first red peak: At the time it reaches that peak, the blue curve is still rising and just about to reach that peak. This argument holds for any position on the graph.
\end{document}