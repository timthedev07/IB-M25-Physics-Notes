\documentclass[a4paper,12pt]{article}
\usepackage{setspace}
\doublespacing
\usepackage[backend=biber,style=apa]{biblatex}
\usepackage{sectsty}
\usepackage{siunitx}
\usepackage{graphicx}
\usepackage[a4paper, total={3in, 9in}, textwidth=16cm,bottom=1in,top=1.4in]{geometry}
\usepackage{xcolor}
\usepackage{amsmath}
\usepackage{esvect}
\usepackage{amsthm}
\usepackage{hyperref}
\usepackage{float}
\usepackage{amssymb}
\usepackage{outlines}
\usepackage{caption}
\usepackage{subcaption}
\usepackage{esdiff}
\usepackage{setspace}
\newtheorem{lemma}{Lemma}
\newtheorem{proposition}{Proposition}
\doublespacing
\newcommand{\RNum}[1]{\uppercase\expandafter{\romannumeral #1\relax}}
\let\oldsi\si
\renewcommand{\si}[1]{\oldsi[per-mode=reciprocal-positive-first]{#1}}
\usepackage{enumitem}
\newcommand{\subtitle}[1]{%
  \posttitle{%
    \par\end{center}
    \begin{center}\large#1\end{center}
    \vskip0.5em}%
}
\newcommand{\degsym}{^{\circ}}
\newcommand{\Mod}[1]{\ (\mathrm{mod}\ #1)}
\usepackage{hyperref}
\hypersetup{
  colorlinks,
  citecolor=black,
  filecolor=black,
  linkcolor=black,
  urlcolor=black
}
\newcommand{\lb}{\\[8pt]}
\newenvironment*{cell}[1][]{\begin{tabular}[c]{@{}c@{}}}{\end{tabular}}
\newcommand{\img}[4]{\begin{center}
  \begin{figure}[H]
    \centering
    \includegraphics[width=#2\textwidth]{#1}
    \caption{#3}
    \label{fig:#4}
  \end{figure}
\end{center}}
\newcommand{\doubleimg}[4]{\begin{center}
  \begin{figure}[H]
    \centering
    \begin{subfigure}{.45\textwidth}
      \centering
      \includegraphics[width=1\linewidth]{#1}
      \caption{#2}
      \label{fig:sub1}
    \end{subfigure}
    \begin{subfigure}{.45\textwidth}
      \centering
      \includegraphics[width=1\linewidth]{#3}
      \caption{#4}
      \label{fig:sub2}
    \end{subfigure}
  \end{figure}
\end{center}}
\usepackage{fancyhdr}
\fancyfoot{}
\newcommand{\vect}[3]{\begin{bmatrix}
  #1 \\
  #2 \\
  #3
\end{bmatrix}}
\fancypagestyle{fancy}{\fancyfoot[R]{\vspace*{1.5\baselineskip}\thepage}}
\renewcommand{\contentsname}{Table of Contents}
\newcommand{\angled}[1]{\langle{#1}\rangle}
\newcommand{\paren}[1]{\left(#1\right)}
\newcommand{\sqb}[1]{\left[#1\right]}
\newcommand{\coord}[3]{\angled{#1,\, #2,\, #3}}
\newcommand{\pair}[2]{\paren{#1,\, #2}}
\usepackage{cleveref}
\crefname{lemma}{Lemma}{Lemmas}
\crefname{proposition}{Proposition}{Propositions}

\begin{document}


\pagenumbering{arabic}
\pagestyle{fancy}


\begin{titlepage}
  \begin{center}
    \vspace*{3cm}

    \textbf{\Large  {A4}}

    \vspace{1cm}


    \vfill

    \vspace{1.5cm}

  \end{center}
\end{titlepage}
\pagebreak
\tableofcontents
\pagebreak

\clearpage
\setcounter{page}{1}
\addtocontents{toc}{\protect\thispagestyle{empty}}

\pagebreak

\section{Rotational Equations of Motion}

Essentially, suvat in angular form.

\begin{outline}[enumerate]
  \1 Angular acceleration $\alpha = \diff{\omega}{t}$
  \1 $\omega_1 = \omega_0 + \alpha t$
  \1 $\theta = \omega_0 t + \dfrac{1}{2}\alpha t^2$
  \1 $\omega_1^2 = \omega_0^2 + 2\alpha\theta$
  \1 $\theta = \left(\dfrac{\omega_0 + \omega_1}{2}\right)t$
\end{outline}

\section{Moment of Inertia}

Inertial mass is the resistance to linear acceleration, while moment of inertia is the resistance to angular acceleration.

$$I=\Sigma{kmr^2}$$
where k is a coefficient that depends on the shape of the object.

\pagebreak

\section{Torque}

Rotational equiv. of Newton's Second Law.
$$\tau = I\alpha = Fr\sin\theta$$

\end{document}