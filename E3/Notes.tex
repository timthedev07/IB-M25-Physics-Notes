\documentclass[a4paper,12pt]{article}
\usepackage{setspace}
\usepackage{sectsty}
\usepackage{siunitx}
\usepackage{graphicx}
\usepackage[a4paper, total={3in, 9in}, textwidth=16cm,bottom=1in,top=1.4in]{geometry}
\usepackage[dvipsnames]{xcolor}
\usepackage{amsmath}
\usepackage{esvect}
\usepackage{soul}
\usepackage{amsthm}
\usepackage{hyperref}
\usepackage{float}
\usepackage{amssymb}
\usepackage{outlines}
\usepackage{caption}
\usepackage{fancyvrb}
\usepackage{subcaption}
\usepackage{esdiff}
\usepackage{colortbl}
\usepackage{booktabs}
\usepackage{setspace}
\usepackage{mathtools}
\usepackage{tikz,pgfplots}
\usepackage[most]{tcolorbox}
\usetikzlibrary{positioning,decorations.markings,calc}
\DeclarePairedDelimiter{\ceil}{\lceil}{\rceil}
\newtheorem{lemma}{Lemma}
\newtheorem{proposition}{Proposition}
\newtheorem{remark}{Remark}
\newtheorem{observation}{Observation}
\doublespacing
\let\oldsection\section
\renewcommand\section{\clearpage\oldsection}
\newcommand{\RNum}[1]{\uppercase\expandafter{\romannumeral #1\relax}}
\let\oldsi\si
\renewcommand{\si}[1]{\oldsi[per-mode=reciprocal-positive-first]{#1}}
\usepackage{enumitem}
\newcommand{\subtitle}[1]{%
  \posttitle{%
    \par\end{center}
    \begin{center}\large#1\end{center}
    \vskip0.5em}%
}
\newcommand{\degsym}{^{\circ}}
\newcommand{\Mod}[1]{\ (\mathrm{mod}\ #1)}
\usepackage{hyperref}
\hypersetup{
  colorlinks=true,
  linkcolor = blue
}
\newcommand{\lb}{\\[8pt]}
\newenvironment*{cell}[1][]{\begin{tabular}[c]{@{}c@{}}}{\end{tabular}}
\newcommand{\img}[4]{\begin{center}
  \begin{figure}[H]
    \centering
    \includegraphics[width=#2\textwidth]{#1}
    \caption{#3}
    \label{fig:#4}
  \end{figure}
\end{center}}
\parindent=0pt
\usepackage{fancyhdr}
\fancyfoot{}
\fancypagestyle{fancy}{\fancyfoot[R]{\vspace*{1.5\baselineskip}\thepage}}
\renewcommand{\contentsname}{Table of Contents}
\newcommand{\angled}[1]{\langle{#1}\rangle}
\newcommand{\paren}[1]{\left(#1\right)}
\newcommand{\sqb}[1]{\left[#1\right]}
\newcommand{\coord}[3]{\angled{#1,\, #2,\, #3}}
\newcommand{\pair}[2]{\paren{#1,\, #2}}
\newcommand{\atom}[3]{{}^{#1}_{#2}\text{#3}}
\usepackage[
  noabbrev,
  capitalise,
  nameinlink,
]{cleveref}

\crefname{lemma}{Lemma}{Lemmas}
\crefname{proposition}{Proposition}{Propositions}
\crefname{remark}{Remark}{Remarks}
\crefname{observation}{Observation}{Observations}

\newtcolorbox[auto counter]{prob}[2][]{fonttitle=\bfseries, title=\strut Problem~\thetcbcounter: #2,#1,colback=Orchid!5!white,colframe=Orchid!75!black,top=5mm,bottom=5mm}

\newtcolorbox[auto counter]{rem}[1][]{fonttitle=\bfseries, title=\strut Remark.~\thetcbcounter,colback=purple!5!white,colframe=purple!65!gray,top=5mm,bottom=5mm}

\newtcolorbox[auto counter]{defin}[1][]{fonttitle=\bfseries, title=\strut Definition.~\thetcbcounter,colback=black!5!white,colframe=black!65!gray,top=5mm,bottom=5mm}

\newtcolorbox[auto counter]{obs}[1][]{fonttitle=\bfseries, title=\strut Observation.~\thetcbcounter,colback=RedViolet!5!white,colframe=RedViolet!65!gray,top=5mm,bottom=5mm}

\newtcolorbox[auto counter]{lem}[1][]{fonttitle=\bfseries, title=\strut Lemma.~\thetcbcounter,colback=Maroon!5!white,colframe=Maroon!65!gray,top=5mm,bottom=5mm}

\newtcolorbox[auto counter]{prop}[1][]{fonttitle=\bfseries, title=\strut Proposition.~\thetcbcounter,colback=RedOrange!5!white,colframe=RedOrange!65!gray,top=5mm,bottom=5mm}

\newtcolorbox[auto counter]{hint}[1][]{fonttitle=\bfseries, title=\strut Hint.~\thetcbcounter,colback=OliveGreen!5!white,colframe=OliveGreen!75!gray,top=5mm,bottom=5mm}

\setlength{\belowcaptionskip}{-20pt}
\begin{document}


\pagenumbering{arabic}
\pagestyle{fancy}


\begin{titlepage}
  \begin{center}

    \vspace*{8cm}
    \textbf{\Large {IB Physics Topic E3 Radioactive Decay; SL \& HL}} \\
    \vspace*{1cm}
    \large{By timthedev07, M25 Cohort}

  \end{center}
\end{titlepage}

\pagebreak
\tableofcontents
\pagebreak

\clearpage
\setcounter{page}{1}
\addtocontents{toc}{\protect\thispagestyle{empty}}

\section{Isotopes and Isotones}

Definitions:

\begin{itemize}
  \item Isotopes are a set of atoms that have the same proton number but different nucleon numbers; i.e. they have different numbers of neutrons.
  \item Isotones are a set of atoms with the same number of neutrons but different numbers of protons \& electrons.
  \item A nuclide is a distinct kind of atom or nucleus characterized by a specific number of protons and neutrons. E.g. carbon-12 is a nuclide of carbon with 6 protons and 6 neutrons.
\end{itemize}

\section{Radioactive Decay}

It is a random and natural process that happens to any unstable nucleus (iterative process until stability). Before the decay, the nucleus is referred to as the \textbf{parent nuclide}, and after the decay, the nucleus is referred to as the \textbf{daughter nuclide}. The sequence of decays is referred to as a \textbf{radioactive series} or \textbf{decay chain}.\lb
Radioactive decay has the below properties
\begin{itemize}
  \item Arbitrary
  \item Spontaneous --- we cannot influence the rate of decay by changing physical conditions of the sample such as its temperature, pressure, etc.
\end{itemize}

For any radioactive decay to take place, the \textbf{difference in energy} between parent and daughter nuclides must be \textbf{sufficiently large}.

\pagebreak

\subsection{Types of Decay}

\subsubsection{Alpha Decay}

Alpha decay is the process where an unstable nucleus emits an alpha particle (\textbf{two protons, two neutrons}, $^4_2\text{He}$), reducing both atomic number $Z$ and mass number $A$:
\[
  \atom{A}{Z}{X} \rightarrow \atom{A-4}{Z-2}{Y} + \atom{4}{2}{He}
\]

For example, uranium-238 decays as:

\[
  \atom{238}{92}{U} \rightarrow \atom{234}{90}{Th} + \atom{4}{2}{He}
\]

Most of the released energy is transferred to the alpha particle due to its smaller mass.

Momentum is conserved in the system. The momentum of the daughter nucleus and the alpha particle must be equal and opposite:

\[
  M_{\text{daughter}} v_{\text{daughter}} = - M_{\alpha} v_{\alpha}
\]

Because $M_{\text{daughter}} \gg M_{\alpha}$, the alpha particle moves much faster.

They are highly ionizing but \textbf{lose energy quickly}, making them easily stopped by paper or a few centimeters of air. All alpha particles emitted by the same decay event have the \textbf{same initial energy}, and they lose \textbf{roughly a fixed amount of energy per collision} with each air atom as they travel.\lb

The \textbf{mass ratio} of the alpha particle and the daughter/parent nucleus can be accurately estimated by the \textbf{ratio of their nucleon numbers}.

\pagebreak

\subsubsection{Beta Decay}

\begin{itemize}
  \item \textbf{Beta-minus decay} is the process where a neutron decays into a proton, an electron, and an antineutrino. The \textbf{electron is emitted} from the nucleus, the \textbf{proton is retained} within the nucleus, and \textbf{the antineutrino (an antiparticle, signified by the overbar) is emitted to conserve momentum}:
        \[
          \atom{A}{Z}{X} \rightarrow \atom{A}{Z+1}{Y} + e^- + \bar{\nu}_e
        \]
        This occurs when the neutron-to-proton ratio is too high, in which case, it is desired to increase the proton number by losing an electron.

        Antineutrinos have no effective mass or charge.

        The momentum equation in this case does not have a single solution, as there are three emitted particles.
  \item \textbf{Beta-plus decay} is the process where a proton decays into a neutron, a positron, and a neutrino:
        \[
          \atom{A}{Z}{X} \rightarrow \atom{A}{Z-1}{Y} + e^+ + \nu_e
        \]
  \item \textbf{Electron capture} is the process where an electron on the \textbf{inner shell} is captured by the nucleus, converting a proton into a neutron and releasing a neutrino. This is another way to increase the neutron-to-proton ratio. The energy difference between the parent and daughter nuclides is small, and so a positron cannot be created.
        \[
          \atom{A}{Z}{X} + e^- \rightarrow \atom{A}{Z-1}{Y} + \nu_e
        \]
\end{itemize}

An alternative notation for $e^+$ is $\atom{0}{-1}{$\beta$}$, and for $e^-$ is $\atom{0}{+1}{$\beta$}$.

\pagebreak

\subsubsection{Gamma Decay}

\begin{itemize}
  \item Gamma decay is the process where an unstable nucleus emits a gamma ray.
  \item A gamma ray is a high-energy photon, its energy level is usually measured in MeV.
  \item The energy of the gamma ray is equal to the energy difference between the parent and daughter nuclides.
  \item The atomic and mass numbers remain the same.
  \item Can be a subsequent decay after alpha or beta decay, when the daughter nucleus is in an excited state with a surplus of energy.
\end{itemize}

\subsection{Energy Levels}

\subsubsection{Discrete Energy Levels}

This holds true for energy changes in \textbf{gamma} and \textbf{alpha} decay, because the former is simply an emission of photons and the latter is simply an emission of a helium nucleus.

\subsubsection{Continuous Energy Levels}

Beta decay [more notes incoming, HL]

\section{Properties of Ionizing Radiation}

To \textbf{ionize} a material means to remove or add electrons to atoms or molecules within the material, thereby creating ions. When ionizing radiation passes through matter, it can ionize the material, thereby \textbf{transferring energy} from the energetic particle to separate the electron from the ion.
\begin{itemize}
  \item For an alpha/beta particle, the ionization reduces the energy of the particle in a continuous manner.
  \item For a gamma ray, or an emitted photon, the photon is either completely absorbed or will experience a \textbf{frequency shift}, where the frequency drops and wavelength increases.
\end{itemize}
\begin{table}[h!]
  \centering
  \def\arraystretch{2}
  \begin{tabular}{|>{\columncolor{gray!10}}c|c|c|c|}
    \hline
    \rowcolor{blue!30}
    \hline
    \rowcolor{blue!30}
    \textbf{Type} & \textbf{Ionizing Power} & \textbf{Penetration} & \textbf{Range}                                   \\
    \hline
    Alpha         & High                    & Low                  & Few cm in air (absorbed by paper)                \\
    \hline
    Beta          & Moderate                & Moderate             & Several m in air (absorbed by mm of plastic)     \\
    \hline
    Gamma         & Low                     & High                 & Hundreds of m in air (absorbed by lead/concrete) \\
    \hline
  \end{tabular}
  \caption{Comparison of Alpha, Beta, and Gamma Radiation}
\end{table}

\section{Mass Defect and Binding Energy}

\textbf{Binding energy} is the energy that holds the nucleus together. It is also the \textbf{required amount of energy to separate the nucleus} into its constituent protons and neutrons. The excess of energy, above the level of the binding energy, that is supplied to the nucleus, will be converted to the kinetic energy of the emitted particles.\lb
Moreover, when a neutron and a proton collide and combine to form a deuteron, they release energy, in the form of a photon, also at the level of binding energy as they form the bound.\lb
When protons and neutrons (collectively, nucleons) combine to form an atomic nucleus, the total mass of the nucleus is slightly less than the sum of the individual masses of the protons and neutrons. This difference in mass is called the \textbf{mass defect}. It arises because some of the mass of the nucleons is converted into energy \textbf{to bind the nucleus together}, the formalized relation is given by
$$E = mc^2$$
\begin{itemize}
  \item $E$ is the energy released when the nucleus is formed; binding energy
  \item $m$ is the mass defect.
  \item $c$ is the speed of light.
\end{itemize}


\subsection{Variation of Binding Energy per Nucleon}

\section{The Nucleus}

\subsection{Nuclear Mass}

\subsection{The Strong Nuclear Force}

This force is short-range and attractive; it has the following properties
\begin{itemize}
  \item It is $10^{38}$ times as strong as the force of gravity.
  \item This range is only between nucleons; usually 5 $\si{\femto\m}$ (femtometers)
  \item Is only between nucleons (protons and neutrons)
\end{itemize}

Previous models that did not involve neutrons in the nuclide were unable to explain the stability of the nucleus --- the repulsive forces between protons would inherently cause the nucleus to disintegrate. Thus, there must be an attractive force in the nucleus too.

\subsection{Evidence for Strong Nuclear Forces}

\subsection{Nuclear Stability}

\section{Measuring Radioactive Decay}

\subsection{Activity and Count Rate}

\subsection{Background Radiation}

\section{Half Lives}

\subsection{Radioactive Dating}

\section{Decay Chains}


\end{document}