\documentclass[a4paper,12pt]{article}
\usepackage{setspace}
\usepackage{sectsty}
\usepackage{siunitx}
\usepackage{graphicx}
\usepackage[a4paper, total={3in, 9in}, textwidth=16cm,bottom=1in,top=1.4in]{geometry}
\usepackage[dvipsnames,table]{xcolor}
\usepackage{amsmath}
\usepackage{esvect}
\usepackage{soul}
\usepackage{amsthm}
\usepackage{hyperref}
\usepackage{float}
\usepackage{amssymb}
\usepackage{outlines}
\usepackage{caption}
\usepackage{fancyvrb}
\usepackage{subcaption}
\usepackage{esdiff}
\usepackage{setspace}
\usepackage{mathtools}
\usepackage{tikz,pgfplots}
\usepackage{dirtytalk}
\usepackage{draftwatermark}
\usepackage{helvet}
\renewcommand{\familydefault}{\sfdefault}
\usepackage[most]{tcolorbox}
\SetWatermarkText{timthedev07}
\SetWatermarkScale{4}
\SetWatermarkColor[gray]{0.97}
\usetikzlibrary{positioning,decorations.markings,calc}
\DeclarePairedDelimiter{\ceil}{\lceil}{\rceil}
\newtheorem{lemma}{Lemma}
\newtheorem{proposition}{Proposition}
\newtheorem{remark}{Remark}
\newtheorem{observation}{Observation}
\doublespacing
\let\oldsection\section
\renewcommand\section{\clearpage\oldsection}
\newcommand{\RNum}[1]{\uppercase\expandafter{\romannumeral #1\relax}}
\let\oldsi\si
\renewcommand{\si}[1]{\oldsi[per-mode=reciprocal-positive-first]{#1}}
\usepackage{enumitem}
\newcommand{\subtitle}[1]{%
  \posttitle{%
    \par\end{center}
    \begin{center}\large#1\end{center}
    \vskip0.5em}%
}
\newcommand{\degsym}{^{\circ}}
\newcommand{\Mod}[1]{\ (\mathrm{mod}\ #1)}
\usepackage{hyperref}
\hypersetup{
  colorlinks=true,
  linkcolor = blue
}
\newcommand{\lb}{\\[8pt]}
\newenvironment*{cell}[1][]{\begin{tabular}[c]{@{}c@{}}}{\end{tabular}}
\newcommand{\img}[4]{\begin{center}
  \begin{figure}[H]
    \centering
    \includegraphics[width=#2\textwidth]{#1}
    \caption{#3}
    \label{fig:#4}
  \end{figure}
\end{center}}
\parindent=0pt
\usepackage{fancyhdr}
\fancyfoot{}
\newcommand{\vect}[3]{\begin{bmatrix}
  #1 \\
  #2 \\
  #3
\end{bmatrix}}
\fancypagestyle{fancy}{\fancyfoot[R]{\vspace*{1.5\baselineskip}\thepage}}
\renewcommand{\contentsname}{Table of Contents}
\newcommand{\angled}[1]{\langle{#1}\rangle}
\newcommand{\paren}[1]{\left(#1\right)}
\newcommand{\sqb}[1]{\left[#1\right]}
\newcommand{\coord}[3]{\angled{#1,\, #2,\, #3}}
\newcommand{\pair}[2]{\paren{#1,\, #2}}
\usepackage[
  noabbrev,
  capitalise,
  nameinlink,
]{cleveref}
\crefname{lemma}{Lemma}{Lemmas}
\crefname{proposition}{Proposition}{Propositions}
\crefname{remark}{Remark}{Remarks}
\crefname{observation}{Observation}{Observations}

\newtcolorbox[auto counter]{defin}[1][]{fonttitle=\bfseries, title=\strut Definition.~\thetcbcounter,colback=black!5!white,colframe=black!65!gray,top=5mm,bottom=5mm}

\newtcolorbox[auto counter]{obs}[1][]{fonttitle=\bfseries, title=\strut Observation.~\thetcbcounter,colback=RedViolet!5!white,colframe=RedViolet!65!gray,top=5mm,bottom=5mm}

\setlength{\belowcaptionskip}{-20pt}

\begin{document}


\pagenumbering{arabic}
\pagestyle{fancy}


\begin{titlepage}
  \begin{center}

    \vspace*{8cm}
    \textbf{\Large {IB Physics Topic C4 Standing Waves and Resonance; SL \& HL}} \\
    \vspace*{1cm}
    \large{By timthedev07, M25 Cohort}


  \end{center}
\end{titlepage}

\pagebreak
\tableofcontents
\pagebreak

\clearpage
\setcounter{page}{1}
\addtocontents{toc}{\protect\thispagestyle{empty}}

\section{Formation of Standing Waves}

A standing wave originates from the superposition of two waves of the \textbf{same frequency and amplitude} traveling in \textbf{opposite directions}.
\img{standingwaveparts.png}{0.8}{Formation of a standing wave}{standingwaveparts}
In the diagram above, the blue and red waves are two traveling waves of the same frequency and amplitude, moving in opposite directions. The result of their superposition (the summation of the amplitudes of the two waves at every point of overlap) is the standing wave shown in green. The standing wave has \textbf{nodes} (points of zero amplitude; destructive inteference) and \textbf{antinodes} (points of maximum amplitude; constructive interference).\lb
Consider a few examples:
\begin{enumerate}
  \item A guitar string is plucked, and the wave travels along the string. The wave reflects off the end of the string and interferes with the incoming wave. This interference creates a standing wave.
  \item A sound wave is produced in a pipe. The wave reflects off the end of the pipe and interferes with the incoming wave. This interference creates a standing wave.
\end{enumerate}

One important characteristic to distinguish between a traveling and a stationary wave is the \textbf{amplitude}. In a traveling wave, the amplitude is \textbf{the same at all points}. In a standing wave, at each point on the wave, the amplitude is different from that of its neighboring points.\lb
Let a \textit{nodal region} be the region of points between any two adjacent nodes.
\begin{itemize}
  \item All points in a nodal region are in phase (0 phase difference).
  \item Points in nodal region $R_1$ are out of phase with points in one of the two neighboring nodal regions $R_2$ by $\pi$ radians.
\end{itemize}

\subsection{Definition of a Traveling Wave (MS)}

\begin{itemize}
  \item The transfer/propagation of energy/momentum/information
  \item Through oscillations/vibrations of medium/fields
  \item Positions of maximum and minimum amplitude OR crests and troughs travel through a medium
\end{itemize}

\subsection{Harmonics}

A harmonic is a standing wave pattern of a particular frequency. The first harmonic occurs at the fundamental frequency.


\section{Pulse Reflection --- Boundary Conditions}
\img{pulse.png}{0.8}{Pulse reflection}{pulse}

\begin{enumerate}
  \item Fixed end; the reflected wave is inverted.
  \item Free end; the reflected wave is not inverted.
\end{enumerate}

\section{Standing Waves on a String}

\subsection{Fixed-Fixed}

\img{string1.png}{0.65}{Fixed-Fixed}{string1}

For the $n$-th harmonic:
\begin{itemize}
  \item $\lambda_n = \dfrac{2L}{n}$
  \item The fundamental frequency is $f_1 = \dfrac{v}{2L}$
  \item Subsequent harmonics have frequencies $f_n = nf_1 = \dfrac{nv}{2L}$
\end{itemize}

\pagebreak

\subsection{Fixed-Free}

\img{string2.png}{0.65}{Fixed-Free}{string2}

It must be noted that the second harmonic that exists actually has a harmonic number of 3. Similarly, the third harmonic has a harmonic number of 5, and so on.

For the harmonic with harmonic number $n$ (not the $n$th harmonic that exists!), where $n$ is odd:

\begin{itemize}
  \item $\lambda_n = \dfrac{4L}{n}$
  \item The fundamental frequency is $f_1 = \dfrac{v}{4L}$
  \item Subsequent harmonics have frequencies $f_n = nf_1 = \dfrac{nv}{4L}$
\end{itemize}

\pagebreak

\subsection{Free-Free}

\img{string3.png}{0.45}{Free-Free}{string3}

The formulae are identical to those for the fixed-free string. The only difference is the positions of the nodes and antinodes.

\section{Standing Waves in Pipes}

In a pipe, there can also be a standing wave pattern. It is slightly different from the case of strings, but similar analysis can be performed.
Consider the following diagram, where the dots represent vibrating air molecules.
\img{pipe.png}{0.8}{Standing wave in a pipe}{pipe}
\begin{itemize}
  \item The particles oscillate left and right in a fixed position. The amplitude, previously the vertical displacement of a point on the string, is now the range of horizontal movement of the particles.
  \item A free end is an antinode
  \item A fixed end is a node
\end{itemize}

\pagebreak

\img{pipe_summary.png}{0.95}{Summary of standing wave patterns in pipes}{pipe_summary}

\subsection{Closed-Closed}

Notice that this is completely analogous to the fixed-fixed string case.

\begin{itemize}
  \item For the $n$th harmonic, $\lambda_n = \dfrac{2L}{n}$
  \item The fundamental frequency is $f_1 = \dfrac{v}{2L}$
  \item Subsequent harmonics have frequencies $f_n = nf_1 = \dfrac{nv}{2L}$
\end{itemize}

\subsection{Closed-Open}

Notice that this is completely analogous to the fixed-free string case.

In this case, the even harmonics are omitted.
\begin{itemize}
  \item For the harmonic numbered $n$, $\lambda_n = \dfrac{4L}{n}$
  \item The fundamental frequency is $f_1 = \dfrac{v}{4L}$
  \item Subsequent harmonics have frequencies $f_n = nf_1 = \dfrac{nv}{4L}$
\end{itemize}

\subsection{Open-Open}

This is completely analogous to the free-free string case as well as the closed-closed pipe case.

\section{Summary --- Standing Wave Equations}

The following table works for both the string and pipe cases, since both are analogous.

\begin{table}[H]
  \centering
  \def\arraystretch{1.5}
  \begin{tabular}{|c|c | c|}\hline
    Case                                           & Wavelength                  & Frequency                     \\ \hline
    Both ends same condition; $n \in \mathbb{Z}^+$ & $\lambda_n = \dfrac{2L}{n}$ & $f_n = nf_1 = \dfrac{nv}{2L}$ \\ \hline
    Two ends different conditions; $n$ is odd      & $\lambda_n = \dfrac{4L}{n}$ & $f_n = nf_1 = \dfrac{nv}{4L}$ \\ \hline
  \end{tabular}
\end{table}

\section{Resonance}

Resonance occurs when the driving frequency $f_D$ is equal to the natural frequency $f_0$ of the system. The amplitude of the system increases significantly; without damping, the amplitude would tend towards infinity, which is impossible.
\img{resonance.png}{0.4}{Resonance}{resonance}
The driven oscillator may initially oscillate at a different frequency than the driving frequency, but it will eventually reach the driving frequency.
\begin{itemize}
  \item \textit{Natural frequency}: The frequency at which a system oscillates on its own.
  \item \textit{Driving frequency}: The frequency of an external force driving the system.
\end{itemize}
\pagebreak

\subsection{Resonance with Damping}

\img{resonancewithdamping.png}{0.4}{Resonance with damping}{resonancewithdamping}
Let's now investigate this refined graph with damping considered.
\begin{itemize}
  \item Each red curve of the family represents an oscillatory system with a different level of damping.
  \item The lower the peak of a red curve, the higher the damping associated with it.
  \item With low damping, the peak occurs closest to the natural frequency. As damping increases, the peak shifts away from the natural frequency and becomes flatter.
\end{itemize}

\pagebreak

\img{damping.png}{0.6}{Damping}{damping}

\begin{itemize}
  \item \textit{Critical damping}: The system returns to equilibrium as quickly as possible without oscillating.
  \item \textit{Heavy damping}: The system returns to equilibrium slowly without oscillating.
  \item \textit{Light damping}: The system returns to equilibrium slowly with oscillations; the amplitude decreases exponentially (decay) with time.
\end{itemize}

\pagebreak

\subsection{Pros and Cons of Resonance}

\begin{table}[H]
  \centering

  \def\arraystretch{1.1}
  \begin{tabular}{|p{0.45\textwidth}|p{0.45\textwidth}|}\hline
    \rowcolor{Magenta!30!white} \textbf{Pros}                                      & \textbf{Cons}                                                                                                                       \\ \hline
    Microwave Ovens: Resonance excites water molecules to heat food efficiently.   & Structural Damage: Bridges like Tacoma Narrows and Millennium Footbridge suffered instability due to resonance; fixed with dampers. \\ \hline
    Ozone Layer: Resonance absorbs harmful UV radiation, protecting living tissue. & Mechanical Vibrations: Unwanted resonance in vehicle mirrors, engines, and washing machines causes noise and wear.                  \\ \hline
    Nuclear Magnetic Resonance (NMR): Used in MRI for medical diagnostics.         &                                                                                                                                     \\ \hline
    Lasers: Produced by setting up standing waves at specific light frequencies.   &                                                                                                                                     \\ \hline
  \end{tabular}
\end{table}

\pagebreak

\section{Exam Questions}

\subsection{Tips}

For this topic, I would give the following general tips, taking into account all the past paper questions available.
\begin{enumerate}
  \item As basic as it sounds, get comfortable with $v = \lambda f$ and $f = \dfrac{1}{T}$.
  \item Make sure you know your formulae for the pipes and strings really well, this involves also being able to easily manipulate them to get ratios. E.g. it's important to note that $f_n = nf_1$ and it will surely help in some ratio questions.
  \item Make sure that, given a scenario without information about the configuration of the pipe or string (e.g. open-open, closed-closed, or open-closed), you can reason it out.
  \item You should be able to link this to s.h.m. in certain cases.
  \item You should be able to describe the formation of standing waves in a pipe or string fully.
  \item You should know that in a pipe the particle displacement is horizontal, while on a string it is vertical.
  \item Reason out quantities using given information without relying on formulae. E.g., given the distance between two maxima (not a quantity directly involved in a formula), you should be able to find the wavelength, frequency, speed, etc.
\end{enumerate}

\pagebreak

\subsection{Chocolate in Oven}

In a microwave oven electromagnetic waves are emitted so that a standing wave
pattern is established inside the oven.
A flat piece of chocolate is placed inside the oven and the microwaves are
switched on. The chocolate is stationary.
Melted spots form on the surface of the chocolate. The diagram shows the
pattern of melting on the chocolate. Each square has a length of 1 cm.

\img{ex/1.png}{0.8}{Chocolate in oven}{chocolate}

\begin{enumerate}[label=(\alph*)]
  \item Outline how this standing wave pattern of melted spots is formed.
        \begin{itemize}
          \item Standing waves form in the oven by \hl{superposition / constructive interference}
          \item \hl{Energy transfer} is \hl{greatest} at the \hl{antinodes} of the standing wave pattern
        \end{itemize}
  \item Determine, taking appropriate measurements from the diagram, the frequency of the electromagnetic waves in the oven.
        \begin{itemize}
          \item The key here is to treat the two melted spots on either side as the two free ends of a vibrating string scenario with three antinodes shown. This leads to the equation $\lambda = L = \qty{12.2}{\cm}$. Hence
                $$f = \dfrac{v}{\lambda} = \dfrac{v}{L}$$
                where $v$ is the speed of the wave. The speed of a microwave is $c = \SI{3.00e8}{\m\per\s}$. Hence $$f = \dfrac{\SI{3.00e8}{}}{0.122} = \SI{2.46e9}{\Hz} = \SI{2.46}{\GHz}$$
        \end{itemize}
\end{enumerate}

\pagebreak

\subsection{Generic -- Describing the Formation of Standing Waves}

Adapt the following framework to your question; note that this can also be used to describe intensity variations and some others alike.

\begin{enumerate}
  \item The incoming wave is reflected at the fixed/free end of the string/pipe.
  \item The reflected wave superposes with the incoming wave.
  \item At the points of constructive interference, the amplitude of the wave is doubled, creating antinodes;
  \item At the points of destructive interference, the amplitude is zero, creating nodes.
  \item The superposition of the reflections is reinforced only for certain wavelengths.
\end{enumerate}

\pagebreak

\subsection{Particle Movement -- Graphically}

The solid line represents the standing wave at time $t$ and the dotted line represents the standing wave at an instant later. The dot is the equilibrium position of a particle P in the pipe. The up arrow indicates displacements to the right and the down arrow displacements to the left.

\img{ex/2.png}{0.8}{Graph}{particle}

On the diagram, draw
\begin{enumerate}[label=(\alph*)]
  \item a dot to indicate the approximate position of P at time $t$,
        \begin{itemize}
          \item One must recognize that the standing wave an instant later being sightly below the current curve means that the particle will now move with a negative displacement. We know that in this case, horizontally, the convention is that negative displacement is to the right and the positive displacement is to the left. Hence, the particle must be slightly to the left of the shown position.
                \img{ex/3.png}{0.6}{Graph}{particle}
        \end{itemize}
  \item an arrow to indicate the velocity of P at time $t$.
        \begin{itemize}
          \item Because the particle will move slightly to the left from its original position, its velocity must be to the left as well.
                \img{ex/4.png}{0.6}{Graph}{particle}
        \end{itemize}
\end{enumerate}

\pagebreak

\subsection{Quick-Fire MCQ \# 1}

The author almost fell sleep so decided to put up an easy question just to get his hands back into banging the keyboard. The first harmonic of a standing sound wave is established in a tube with one end open and one end closed. When the length of the tube is increased by 0.10 m the next harmonic is formed. What is the wavelength of the sound?

\begin{enumerate}[label=\Alph*]
  \item 0.10m
  \item 0.13m
  \item 0.20m
  \item 0.40m
\end{enumerate}

\begin{itemize}
  \item If we consider the first two harmonics of this open-free situation, we find that the first harmonic follows $L = \dfrac{\lambda}{4}$ and the second harmonic follows $L = \dfrac{3\lambda}{4}$. The difference between the two lengths is $\dfrac{\lambda}{2} = 0.10 \implies \lambda = 0.20$.
\end{itemize}

\pagebreak

\subsection{Quick-Fire MCQ \# 2}
A standing wave is formed between two loud speakers that emit sound waves of frequency $f$. A student walking between the two loudspeakers finds that the distance between two consecutive sound maxima is 1.5 m. The speed of sound is $\qty{300}{\m\per\s}$.

What if $f$?
\begin{itemize}
  \item If we consider the shape of a single cycle of a sin/cos wave, we can see that the distance between the positive and negative maxima is exactly half the wavelength. Hence, the wavelength is $2 \times 1.5 = 3$.
        \begin{align*}
          f = \dfrac{v}{\lambda} = \dfrac{300}{3} = \qty{100}{\Hz}
        \end{align*}
\end{itemize}

\pagebreak

\subsection{Miscellaneous \# 1}

A string of length 0.80 m is fixed at both ends. The diagram shows a standing wave formed on the string. P and Q are two particles on the string.

\img{ex/5.png}{0.8}{String}{string}

It is suggested that the speed $v$ of waves in the string is related to the tension force $T$ in the string according to the equation $T = av^2$, where $a$ is a constant.

\begin{enumerate}[label=(\alph*)]
  \item \begin{enumerate}[label=(\roman*)]
          \item Draw, on the axes, a graph to show the variation with $t$ of the displacement of particle Q.
                \img{ex/6.png}{0.6}{Answer}{string}
                \begin{itemize}
                  \item Firstly, we must recognize that P is at an antinode and so has the maximum possible displacement. Q is at a position with a smaller amplitude, so the graph must also be compressed vertically to show this.
                  \item Also, because two points in adjacent nodal regions are $\pi$ out of phase, the graph must be shifted by $\pi$ radians, which means that our Q graph should start at its negative peak at $t = 0$.
                \end{itemize}
        \end{enumerate}
  \item The tension force on the string is doubled. Describe the effect, if any, of this change on the frequency of the standing wave.
        \begin{itemize}
          \item We have
                $$f \propto v \implies f^2 \propto \frac{T}{a} $$
          \item Hence, if the tension is doubled, the frequency of the standing wave will be amplified by a factor of $\sqrt{2}$.
        \end{itemize}
\end{enumerate}

\pagebreak

\subsection{Quick-Fire MCQ \# 3}

A pipe of length $L$ is closed at one end. Another pipe is open at both ends and has length $2L$. What is the lowest common frequency for the standing waves in the pipes?

\begin{enumerate}[label=\Alph*. ]
  \item $\dfrac{v_\text{air}}{8L}$
  \item $\dfrac{v_\text{air}}{4L}$
  \item $\dfrac{v_\text{air}}{2L}$
  \item $\dfrac{v_\text{air}}{L}$
\end{enumerate}

\begin{itemize}
  \item The first pipe has a fundamental frequency of $\dfrac{v}{4L}$ and the second pipe has a fundamental frequency of $\dfrac{v}{2(2L)} = \dfrac{v}{4L}$. Hence, the lowest common frequency is in fact their fundamental frequency $\dfrac{v}{4L}$.
  \item Mate I thought this question was harder... I regret including it.
\end{itemize}

\end{document}