\documentclass[a4paper,12pt]{article}
\usepackage{setspace}
\usepackage{sectsty}
\usepackage{siunitx}
\usepackage{graphicx}
\usepackage[a4paper, total={3in, 9in}, textwidth=16cm,bottom=1in,top=1.4in]{geometry}
\usepackage[dvipsnames]{xcolor}
\usepackage{amsmath}
\usepackage{esvect}
\usepackage{soul}
\usepackage{amsthm}
\usepackage{hyperref}
\usepackage{longtable}
\usepackage{float}
\usepackage{amssymb}
\usepackage{outlines}
\usepackage{caption}
\usepackage{fancyvrb}
\usepackage{subcaption}
\usepackage{esdiff}
\usepackage{dirtytalk}
\usepackage{colortbl}
\usepackage{booktabs}
\usepackage{setspace}
\usepackage{mathtools}
\usepackage{tikz,pgfplots}
\usepackage[most]{tcolorbox}
\usetikzlibrary{positioning,decorations.markings,arrows.meta,angles,quotes}
\DeclarePairedDelimiter{\ceil}{\lceil}{\rceil}
\newtheorem{lemma}{Lemma}
\newtheorem{proposition}{Proposition}
\newtheorem{remark}{Remark}
\newtheorem{observation}{Observation}
\doublespacing
\let\oldsection\section
\renewcommand\section{\clearpage\oldsection}
\newcommand{\RNum}[1]{\uppercase\expandafter{\romannumeral #1\relax}}
\let\oldsi\si
\renewcommand{\si}[1]{\oldsi[per-mode=reciprocal-positive-first]{#1}}
\usepackage{enumitem}
\newcommand{\subtitle}[1]{%
  \posttitle{%
    \par\end{center}
    \begin{center}\large#1\end{center}
    \vskip0.5em}%
}
\newcommand{\degsym}{^{\circ}}
\newcommand{\eqor}{\quad \text{or} \quad}
\newcommand{\Mod}[1]{\ (\mathrm{mod}\ #1)}
\usepackage{hyperref}
\hypersetup{
  colorlinks=true,
  linkcolor = blue
}
\newcommand{\lb}{\\[8pt]}
\newenvironment*{cell}[1][]{\begin{tabular}[c]{@{}c@{}}}{\end{tabular}}
\newcommand{\img}[4]{\begin{center}
  \begin{figure}[H]
    \centering
    \includegraphics[width=#2\textwidth]{#1}
    \caption{#3}
    \label{fig:#4}
  \end{figure}
\end{center}}
\parindent=0pt
\usepackage{fancyhdr}
\fancyfoot{}
\fancypagestyle{fancy}{\fancyfoot[R]{\vspace*{1.5\baselineskip}\thepage}}
\renewcommand{\contentsname}{Table of Contents}
\newcommand{\angled}[1]{\langle{#1}\rangle}
\newcommand{\paren}[1]{\left(#1\right)}
\newcommand{\sqb}[1]{\left[#1\right]}
\newcommand{\coord}[3]{\angled{#1,\, #2,\, #3}}
\newcommand{\pair}[2]{\paren{#1,\, #2}}
\newcommand{\atom}[3]{{}^{#1}_{#2}\text{#3}}
\usepackage[
  noabbrev,
  capitalise,
  nameinlink,
]{cleveref}

\crefname{lemma}{Lemma}{Lemmas}
\crefname{proposition}{Proposition}{Propositions}
\crefname{remark}{Remark}{Remarks}
\crefname{observation}{Observation}{Observations}

\newtcolorbox[auto counter]{prob}[2][]{fonttitle=\bfseries, title=\strut Problem~\thetcbcounter: #2,#1,colback=Orchid!5!white,colframe=Orchid!75!black,top=5mm,bottom=5mm}

\newtcolorbox[auto counter]{rem}[1][]{fonttitle=\bfseries, title=\strut Remark.~\thetcbcounter,colback=purple!5!white,colframe=purple!65!gray,top=5mm,bottom=5mm}

\newtcolorbox[auto counter]{defin}[1][]{fonttitle=\bfseries, title=\strut Definition.~\thetcbcounter,colback=black!5!white,colframe=black!65!gray,top=5mm,bottom=5mm}

\newtcolorbox[auto counter]{obs}[1][]{fonttitle=\bfseries, title=\strut Observation.~\thetcbcounter,colback=RedViolet!5!white,colframe=RedViolet!65!gray,top=5mm,bottom=5mm}

\newtcolorbox[auto counter]{law}[1][]{fonttitle=\bfseries, title=\strut Law.~\thetcbcounter,colback=Maroon!5!white,colframe=Maroon!65!gray,top=5mm,bottom=5mm}

\newtcolorbox[auto counter]{prop}[1][]{fonttitle=\bfseries, title=\strut Proposition.~\thetcbcounter,colback=RedOrange!5!white,colframe=RedOrange!65!gray,top=5mm,bottom=5mm}

\newtcolorbox[auto counter]{hint}[1][]{fonttitle=\bfseries, title=\strut Hint.~\thetcbcounter,colback=OliveGreen!5!white,colframe=OliveGreen!75!gray,top=5mm,bottom=5mm}

\newcommand{\assref}[1]{\textcolor{orange!100!black!90}{assumption #1}}

\setlength{\belowcaptionskip}{-20pt}
\begin{document}


\pagenumbering{arabic}
\pagestyle{fancy}


\begin{titlepage}
  \begin{center}

    \vspace*{8cm}
    \textbf{\Large {IB Physics Topic B4 Thermodynamics; SL \& HL}} \\
    \vspace*{1cm}
    \large{By timthedev07, M25 Cohort}

  \end{center}
\end{titlepage}

\pagebreak
\tableofcontents
\pagebreak

\clearpage
\setcounter{page}{1}
\addtocontents{toc}{\protect\thispagestyle{empty}}

\section{System and Surroundings}

Definitions:
\begin{itemize}
  \item \textbf{System}: A system is a portion of the universe that has been chosen for study. Put simply, it is a set of objects to be analyzed. Conversely, the universe is a system that is made up of sub-systems within it.
        \begin{enumerate}
          \item A \textbf{closed system} is one that \hl{does not exchange matter} with its surroundings. I.e. the quantity of matter in the system is constant.
          \item An \textbf{isolated system} has even stricter requirements --- it is one where \hl{neither matter nor energy} can be exchanged with the surroundings.
        \end{enumerate}
  \item In contrast, the \textbf{surroundings} are everything external to the system that may also interact with it.
\end{itemize}

\section{First Law of Thermodynamics}

\begin{law}
  \begin{equation}\label{eq:1stlaw}
    \Delta U = Q - W
  \end{equation}
  \begin{itemize}
    \item $\Delta U$ is the change in internal energy of the system.
    \item $Q$ is the energy supplied to the system.
    \item $W$ is the work done by the system.
  \end{itemize}
  By the \textbf{Clausius sign convention}:
  \begin{enumerate}
    \item $Q > 0$ when energy is transferred to the system, and vice versa.
    \item $\Delta U > 0$ when the internal energy of the system increases, and vice versa.
    \item $W > 0$ when work is done by the system \textbf{on the surroundings}, and vice versa.
  \end{enumerate}
\end{law}

This is a result of energy conservation. The \hl{internal energy of a system changes} when:
\begin{enumerate}
  \item The system does work, or work has been done on the system.
  \item Energy is transferred to or from the system when there are temperature differences between the system and the surroundings.
\end{enumerate}
An example of a measure of the internal energy of a system is the temperature of an ideal gas.

\pagebreak

\subsection{Piston -- Common Scenario Analysis}

The three quantities in \cref{eq:1stlaw} in this particular system are
\begin{itemize}
  \item $\Delta U$: The \textbf{change in temperature}, and this can be converted using the temperature equation under section 4.3 (ideal gas kinetic model) in B3.
  \item $Q$: Any thermal energy supplied to the gas, or inversely, the thermal energy lost by the gas.
  \item $W$: The work done by the gas on the piston (e.g. when the gas expands to push the piston), or the work done on the gas by the piston (e.g. when the gas is compressed).
\end{itemize}


\section{Pressure-Volume Graphs}

Consider the following situation
\begin{itemize}
  \item The system consists of an \textbf{ideal gas} in a cylinder with a movable piston.
  \item We assume, for the sake of simplicity, that the gas is kept at a \textbf{constant pressure}. This of course requires energy to be somehow transferred from the surroundings to the gas.
\end{itemize}

As the gas expands, the following occurs:
\begin{enumerate}
  \item The volume of the gas increases.
  \item The expansion pushes the piston upwards, doing work on the surroundings.
  \item
\end{enumerate}

\section{Types of Changes in a Gas}

\subsection{Isobaric Change}

\subsection{Isovolumetric Change}

\subsection{Isothermal Change}

\subsection{Adiabatic Change}

\section{Heat Cycle and Engines}

\subsection{Refrigerators and Heat Pumps}

\section{The Second Law of Thermodynamics}

\section{Entropy}

\subsection{A Macroscopic Interpretation}

\subsection{A Microscopic Interpretation}

\end{document}