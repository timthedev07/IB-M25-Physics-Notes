\documentclass[a4paper,12pt]{article}
\usepackage{setspace}
\usepackage{sectsty}
\usepackage{siunitx}
\usepackage{graphicx}
\usepackage[a4paper, total={3in, 9in}, textwidth=16cm,bottom=1in,top=1.4in]{geometry}
\usepackage[dvipsnames]{xcolor}
\usepackage{amsmath}
\usepackage{esvect}
\usepackage{soul}
\usepackage{amsthm}
\usepackage{hyperref}
\usepackage{longtable}
\usepackage{float}
\usepackage{draftwatermark}
\usepackage{amssymb}
\usepackage{outlines}
\usepackage{caption}
\usepackage{fancyvrb}
\usepackage{subcaption}
\usepackage{esdiff}
\usepackage{dirtytalk}
\usepackage{colortbl}
\usepackage{booktabs}
\usepackage{setspace}
\usepackage{mathtools}
\usepackage{tikz,pgfplots}
\usepackage[most]{tcolorbox}
\SetWatermarkText{timthedev07}
\SetWatermarkScale{4}
\SetWatermarkColor[gray]{0.97}
\usetikzlibrary{positioning,decorations.markings,arrows.meta,angles,quotes}
\DeclarePairedDelimiter{\ceil}{\lceil}{\rceil}
\newtheorem{lemma}{Lemma}
\newtheorem{proposition}{Proposition}
\newtheorem{remark}{Remark}
\newtheorem{observation}{Observation}
\doublespacing
\let\oldsection\section
\renewcommand\section{\clearpage\oldsection}
\newcommand{\RNum}[1]{\uppercase\expandafter{\romannumeral #1\relax}}
\let\oldsi\si
\renewcommand{\si}[1]{\oldsi[per-mode=reciprocal-positive-first]{#1}}
\usepackage{enumitem}
\newcommand{\subtitle}[1]{%
  \posttitle{%
    \par\end{center}
    \begin{center}\large#1\end{center}
    \vskip0.5em}%
}
\newcommand{\degsym}{^{\circ}}
\newcommand{\eqor}{\quad \text{or} \quad}
\newcommand{\eqand}{\quad \text{and} \quad}
\newcommand{\Mod}[1]{\ (\mathrm{mod}\ #1)}
\usepackage{hyperref}
\hypersetup{
  colorlinks=true,
  linkcolor = blue
}
\newcommand{\lb}{\\[8pt]}
\newenvironment*{cell}[1][]{\begin{tabular}[c]{@{}c@{}}}{\end{tabular}}
\newcommand{\img}[4]{\begin{center}
  \begin{figure}[H]
    \centering
    \includegraphics[width=#2\textwidth]{#1}
    \caption{#3}
    \label{fig:#4}
  \end{figure}
\end{center}}
\parindent=0pt
\usepackage{fancyhdr}
\fancyfoot{}
\fancypagestyle{fancy}{\fancyfoot[R]{\vspace*{1.5\baselineskip}\thepage}}
\renewcommand{\contentsname}{Table of Contents}
\newcommand{\angled}[1]{\langle{#1}\rangle}
\newcommand{\paren}[1]{\left(#1\right)}
\newcommand{\sqb}[1]{\left[#1\right]}
\newcommand{\coord}[3]{\angled{#1,\, #2,\, #3}}
\newcommand{\pair}[2]{\paren{#1,\, #2}}
\newcommand{\atom}[3]{{}^{#1}_{#2}\text{#3}}
\usepackage[
  noabbrev,
  capitalise,
  nameinlink,
]{cleveref}
\newcolumntype{P}[1]{>{\centering\arraybackslash}p{#1}}

\crefname{lemma}{Lemma}{Lemmas}
\crefname{proposition}{Proposition}{Propositions}
\crefname{remark}{Remark}{Remarks}
\crefname{observation}{Observation}{Observations}

\newtcolorbox[auto counter]{prob}[2][]{fonttitle=\bfseries, title=\strut Problem~\thetcbcounter: #2,#1,colback=Orchid!5!white,colframe=Orchid!75!black,top=5mm,bottom=5mm}

\newtcolorbox[auto counter]{rem}[1][]{fonttitle=\bfseries, title=\strut Remark.~\thetcbcounter,colback=purple!5!white,colframe=purple!65!gray,top=5mm,bottom=5mm}

\newtcolorbox[auto counter]{defin}[1][]{fonttitle=\bfseries, title=\strut Definition.~\thetcbcounter,colback=black!5!white,colframe=black!65!gray,top=5mm,bottom=5mm}

\newtcolorbox[auto counter]{obs}[1][]{fonttitle=\bfseries, title=\strut Observation.~\thetcbcounter,colback=RedViolet!5!white,colframe=RedViolet!65!gray,top=5mm,bottom=5mm}

\newtcolorbox[auto counter]{law}[1][]{fonttitle=\bfseries, title=\strut Law.~\thetcbcounter,colback=Maroon!5!white,colframe=Maroon!65!gray,top=5mm,bottom=5mm}

\newtcolorbox[auto counter]{prop}[1][]{fonttitle=\bfseries, title=\strut Proposition.~\thetcbcounter,colback=RedOrange!5!white,colframe=RedOrange!65!gray,top=5mm,bottom=5mm}

\newtcolorbox[auto counter]{hint}[1][]{fonttitle=\bfseries, title=\strut Hint.~\thetcbcounter,colback=OliveGreen!5!white,colframe=OliveGreen!75!gray,top=5mm,bottom=5mm}

\newcommand{\assref}[1]{\textcolor{orange!100!black!90}{assumption #1}}

\newcommand{\tripleimg}[9]{
  \begin{minipage}{0.3\textwidth}
    \img{#1}{1}{#2}{#3}
  \end{minipage}%
  \hspace*{0.05\textwidth}%
  \begin{minipage}{0.3\textwidth}
    \img{#4}{1}{#5}{#6}
  \end{minipage}%
  \hspace*{0.05\textwidth}%
  \begin{minipage}{0.3\textwidth}
    \img{#7}{1}{#8}{#9}
  \end{minipage}%
}

\setlength{\belowcaptionskip}{-20pt}
\begin{document}


\pagenumbering{arabic}
\pagestyle{fancy}


\begin{titlepage}
  \begin{center}

    \vspace*{8cm}
    \textbf{\Large {IB Physics Topic B4 Thermodynamics; SL \& HL}} \\
    \vspace*{1cm}
    \large{By timthedev07, M25 Cohort}

  \end{center}
\end{titlepage}

\pagebreak
\tableofcontents
\pagebreak

\clearpage
\setcounter{page}{1}
\addtocontents{toc}{\protect\thispagestyle{empty}}

\section{System and Surroundings}

Definitions:
\begin{itemize}
  \item \textbf{System}: A system is a portion of the universe that has been chosen for study. Put simply, it is a set of objects to be analyzed. Conversely, the universe is a system that is made up of sub-systems within it.
        \begin{enumerate}
          \item A \textbf{closed system} is one that \hl{does not exchange matter} with its surroundings. I.e. the quantity of matter in the system is constant.
          \item An \textbf{isolated system} has even stricter requirements --- it is one where \hl{neither matter nor energy} can be exchanged with the surroundings.
        \end{enumerate}
  \item In contrast, the \textbf{surroundings} are everything external to the system that may also interact with it.
\end{itemize}

\section{First Law of Thermodynamics}

\begin{law}
  \begin{equation}\label{eq:1stlaw}
    \Delta U = Q - W
  \end{equation}
  \begin{itemize}
    \item $\Delta U$ is the change in internal energy of the system.
    \item $Q$ is the energy supplied to the system.
    \item $W$ is the work done by the system.
  \end{itemize}
  By the \textbf{Clausius sign convention}:
  \begin{enumerate}
    \item $Q > 0$ when energy is transferred to the system, and vice versa.
    \item $\Delta U > 0$ when the internal energy of the system increases, and vice versa.
    \item $W > 0$ when work is done by the system \textbf{on the surroundings}, and vice versa.
  \end{enumerate}
\end{law}

This is a result of energy conservation. The \hl{internal energy of a system changes} when:
\begin{enumerate}
  \item The system does work, or work has been done on the system.
  \item Energy is transferred to or from the system when there are temperature differences between the system and the surroundings.
\end{enumerate}
An example of a measure of the internal energy of a system is the temperature of an ideal gas.

\pagebreak

\subsection{Piston -- Common Scenario Analysis}

The three quantities in \cref{eq:1stlaw} in this particular system are
\begin{itemize}
  \item $\Delta U$: The \textbf{change in temperature}, and this can be converted using the temperature equation under section 4.3 (ideal gas kinetic model) in B3.
  \item $Q$: Any thermal energy supplied to the gas, or inversely, the thermal energy lost by the gas.
  \item $W$: The work done by the gas on the piston (e.g. when the gas expands to push the piston), or the work done on the gas by the piston (e.g. when the gas is compressed).
\end{itemize}


\section{Pressure-Volume Graphs}

Consider the following situation
\begin{itemize}
  \item The system consists of an \textbf{ideal gas} in a cylinder with a movable piston.
  \item We assume, for the sake of simplicity, that the gas is kept at a \textbf{constant pressure}. This of course requires energy to be somehow transferred from the surroundings to the gas. Analogously, \hl{there is a positive $Q$}.
\end{itemize}

As the gas expands, the following occurs:
\begin{enumerate}
  \item The volume of the gas increases.
  \item The expansion pushes the piston upwards, doing work on the surroundings. Thus, \hl{there is a positive $W$}.
\end{enumerate}

This work done is given as \begin{equation}\label{eq:pistonwork}
  W = P \Delta V > 0
\end{equation}
\begin{proof}
  Let $\Delta x$ be the distance moved by the piston. Then, the work done is given as $F\Delta x$. Since $F = PA$, where $A$ is the area of the piston, we have $W = PA\Delta x$. Notice that $A\Delta x = \Delta V$, so $W = P\Delta V$.
\end{proof}

\subsection{Graphical Interpretation}

\begin{minipage}{0.35\textwidth}
  \img{pvarea.png}{1}{Pressure-Volume Graph}{pvarea}
\end{minipage}%
\hspace*{0.02\textwidth}%
\begin{minipage}{0.6\textwidth}
  \begin{itemize}
    \item The area under the graph represents the work done by the gas.
    \item The work done is positive when the gas expands, and negative when the gas is compressed. Use the arrow to verify this.
    \item \textbf{Make sure that the axes start at 0}; otherwise, in the case of a false origin, add in the missing bit.
  \end{itemize}

\end{minipage}

\section{Types of Changes in a Gas}

\say{Iso} is a word of greek origin that means \say{same}.


\tripleimg %
{isovolumetric.png}{Isovolumetric Change}{isovolumetric}%
{isothermal.png}{Isothermal Change}{isothermal}%
{adiabatic.png}{Adiabatic Change}{adiabatic}


\subsection{Isobaric Change}

A change in a gas carried out at \hl{constant pressure} throughout.\lb
The gas law that applies in this case, is the \textbf{Charle's Law}, i.e. $\dfrac{V}{T} = \text{constant}$.\lb
On a P-V graph, this is represented by a horizontal line (see \cref{fig:pvarea}).\lb
Note also, that \cref{eq:1stlaw} can be rewritten as
$$\Delta U = Q - P\Delta V$$

\pagebreak

\subsection{Isovolumetric Change}

A change in gas carried out at \hl{constant volume} throughout.\lb
The gas law that applies is \textbf{Gay-Lussac's Law}, i.e. $\dfrac{P}{T} = \text{constant}$.\lb
On a P-V graph, this is represented by a vertical line (see \cref{fig:isovolumetric}).\lb
In this case, the first law is rewritten as
\begin{align*}
  \Delta U = Q - P\Delta V = Q - 0P = Q \\
  \Delta U = Q
\end{align*}

\subsection{Isothermal Change}

A change in gas carried out at \hl{constant temperature} (and hence constant internal energy) throughout.\lb
The gas law that applies is \textbf{Boyle's Law}, i.e. $PV = \text{constant}$.\lb
On a P-V graph, this is represented by an \textbf{isothermal curve}. Each temperature has a different curve (see \cref{fig:isothermal}). Each curve is referred to as an isotherm.\lb
Since the internal energy goes through no change, we have $\Delta U = 0$. Thus, the first law is rewritten as
$$Q = W$$
\begin{enumerate}
  \item The implication is that \textbf{all} the energy supplied to the gas is used to do work.
  \item If $Q, W > 0$, then, the gas is \hl{expanding}, and energy is supplied to the gas, and the gas does work on the surroundings.
  \item If $Q, W < 0$, then, the gas is \hl{compressed}, and energy is lost by the gas, and work is done on the gas by the surroundings.
\end{enumerate}

\pagebreak

Practically, isothermal changes are not possible:
\begin{enumerate}
  \item Perfect isothermal changes would theoretically require infinite time, here's why
        \begin{enumerate}
          \item This requires the pressure change to be \textbf{quasi-static} (meaning it happens in such small, incremental steps that the system remains in thermal equilibrium throughout).
          \item Each tiny change in volume must be \textbf{instantaneously counterbalanced} by an appropriate heat transfer to prevent a temperature drop.
          \item This will then \hl{require each $\Delta V$ to be infinitely small}, which would take infinite time for the entire process to take place.
        \end{enumerate}
  \item A slow enough change can be considered near-isothermal.
\end{enumerate}

\subsection{Adiabatic Change}

A change in gas carried out with \hl{no energy transfer} between the system and the surroundings. I.e the system is \textbf{thermally isolated}. This is done through insulating the system.\lb
In this case, $Q = 0$, and so $\Delta U = -W$. This actually encapsulates two physical situations. It's straightforward that the two quantities have matching magnitudes; let us see why they have opposite signs.
\begin{itemize}
  \item \textbf{Gas compression}: The surrounding is doing work on the gas, and so $W < 0$. The gas is increasing in energy, and thus $\Delta U > 0$.
  \item \textbf{Gas expansion}: The gas is doing work on the surroundings, and so $W > 0$. The gas is losing energy, and thus $\Delta U < 0$.
\end{itemize}

We introduce a new equation that has not been seen before. This is specific to adiabatic changes:
\begin{equation}\label{eq:adiabatic}
  PV^{\frac{5}{3}} = \text{constant} \eqand TV^{\frac{2}{3}} = \text{constant}
\end{equation}
Note that \textbf{the latter requires constant pressure} in addition to the adiabatic condition.

Note that perfect adiabatic changes are also not practically feasible.
\begin{enumerate}
  \item The system must be perfectly insulated, which is impossible.
  \item Another important reason is that a perfect adiabatic change would allow no time for change to take place.
        \begin{enumerate}
          \item The duration of the change must be \textbf{infinitely short} to prevent any heat loss through the boundaries of the system to the surroundings.
          \item Thus, for a perfect adiabatic change to take place, the time taken must be 0.
        \end{enumerate}
  \item Again, rapid enough changes can approximate adiabatic changes.
\end{enumerate}

\subsection{Combining Different Types of Changes}

There are many different ways in which the different types of changes can be combined.

\tripleimg%
{combined0.png}{Isobaric, isovolumetric, isothermal}{combined0}%
{combined1.png}{Isothermal, adiabatic}{combined1}%
{combined2.png}{Adiabatic, isovolumetric}{combined2}

A general strategy is to
\begin{enumerate}
  \item For each change, \hl{recognize the type}.
  \item Then, consider \hl{what is changing} and \hl{what is constant} based on the type of change. Considering the area can help.
  \item Where necessary, \hl{invoke the associated gas laws}.
\end{enumerate}

The table below summarizes the changes

\begin{table}[H]
  \centering
  \begin{tabular}{|P{0.2\textwidth}|c|c|P{0.23\textwidth}|c|}
    \hline
    \rowcolor{cyan!80!black!40} \textbf{Shape} & \textbf{Type} & \textbf{Constant} & \textbf{Possible Variable}    & \textbf{Gas Laws}   \\ \hline
    Horizontal line                            & Isobaric      & Pressure          & Volume                        & Charles'            \\ \hline
    Vertical line                              & Isovolumetric & Volume            & Temperature                   & Gay-Lussac's        \\ \hline
    Isothermal curve                           & Isothermal    & Temperature       & Pressure                      & Boyle's             \\ \hline
    Skipping between two isotherms             & Adiabatic     & Internal energy   & Pressure, volume, temperature & \cref{eq:adiabatic} \\ \hline
  \end{tabular}
\end{table}

\section{Heat Cycle and Engines}


\begin{minipage}{0.35\textwidth}
  \img{engine.png}{1}{Heat Engine}{heatengine}
\end{minipage}\hspace*{0.02\textwidth}%
\begin{minipage}{0.6\textwidth}
  A \textbf{heat engine} is a device that converts \hl{thermal energy into mechanical work}. It takes in heat from a high-temperature source, does work, and releases some wasteful heat to a low-temperature sink.\lb
  Simplified explanation of the mechanism:
  \begin{enumerate}
    \item The engine absorbs $Q_1$ energy from the hotter reservoir.
    \item In the process of doing work, the engine releases $Q_2$ energy to the colder reservoir. This is because it is never 100\% efficient.
    \item The useful work produced by the engine is $W = Q_1 - Q_2$.
  \end{enumerate}
\end{minipage}\lb

The efficiency of this process $\eta$ is given by
\begin{equation}\label{eq:heat_engine_efficiency}
  \eta = \frac{Q_1 - Q_2}{Q_1} = 1 - \dfrac{Q_2}{Q_1}
\end{equation}

\pagebreak

\subsection{The Carnot Cycle}

\img{carnot.png}{0.5}{Carnot cycle; area enclosed is work done by the gas}{carnot}

The Carnot cycle is a theoretical cycle that is the most efficient possible heat engine cycle. It consists of four stages, two isothermal and two adiabatic stages. The stages are as follows:
\begin{enumerate}
  \item AB: Isothermal expansion as $Q_h = Q_1$ is supplied by the hot reservoir.
  \item BC: Adiabatic expansion as the gas continues to expand, and the temperature drops to the lower $T_c$. There is a drop in the internal energy of the system, and the gas does work on the surroundings.
  \item CD: Isothermal compression as $Q_c = Q_2 < 0$ is dumped to the cold reservoir. The surrounding is doing work on the gas.
  \item DA: Adiabatic compression; the gas is compressed, and the temperature rises back to $T_h$, completely returning to its original state.
\end{enumerate}

\pagebreak

The Carnot cycle is reversible:
\begin{center}
  A \textit{reversible process} is one in which a system can be returned to its previous state with only an extremely small change in the system or surroundings.
\end{center}
An alternative definition is
\begin{center}
  A \textit{reversible process} operates continuously in \hl{quasi-static state}.
\end{center}

This means that for a change to be reversible, it must have happened extremely slowly.\lb
By way of illustration, consider the melting of ice:
\begin{enumerate}
  \item Initial state: The ice and its surroundings are both at the same temperature, 0°C. There is no temperature difference between the ice and the environment, so they are in thermal equilibrium.

  \item Slow heat transfer: We now very slowly supply a tiny bit of heat energy to the ice, just enough for a small fraction of the ice to melt. Because the temperature of both the ice and the surroundings is the same, this tiny heat input occurs without a significant temperature gradient, ensuring that the process remains reversible. The ice starts to melt, but only a very small part, and the system stays close to equilibrium the entire time.

  \item Melting continues slowly: As we keep adding heat slowly, more ice melts. At each tiny step, we could reverse the process by removing the exact amount of heat we added. If we did this, the water would freeze back into ice. This reversibility means the system can return to its original state without any net change in entropy for the entire system (ice + surroundings).

  \item End state: Eventually, the entire block of ice melts. The process was done slowly enough that we can consider it a reversible operation. If we reversed the process step-by-step (cooling the water slowly), the water would freeze back into ice, and there would be no leftover changes.
\end{enumerate}

\subsubsection{Efficiency}
In practice, a perfect Carnot engine is not possible. However, one can achieve near-Carnot engines by maximizing efficiency.\lb
The efficiency of the Carnot cycle is given by
$$\eta_{\text{Carnot}} = 1 - \dfrac{T_c}{T_h} = 1 - \dfrac{Q_c}{Q_h}$$
where the temperatures are in Kelvin.\lb
To maximize the efficiency of the engine, one must \hl{maximize the temperature difference} between the hot and cold reservoirs.

\pagebreak

\subsection{Refrigerators and Heat Pumps}

As previously stated, an ideal heat engine is reversible. A diagram of the directions of energy transfer is shown below.

\img{reversecarnot.jpg}{0.7}{Reversed Carnot Cycle}{reversed}

\begin{enumerate}
  \item Isothermal compression (ON): The gas is compressed at a high temperature; to keep the temperature constant, some \hl{heat is released to the hotter reservoir}.
  \item Adiabatic compression (NM): The gas continues to expand, but now without exchanging heat with its surroundings. Thus, the gas cools as it expands. The temperature of the gas drops below the temperature of the cold reservoir.
  \item Isothermal expansion (MP): The gas is now at the same temperature as the cold reservoir. As the gas expands under constant temperature, it absorbs heat from the cold reservoir to maintain the temperature. \hl{Heat is absorbed from the cold reservoir}.
  \item Adiabatic expansion (PO): The gas continues to expand, but now without exchanging heat with its surroundings. This forces it to drop in temperature, returning to the initial state.
\end{enumerate}

The main difference between the Carnot cycle and the reversed Carnot cycle is that the former transfers thermal energy to mechanical work, while the latter transfers mechanical work to thermal energy.

\subsubsection{Refrigerators}

The coils of a refrigerator contain a liquid called the \say{refrigerant}. A good refrigerant has the following properties
\begin{itemize}
  \item low boiling point
  \item high s.l.h. of evaporation
  \item low s.h.c. of liquid
  \item low vapor density
  \item easily liquefiable
\end{itemize}

\img{refrigerator.png}{0.7}{Refrigerator}{refrigerator}
We can match each component in this system to those in the heat engine:
\begin{itemize}
  \item The \say{gas} is the refrigerant.
  \item They \say{work} is done by the compressor and the expansion valve.
  \item The \say{cooler reservoir} is the internal heat exchange coil in the refrigerator; this provides the latent heat that will be extracted and ejected by the refrigerant.
  \item The \say{hotter reservoir} is the external heat exchange coil.
\end{itemize}

The workflow is as follows
\begin{enumerate}
  \item The compressor raises the temperature of the refrigerant (currently a gas)
  \item The refrigerant releases heat to the surroundings through the external coil.
  \item It cools down and condenses into a liquid.
  \item It then passes through the expansion valve, where it expands back into a gas using the latent heat from the internal coil.
  \item This process cools the internal coil, and the cycle repeats.
\end{enumerate}

\pagebreak

\subsection{Heat Pumps}

A heat pump follows the identical mechanism as a refrigerator, but with a different purpose. The goal of a heat pump is to, for example, transfer the heat from the outside of a house to the inside.

\section{The Second Law of Thermodynamics}

\begin{law}
  The second law of thermodynamics states that\\
  \begin{center}
    \begin{minipage}{0.9\textwidth}
      \begin{center}
        heat cannot spontaneously (without external work done on the system) flow from a colder body to a hotter body.
      \end{center}
    \end{minipage}
  \end{center}\vspace*{1cm}

  An alternative definition is due to Kelvin and Planck:\\
  \begin{center}
    \begin{minipage}{0.9\textwidth}
      \begin{center}
        Energy cannot be extracted from a hot object and transferred entirely into work.

      \end{center}

    \end{minipage}
  \end{center}

\end{law}

A consequence of this law is that the efficiency of a heat engine is always less than 100\%, or equivalently, the rejected (dumped) energy $Q_c$ is always greater than 0.

\section{Entropy}

Entropy is a measure of the disorder of a system and it quantifies the amount of energy in a system that is not available to do work.

\subsection{A Macroscopic Interpretation}

On a macroscopic level, the change in entropy ($\si{\joule\per\kelvin}$) for a reversible change in a system is defined as
\begin{equation}\label{eq:macro_entropy}
  \Delta S = \dfrac{\Delta Q}{T}
\end{equation}
where $\Delta Q$ is the heat supplied to the system, and $T$ is the temperature of the system at which the change occurs. This means that
\begin{itemize}
  \item If energy is removed from the system $\Delta Q < 0$, then $\Delta S < 0$, i.e. the entropy decreases.
  \item If energy is added to the system $\Delta Q > 0$, then $\Delta S > 0$, i.e. the entropy increases.
\end{itemize}
Consider a reversible action $A \rightarrow B$
\begin{itemize}
  \item One can calculate the entropy change using \cref{eq:macro_entropy}
  \item The total entropy change for a cycle is 0. This means that $A\rightarrow B \rightarrow A$ has a total entropy change of 0.
\end{itemize}

\pagebreak

\subsubsection{Entropy for Irreversible Changes}

The entropy change for an irreversible change, where heat flows from a hotter body to a colder body (the surroundings) is given by
\begin{equation}\label{eq:irreversible}
  \Delta S =\Delta Q\left(\frac{1}{T_\text{surroundings}} - \frac{1}{T_\text{gas}}\right)
\end{equation}
This change results in an increase in the entropy, and this implies that the \hl{entropy of the universe is always increasing}. This is another way of stating the second law of thermodynamics.

\pagebreak

\subsection{A Microscopic Interpretation}

This definition of entropy is based on the number of possible microstates of a system. Consider a system with $\Omega$ different arrangements (a.k.a. \textbf{microstates}) of its particles, then, the entropy (\textbf{not the entropy change}) of these molecules is defined as

\begin{equation}\label{eq:microentropy}
  S(\Omega) = k_B\ln\Omega
\end{equation}

Consider a substance that originally has a perfect fixed lattice structure.
\begin{enumerate}
  \item Initially, there is just a single microstate, as the particles are all in a fixed position and cannot exchange positions with others to form new arrangements.
  \item The entropy, in this case, is $$S(1) = k_B\ln 1 = 0$$
  \item Remove one atom from the lattice, now there are more than just one microstate. Denote the number of arrangements now as $\Omega > 1$.
  \item The entropy in this case is $$S(\Omega) = k_B\ln\Omega > 0$$
        in fact, it would be a very large number.
\end{enumerate}

\pagebreak

\subsubsection{Alternative Definition}

Another definition is that the change in entropy arises from the energy $\Delta Q$ that has been required to remove the atom from the lattice at temperature $T$. This also leads to
$$
  \Delta S = \dfrac{\Delta Q}{T}
$$


\end{document}