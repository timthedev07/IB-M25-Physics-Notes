\documentclass[a4paper,12pt]{article}
\usepackage{setspace}
\usepackage{sectsty}
\usepackage{siunitx}
\usepackage{graphicx}
\usepackage[a4paper, total={3in, 9in}, textwidth=16cm,bottom=1in,top=1.4in]{geometry}
\usepackage[dvipsnames]{xcolor}
\usepackage{amsmath}
\usepackage{esvect}
\usepackage{soul}
\usepackage{amsthm}
\usepackage{hyperref}
\usepackage{longtable}
\usepackage{float}
\usepackage{amssymb}
\usepackage{outlines}
\usepackage{caption}
\usepackage{fancyvrb}
\usepackage{subcaption}
\usepackage{esdiff}
\usepackage{dirtytalk}
\usepackage{colortbl}
\usepackage{booktabs}
\usepackage{setspace}
\usepackage{mathtools}
\usepackage{tikz,pgfplots}
\usepackage[most]{tcolorbox}
\usetikzlibrary{positioning,decorations.markings,arrows.meta,angles,quotes}
\DeclarePairedDelimiter{\ceil}{\lceil}{\rceil}
\newtheorem{lemma}{Lemma}
\newtheorem{proposition}{Proposition}
\newtheorem{remark}{Remark}
\newtheorem{observation}{Observation}
\doublespacing
\let\oldsection\section
\renewcommand\section{\clearpage\oldsection}
\newcommand{\RNum}[1]{\uppercase\expandafter{\romannumeral #1\relax}}
\let\oldsi\si
\renewcommand{\si}[1]{\oldsi[per-mode=reciprocal-positive-first]{#1}}
\usepackage{enumitem}
\newcommand{\subtitle}[1]{%
  \posttitle{%
    \par\end{center}
    \begin{center}\large#1\end{center}
    \vskip0.5em}%
}
\newcommand{\degsym}{^{\circ}}
\newcommand{\eqor}{\quad \text{or} \quad}
\newcommand{\eqand}{\quad \text{and} \quad}
\newcommand{\Mod}[1]{\ (\mathrm{mod}\ #1)}
\usepackage{hyperref}
\hypersetup{
  colorlinks=true,
  linkcolor = blue
}
\newcommand{\lb}{\\[8pt]}
\newenvironment*{cell}[1][]{\begin{tabular}[c]{@{}c@{}}}{\end{tabular}}
\newcommand{\img}[4]{\begin{center}
  \begin{figure}[H]
    \centering
    \includegraphics[width=#2\textwidth]{#1}
    \caption{#3}
    \label{fig:#4}
  \end{figure}
\end{center}}
\parindent=0pt
\usepackage{fancyhdr}
\fancyfoot{}
\fancypagestyle{fancy}{\fancyfoot[R]{\vspace*{1.5\baselineskip}\thepage}}
\renewcommand{\contentsname}{Table of Contents}
\newcommand{\angled}[1]{\langle{#1}\rangle}
\newcommand{\paren}[1]{\left(#1\right)}
\newcommand{\sqb}[1]{\left[#1\right]}
\newcommand{\coord}[3]{\angled{#1,\, #2,\, #3}}
\newcommand{\pair}[2]{\paren{#1,\, #2}}
\newcommand{\atom}[3]{{}^{#1}_{#2}\text{#3}}
\usepackage[
  noabbrev,
  capitalise,
  nameinlink,
]{cleveref}
\newcolumntype{P}[1]{>{\centering\arraybackslash}p{#1}}

\crefname{lemma}{Lemma}{Lemmas}
\crefname{proposition}{Proposition}{Propositions}
\crefname{remark}{Remark}{Remarks}
\crefname{observation}{Observation}{Observations}

\newtcolorbox[auto counter]{prob}[2][]{fonttitle=\bfseries, title=\strut Problem~\thetcbcounter: #2,#1,colback=Orchid!5!white,colframe=Orchid!75!black,top=5mm,bottom=5mm}

\newtcolorbox[auto counter]{rem}[1][]{fonttitle=\bfseries, title=\strut Remark.~\thetcbcounter,colback=purple!5!white,colframe=purple!65!gray,top=5mm,bottom=5mm}

\newtcolorbox[auto counter]{defin}[1][]{fonttitle=\bfseries, title=\strut Definition.~\thetcbcounter,colback=black!5!white,colframe=black!65!gray,top=5mm,bottom=5mm}

\newtcolorbox[auto counter]{obs}[1][]{fonttitle=\bfseries, title=\strut Observation.~\thetcbcounter,colback=RedViolet!5!white,colframe=RedViolet!65!gray,top=5mm,bottom=5mm}

\newtcolorbox[auto counter]{law}[1][]{fonttitle=\bfseries, title=\strut Law.~\thetcbcounter,colback=Maroon!5!white,colframe=Maroon!65!gray,top=5mm,bottom=5mm}

\newtcolorbox[auto counter]{prop}[1][]{fonttitle=\bfseries, title=\strut Proposition.~\thetcbcounter,colback=RedOrange!5!white,colframe=RedOrange!65!gray,top=5mm,bottom=5mm}

\newtcolorbox[auto counter]{hint}[1][]{fonttitle=\bfseries, title=\strut Hint.~\thetcbcounter,colback=OliveGreen!5!white,colframe=OliveGreen!75!gray,top=5mm,bottom=5mm}

\newcommand{\assref}[1]{\textcolor{orange!100!black!90}{assumption #1}}

\newcommand{\tripleimg}[9]{
  \begin{minipage}{0.3\textwidth}
    \img{#1}{1}{#2}{#3}
  \end{minipage}%
  \hspace*{0.05\textwidth}%
  \begin{minipage}{0.3\textwidth}
    \img{#4}{1}{#5}{#6}
  \end{minipage}%
  \hspace*{0.05\textwidth}%
  \begin{minipage}{0.3\textwidth}
    \img{#7}{1}{#8}{#9}
  \end{minipage}%
}

\setlength{\belowcaptionskip}{-20pt}
\begin{document}


\pagenumbering{arabic}
\pagestyle{fancy}


\begin{titlepage}
  \begin{center}

    \vspace*{8cm}
    \textbf{\Large {IB Physics Topic B4 Thermodynamics; SL \& HL}} \\
    \vspace*{1cm}
    \large{By timthedev07, M25 Cohort}

  \end{center}
\end{titlepage}

\pagebreak
\tableofcontents
\pagebreak

\clearpage
\setcounter{page}{1}
\addtocontents{toc}{\protect\thispagestyle{empty}}

\section{System and Surroundings}

Definitions:
\begin{itemize}
  \item \textbf{System}: A system is a portion of the universe that has been chosen for study. Put simply, it is a set of objects to be analyzed. Conversely, the universe is a system that is made up of sub-systems within it.
        \begin{enumerate}
          \item A \textbf{closed system} is one that \hl{does not exchange matter} with its surroundings. I.e. the quantity of matter in the system is constant.
          \item An \textbf{isolated system} has even stricter requirements --- it is one where \hl{neither matter nor energy} can be exchanged with the surroundings.
        \end{enumerate}
  \item In contrast, the \textbf{surroundings} are everything external to the system that may also interact with it.
\end{itemize}

\section{First Law of Thermodynamics}

\begin{law}
  \begin{equation}\label{eq:1stlaw}
    \Delta U = Q - W
  \end{equation}
  \begin{itemize}
    \item $\Delta U$ is the change in internal energy of the system.
    \item $Q$ is the energy supplied to the system.
    \item $W$ is the work done by the system.
  \end{itemize}
  By the \textbf{Clausius sign convention}:
  \begin{enumerate}
    \item $Q > 0$ when energy is transferred to the system, and vice versa.
    \item $\Delta U > 0$ when the internal energy of the system increases, and vice versa.
    \item $W > 0$ when work is done by the system \textbf{on the surroundings}, and vice versa.
  \end{enumerate}
\end{law}

This is a result of energy conservation. The \hl{internal energy of a system changes} when:
\begin{enumerate}
  \item The system does work, or work has been done on the system.
  \item Energy is transferred to or from the system when there are temperature differences between the system and the surroundings.
\end{enumerate}
An example of a measure of the internal energy of a system is the temperature of an ideal gas.

\pagebreak

\subsection{Piston -- Common Scenario Analysis}

The three quantities in \cref{eq:1stlaw} in this particular system are
\begin{itemize}
  \item $\Delta U$: The \textbf{change in temperature}, and this can be converted using the temperature equation under section 4.3 (ideal gas kinetic model) in B3.
  \item $Q$: Any thermal energy supplied to the gas, or inversely, the thermal energy lost by the gas.
  \item $W$: The work done by the gas on the piston (e.g. when the gas expands to push the piston), or the work done on the gas by the piston (e.g. when the gas is compressed).
\end{itemize}


\section{Pressure-Volume Graphs}

Consider the following situation
\begin{itemize}
  \item The system consists of an \textbf{ideal gas} in a cylinder with a movable piston.
  \item We assume, for the sake of simplicity, that the gas is kept at a \textbf{constant pressure}. This of course requires energy to be somehow transferred from the surroundings to the gas. Analogously, \hl{there is a positive $Q$}.
\end{itemize}

As the gas expands, the following occurs:
\begin{enumerate}
  \item The volume of the gas increases.
  \item The expansion pushes the piston upwards, doing work on the surroundings. Thus, \hl{there is a positive $W$}.
\end{enumerate}

This work done is given as \begin{equation}\label{eq:pistonwork}
  W = P \Delta V > 0
\end{equation}
\begin{proof}
  Let $\Delta x$ be the distance moved by the piston. Then, the work done is given as $F\Delta x$. Since $F = PA$, where $A$ is the area of the piston, we have $W = PA\Delta x$. Notice that $A\Delta x = \Delta V$, so $W = P\Delta V$.
\end{proof}

\subsection{Graphical Interpretation}

\begin{minipage}{0.35\textwidth}
  \img{pvarea.png}{1}{Pressure-Volume Graph}{pvarea}
\end{minipage}%
\hspace*{0.02\textwidth}%
\begin{minipage}{0.6\textwidth}
  \begin{itemize}
    \item The area under the graph represents the work done by the gas.
    \item The work done is positive when the gas expands, and negative when the gas is compressed. Use the arrow to verify this.
    \item \textbf{Make sure that the axes start at 0}; otherwise, in the case of a false origin, add in the missing bit.
  \end{itemize}

\end{minipage}

\section{Types of Changes in a Gas}

\say{Iso} is a word of greek origin that means \say{same}.


\tripleimg %
{isovolumetric.png}{Isovolumetric Change}{isovolumetric}%
{isothermal.png}{Isothermal Change}{isothermal}%
{adiabatic.png}{Adiabatic Change}{adiabatic}


\subsection{Isobaric Change}

A change in a gas carried out at \hl{constant pressure} throughout.\lb
The gas law that applies in this case, is the \textbf{Charle's Law}, i.e. $\dfrac{V}{T} = \text{constant}$.\lb
On a P-V graph, this is represented by a horizontal line (see \cref{fig:pvarea}).\lb
Note also, that \cref{eq:1stlaw} can be rewritten as
$$\Delta U = Q - P\Delta V$$

\pagebreak

\subsection{Isovolumetric Change}

A change in gas carried out at \hl{constant volume} throughout.\lb
The gas law that applies is \textbf{Gay-Lussac's Law}, i.e. $\dfrac{P}{T} = \text{constant}$.\lb
On a P-V graph, this is represented by a vertical line (see \cref{fig:isovolumetric}).\lb
In this case, the first law is rewritten as
\begin{align*}
  \Delta U = Q - P\Delta V = Q - 0P = Q \\
  \Delta U = Q
\end{align*}

\subsection{Isothermal Change}

A change in gas carried out at \hl{constant temperature} (and hence constant internal energy) throughout.\lb
The gas law that applies is \textbf{Boyle's Law}, i.e. $PV = \text{constant}$.\lb
On a P-V graph, this is represented by an \textbf{isothermal curve}. Each temperature has a different curve (see \cref{fig:isothermal}). Each curve is referred to as an isotherm.\lb
Since the internal energy goes through no change, we have $\Delta U = 0$. Thus, the first law is rewritten as
$$Q = W$$
\begin{enumerate}
  \item The implication is that \textbf{all} the energy supplied to the gas is used to do work.
  \item If $Q, W > 0$, then, the gas is \hl{expanding}, and energy is supplied to the gas, and the gas does work on the surroundings.
  \item If $Q, W < 0$, then, the gas is \hl{compressed}, and energy is lost by the gas, and work is done on the gas by the surroundings.
\end{enumerate}

\pagebreak

Practically, isothermal changes are not possible:
\begin{enumerate}
  \item Perfect isothermal changes would theoretically require infinite time, here's why
        \begin{enumerate}
          \item This requires the pressure change to be \textbf{quasi-static} (meaning it happens in such small, incremental steps that the system remains in thermal equilibrium throughout).
          \item Each tiny change in volume must be \textbf{instantaneously counterbalanced} by an appropriate heat transfer to prevent a temperature drop.
          \item This will then \hl{require each $\Delta V$ to be infinitely small}, which would take infinite time for the entire process to take place.
        \end{enumerate}
  \item A slow enough change can be considered near-isothermal.
\end{enumerate}

\subsection{Adiabatic Change}

A change in gas carried out with \hl{no energy transfer} between the system and the surroundings. I.e the system is \textbf{thermally isolated}. This is done through insulating the system.\lb
In this case, $Q = 0$, and so $\Delta U = -W$. This actually encapsulates two physical situations. It's straightforward that the two quantities have matching magnitudes; let us see why they have opposite signs.
\begin{itemize}
  \item \textbf{Gas compression}: The surrounding is doing work on the gas, and so $W < 0$. The gas is increasing in energy, and thus $\Delta U > 0$.
  \item \textbf{Gas expansion}: The gas is doing work on the surroundings, and so $W > 0$. The gas is losing energy, and thus $\Delta U < 0$.
\end{itemize}

We introduce a new equation that has not been seen before. This is specific to adiabatic changes:
\begin{equation}\label{eq:adiabatic}
  PV^{\frac{5}{3}} = \text{constant} \eqand TV^{\frac{5}{3}} = \text{constant}
\end{equation}
Note that \textbf{the latter requires constant pressure} in addition to the adiabatic condition.

Note that perfect adiabatic changes are also not practically feasible.
\begin{enumerate}
  \item The system must be perfectly insulated, which is impossible.
  \item Another important reason is that a perfect adiabatic change would allow no time for change to take place.
        \begin{enumerate}
          \item The duration of the change must be \textbf{infinitely short} to prevent any heat loss through the boundaries of the system to the surroundings.
          \item Thus, for a perfect adiabatic change to take place, the time taken must be 0.
        \end{enumerate}
  \item Again, rapid enough changes can approximate adiabatic changes.
\end{enumerate}

\subsection{Combining Different Types of Changes}

There are many different ways in which the different types of changes can be combined.

\tripleimg%
{combined0.png}{Isobaric, isovolumetric, isothermal}{combined0}%
{combined1.png}{Isothermal, adiabatic}{combined1}%
{combined2.png}{Adiabatic, isovolumetric}{combined2}

A general strategy is to
\begin{enumerate}
  \item For each change, \hl{recognize the type}.
  \item Then, consider \hl{what is changing} and \hl{what is constant} based on the type of change. Considering the area can help.
  \item Where necessary, \hl{invoke the associated gas laws}.
\end{enumerate}

The table below summarizes the changes

\begin{table}[H]
  \centering
  \begin{tabular}{|P{0.2\textwidth}|c|c|P{0.23\textwidth}|c|}
    \hline
    \rowcolor{cyan!80!black!40} \textbf{Shape} & \textbf{Type} & \textbf{Constant} & \textbf{Possible Variable}    & \textbf{Gas Laws}   \\ \hline
    Horizontal line                            & Isobaric      & Pressure          & Volume                        & Charles'            \\ \hline
    Vertical line                              & Isovolumetric & Volume            & Temperature                   & Gay-Lussac's        \\ \hline
    Isothermal curve                           & Isothermal    & Temperature       & Pressure                      & Boyle's             \\ \hline
    Skipping between two isotherms             & Adiabatic     & Internal energy   & Pressure, volume, temperature & \cref{eq:adiabatic} \\ \hline
  \end{tabular}
\end{table}

\section{Heat Cycle and Engines}


\begin{minipage}{0.35\textwidth}
  \img{engine.png}{1}{Heat Engine}{heatengine}
\end{minipage}\hspace*{0.02\textwidth}%
\begin{minipage}{0.6\textwidth}
  A \textbf{heat engine} is a device that converts \hl{thermal energy into mechanical work}. It takes in heat from a high-temperature source, does work, and releases some wasteful heat to a low-temperature sink.\lb
  Simplified explanation of the mechanism:
  \begin{enumerate}
    \item The engine absorbs $Q_1$ energy from the hotter reservoir.
    \item In the process of doing work, the engine releases $Q_2$ energy to the colder reservoir. This is because it is never 100\% efficient.
    \item The useful work produced by the engine is $W = Q_1 - Q_2$.
  \end{enumerate}
\end{minipage}\lb

The efficiency of this process $\eta$ is given by
\begin{equation}\label{eq:heat_engine_efficiency}
  \eta = \frac{Q_1 - Q_2}{Q_1} = 1 - \dfrac{Q_2}{Q_1}
\end{equation}

\pagebreak

\subsection{The Carnot Cycle}



\subsection{Refrigerators and Heat Pumps}

\section{The Second Law of Thermodynamics}

\section{Entropy}

\subsection{A Macroscopic Interpretation}

\subsection{A Microscopic Interpretation}

\end{document}