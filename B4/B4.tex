\documentclass[a4paper,12pt]{article}
\usepackage{setspace}
\usepackage{sectsty}
\usepackage{siunitx}
\usepackage{graphicx}
\usepackage[a4paper, total={3in, 9in}, textwidth=16cm,bottom=1in,top=1.4in]{geometry}
\usepackage[dvipsnames]{xcolor}
\usepackage{amsmath}
\usepackage{esvect}
\usepackage{soul}
\usepackage{amsthm}
\usepackage{hyperref}
\usepackage{longtable}
\usepackage{float}
\usepackage{draftwatermark}
\usepackage{amssymb}
\usepackage{outlines}
\usepackage{caption}
\usepackage{fancyvrb}
\usepackage{subcaption}
\usepackage{esdiff}
\usepackage{dirtytalk}
\usepackage{colortbl}
\usepackage{booktabs}
\usepackage{setspace}
\usepackage{mathtools}
\usepackage{tikz,pgfplots}
\usepackage[most]{tcolorbox}
\SetWatermarkText{timthedev07}
\SetWatermarkScale{4}
\SetWatermarkColor[gray]{0.97}
\usetikzlibrary{positioning,decorations.markings,arrows.meta,angles,quotes}
\DeclarePairedDelimiter{\ceil}{\lceil}{\rceil}
\newtheorem{lemma}{Lemma}
\newtheorem{proposition}{Proposition}
\newtheorem{remark}{Remark}
\newtheorem{observation}{Observation}
\doublespacing
\let\oldsection\section
\renewcommand\section{\clearpage\oldsection}
\newcommand{\RNum}[1]{\uppercase\expandafter{\romannumeral #1\relax}}
\let\oldsi\si
\renewcommand{\si}[1]{\oldsi[per-mode=reciprocal-positive-first]{#1}}
\usepackage{enumitem}
\newcommand{\subtitle}[1]{%
  \posttitle{%
    \par\end{center}
    \begin{center}\large#1\end{center}
    \vskip0.5em}%
}
\newcommand{\degsym}{^{\circ}}
\newcommand{\eqor}{\quad \text{or} \quad}
\newcommand{\eqand}{\quad \text{and} \quad}
\newcommand{\Mod}[1]{\ (\mathrm{mod}\ #1)}
\usepackage{hyperref}
\hypersetup{
  colorlinks=true,
  linkcolor = blue
}
\newcommand{\lb}{\\[8pt]}
\newenvironment*{cell}[1][]{\begin{tabular}[c]{@{}c@{}}}{\end{tabular}}
\newcommand{\img}[4]{\begin{center}
  \begin{figure}[H]
    \centering
    \includegraphics[width=#2\textwidth]{#1}
    \caption{#3}
    \label{fig:#4}
  \end{figure}
\end{center}}
\parindent=0pt
\usepackage{fancyhdr}
\fancyfoot{}
\fancypagestyle{fancy}{\fancyfoot[R]{\vspace*{1.5\baselineskip}\thepage}}
\renewcommand{\contentsname}{Table of Contents}
\newcommand{\angled}[1]{\langle{#1}\rangle}
\newcommand{\paren}[1]{\left(#1\right)}
\newcommand{\sqb}[1]{\left[#1\right]}
\newcommand{\coord}[3]{\angled{#1,\, #2,\, #3}}
\newcommand{\pair}[2]{\paren{#1,\, #2}}
\newcommand{\atom}[3]{{}^{#1}_{#2}\text{#3}}
\usepackage[
  noabbrev,
  capitalise,
  nameinlink,
]{cleveref}
\newcolumntype{P}[1]{>{\centering\arraybackslash}p{#1}}

\crefname{lemma}{Lemma}{Lemmas}
\crefname{proposition}{Proposition}{Propositions}
\crefname{remark}{Remark}{Remarks}
\crefname{observation}{Observation}{Observations}

\newtcolorbox[auto counter]{prob}[2][]{fonttitle=\bfseries, title=\strut Problem~\thetcbcounter: #2,#1,colback=Orchid!5!white,colframe=Orchid!75!black,top=5mm,bottom=5mm}

\newtcolorbox[auto counter]{rem}[1][]{fonttitle=\bfseries, title=\strut Remark.~\thetcbcounter,colback=purple!5!white,colframe=purple!65!gray,top=5mm,bottom=5mm}

\newtcolorbox[auto counter]{defin}[1][]{fonttitle=\bfseries, title=\strut Definition.~\thetcbcounter,colback=black!5!white,colframe=black!65!gray,top=5mm,bottom=5mm}

\newtcolorbox[auto counter]{obs}[1][]{fonttitle=\bfseries, title=\strut Observation.~\thetcbcounter,colback=RedViolet!5!white,colframe=RedViolet!65!gray,top=5mm,bottom=5mm}

\newtcolorbox[auto counter]{law}[1][]{fonttitle=\bfseries, title=\strut Law.~\thetcbcounter,colback=Maroon!5!white,colframe=Maroon!65!gray,top=5mm,bottom=5mm}

\newtcolorbox[auto counter]{prop}[1][]{fonttitle=\bfseries, title=\strut Proposition.~\thetcbcounter,colback=RedOrange!5!white,colframe=RedOrange!65!gray,top=5mm,bottom=5mm}

\newtcolorbox[auto counter]{hint}[1][]{fonttitle=\bfseries, title=\strut Hint.~\thetcbcounter,colback=OliveGreen!5!white,colframe=OliveGreen!75!gray,top=5mm,bottom=5mm}

\newcommand{\assref}[1]{\textcolor{orange!100!black!90}{assumption #1}}

\newcommand{\tripleimg}[9]{
  \begin{minipage}{0.3\textwidth}
    \img{#1}{1}{#2}{#3}
  \end{minipage}%
  \hspace*{0.05\textwidth}%
  \begin{minipage}{0.3\textwidth}
    \img{#4}{1}{#5}{#6}
  \end{minipage}%
  \hspace*{0.05\textwidth}%
  \begin{minipage}{0.3\textwidth}
    \img{#7}{1}{#8}{#9}
  \end{minipage}%
}

\setlength{\belowcaptionskip}{-20pt}
\begin{document}


\pagenumbering{arabic}
\pagestyle{fancy}


\begin{titlepage}
  \begin{center}

    \vspace*{8cm}
    \textbf{\Large {IB Physics Topic B4 Thermodynamics; SL \& HL}} \\
    \vspace*{1cm}
    \large{By timthedev07, M25 Cohort}

  \end{center}
\end{titlepage}

\pagebreak
\tableofcontents
\pagebreak

\clearpage
\setcounter{page}{1}
\addtocontents{toc}{\protect\thispagestyle{empty}}

\section{System and Surroundings}

Definitions:
\begin{itemize}
  \item \textbf{System}: A system is a portion of the universe that has been chosen for study. Put simply, it is a set of objects to be analyzed. Conversely, the universe is a system that is made up of sub-systems within it.
        \begin{enumerate}
          \item A \textbf{closed system} is one that \hl{does not exchange matter} with its surroundings. I.e. the quantity of matter in the system is constant.
          \item An \textbf{isolated system} has even stricter requirements --- it is one where \hl{neither mass nor energy} can be exchanged with the surroundings.
        \end{enumerate}
  \item In contrast, the \textbf{surroundings} are everything external to the system that may also interact with it.
\end{itemize}

\section{First Law of Thermodynamics}

\begin{law}
  \begin{equation}\label{eq:1stlaw}
    Q = \Delta U + W
  \end{equation}
  \begin{itemize}
    \item $\Delta U$ is the change in internal energy of the system.
    \item $Q$ is the energy supplied to the system.
    \item $W$ is the work done by the system.
  \end{itemize}
  By the \textbf{Clausius sign convention}:
  \begin{enumerate}
    \item $Q > 0$ when energy is transferred to the system, and vice versa.
    \item $\Delta U > 0$ when the internal energy of the system increases, and vice versa.
    \item $W > 0$ when work is done by the system \textbf{on the surroundings}, and vice versa.
  \end{enumerate}
\end{law}

This is a result of energy conservation. The \hl{internal energy of a system changes} when:
\begin{enumerate}
  \item The system does work, or work has been done on the system.
  \item Energy is transferred to or from the system when there are temperature differences between the system and the surroundings.
\end{enumerate}
An example of a measure of the internal energy of a system is the temperature of an ideal gas.

\pagebreak

\subsection{Piston -- Common Scenario Analysis}

The three quantities in \cref{eq:1stlaw} in this particular system are
\begin{itemize}
  \item $\Delta U$: The \textbf{change in temperature}, and this can be converted using the temperature equation under section 4.3 (ideal gas kinetic model) in B3.
  \item $Q$: Any thermal energy supplied to the gas, or inversely, the thermal energy lost by the gas.
  \item $W$: The work done by the gas on the piston (e.g. when the gas expands to push the piston), or the work done on the gas by the piston (e.g. when the gas is compressed).
\end{itemize}


\section{Pressure-Volume Graphs}

Consider the following situation
\begin{itemize}
  \item The system consists of an \textbf{ideal gas} in a cylinder with a movable piston.
  \item We assume, for the sake of simplicity, that the gas is kept at a \textbf{constant pressure}. This of course requires energy to be somehow transferred from the surroundings to the gas. Analogously, \hl{there is a positive $Q$}.
\end{itemize}

As the gas expands, the following occurs:
\begin{enumerate}
  \item The volume of the gas increases.
  \item The expansion pushes the piston upwards, doing work on the surroundings. Thus, \hl{there is a positive $W$}.
\end{enumerate}

This work done is given as \begin{equation}\label{eq:pistonwork}
  W = P \Delta V > 0
\end{equation}
\begin{proof}
  Let $\Delta x$ be the distance moved by the piston. Then, the work done is given as $F\Delta x$. Since $F = PA$, where $A$ is the area of the piston, we have $W = PA\Delta x$. Notice that $A\Delta x = \Delta V$, so $W = P\Delta V$.
\end{proof}

\subsection{Graphical Interpretation}

\begin{minipage}{0.35\textwidth}
  \img{pvarea.png}{1}{Pressure-Volume Graph}{pvarea}
\end{minipage}%
\hspace*{0.02\textwidth}%
\begin{minipage}{0.6\textwidth}
  \begin{itemize}
    \item The area under the graph represents the work done by the gas.
    \item The work done is positive when the gas expands, and negative when the gas is compressed. Use the arrow to verify this.
    \item \textbf{Make sure that the axes start at 0}; otherwise, in the case of a false origin, add in the missing bit.
  \end{itemize}

\end{minipage}

\section{Types of Changes in a Gas}

\say{Iso} is a word of greek origin that means \say{same}.


\tripleimg %
{isovolumetric.png}{Isovolumetric Change}{isovolumetric}%
{isothermal.png}{Isothermal Change}{isothermal}%
{adiabatic.png}{Adiabatic Change}{adiabatic}


\subsection{Isobaric Change}

A change in a gas carried out at \hl{constant pressure} throughout.\lb
The gas law that applies in this case, is the \textbf{Charle's Law}, i.e. $\dfrac{V}{T} = \text{constant}$.\lb
On a P-V graph, this is represented by a horizontal line (see \cref{fig:pvarea}).\lb
Note also, that \cref{eq:1stlaw} can be rewritten as
$$\Delta U = Q - P\Delta V$$

\pagebreak

\subsection{Isovolumetric Change}

A change in gas carried out at \hl{constant volume} throughout.\lb
The gas law that applies is \textbf{Gay-Lussac's Law}, i.e. $\dfrac{P}{T} = \text{constant}$.\lb
On a P-V graph, this is represented by a vertical line (see \cref{fig:isovolumetric}).\lb
In this case, the first law is rewritten as
\begin{align*}
  \Delta U = Q - P\Delta V = Q - 0P = Q \\
  \Delta U = Q
\end{align*}

\subsection{Isothermal Change}

A change in gas carried out at \hl{constant temperature} (and hence constant internal energy) throughout.\lb
The gas law that applies is \textbf{Boyle's Law}, i.e. $PV = \text{constant}$.\lb
On a P-V graph, this is represented by an \textbf{isothermal curve}. Each temperature has a different curve (see \cref{fig:isothermal}). Each curve is referred to as an isotherm.\lb
Since the internal energy goes through no change, we have $\Delta U = 0$. Thus, the first law is rewritten as
$$Q = W$$
\begin{enumerate}
  \item The implication is that \textbf{all} the energy supplied to the gas is used to do work.
  \item If $Q, W > 0$, then, the gas is \hl{expanding}, and energy is supplied to the gas, and the gas does work on the surroundings.
  \item If $Q, W < 0$, then, the gas is \hl{compressed}, and energy is lost by the gas, and work is done on the gas by the surroundings.
\end{enumerate}

\pagebreak

Practically, isothermal changes are not possible:
\begin{enumerate}
  \item Perfect isothermal changes would theoretically require infinite time, here's why
        \begin{enumerate}
          \item This requires the pressure change to be \textbf{quasi-static} (meaning it happens in such small, incremental steps that the system remains in thermal equilibrium throughout).
          \item Each tiny change in volume must be \textbf{instantaneously counterbalanced} by an appropriate heat transfer to prevent a temperature drop.
          \item This will then \hl{require each $\Delta V$ to be infinitely small}, which would take infinite time for the entire process to take place.
        \end{enumerate}
  \item A slow enough change can be considered near-isothermal.
\end{enumerate}

\subsection{Adiabatic Change}

A change in gas carried out with \hl{no energy transfer} between the system and the surroundings. I.e the system is \textbf{thermally isolated}. This is done through insulating the system.\lb
In this case, $Q = 0$, and so $\Delta U = -W$. This actually encapsulates two physical situations. It's straightforward that the two quantities have matching magnitudes; let us see why they have opposite signs.
\begin{itemize}
  \item \textbf{Gas compression}: The surrounding is doing work on the gas, and so $W < 0$. The gas is increasing in energy, and thus $\Delta U > 0$.
  \item \textbf{Gas expansion}: The gas is doing work on the surroundings, and so $W > 0$. The gas is losing energy, and thus $\Delta U < 0$.
\end{itemize}

We introduce a new equation that has not been seen before. This is specific to adiabatic changes:
\begin{equation}\label{eq:adiabatic}
  PV^{\frac{5}{3}} = \text{constant} \eqand TV^{\frac{2}{3}} = \text{constant}
\end{equation}

Note that perfect adiabatic changes are also not practically feasible.
\begin{enumerate}
  \item The system must be perfectly insulated, which is impossible.
  \item Another important reason is that a perfect adiabatic change would allow no time for change to take place.
        \begin{enumerate}
          \item The duration of the change must be \textbf{infinitely short} to prevent any heat loss through the boundaries of the system to the surroundings.
          \item Thus, for a perfect adiabatic change to take place, the time taken must be 0.
        \end{enumerate}
  \item Again, rapid enough changes can approximate adiabatic changes.
\end{enumerate}

\subsection{Combining Different Types of Changes}

There are many different ways in which the different types of changes can be combined.

\tripleimg%
{combined0.png}{Isobaric, isovolumetric, isothermal}{combined0}%
{combined1.png}{Isothermal, adiabatic}{combined1}%
{combined2.png}{Adiabatic, isovolumetric}{combined2}

A general strategy is to
\begin{enumerate}
  \item For each change, \hl{recognize the type}.
  \item Then, consider \hl{what is changing} and \hl{what is constant} based on the type of change. Considering the area can help.
  \item Where necessary, \hl{invoke the associated gas laws}.
\end{enumerate}

The table below summarizes the changes

\begin{table}[H]
  \centering
  \begin{tabular}{|P{0.2\textwidth}|c|c|P{0.23\textwidth}|c|}
    \hline
    \rowcolor{cyan!80!black!40} \textbf{Shape} & \textbf{Type} & \textbf{Constant} & \textbf{Gas Laws}   \\ \hline
    Horizontal line                            & Isobaric      & Pressure          & Charles'            \\ \hline
    Vertical line                              & Isovolumetric & Volume            & Gay-Lussac's        \\ \hline
    Isothermal curve                           & Isothermal    & Temperature       & Boyle's             \\ \hline
    Skipping between two isotherms             & Adiabatic     & Internal energy   & \cref{eq:adiabatic} \\ \hline
  \end{tabular}
\end{table}

For any change, the equation $$\frac{P_1V_1}{T_1} = \frac{P_2V_2}{T_2}$$ can be used to relate the initial and final states of the gas.\lb
When the temperature changes, one can find the change in internal energy using the equation $$\Delta U = \frac{3}{2}nR(T_2 - T_1)$$

\section{Heat Cycle and Engines}


\begin{minipage}{0.35\textwidth}
  \img{engine.png}{1}{Heat Engine}{heatengine}
\end{minipage}\hspace*{0.02\textwidth}%
\begin{minipage}{0.6\textwidth}
  A \textbf{heat engine} is a device that converts \hl{thermal energy into mechanical work}. It takes in heat from a high-temperature source, does work, and releases some wasteful heat to a low-temperature sink.\lb
  Simplified explanation of the mechanism:
  \begin{enumerate}
    \item The engine absorbs $Q_1$ energy from the hotter reservoir.
    \item In the process of doing work, the engine releases $Q_2$ energy to the colder reservoir. This is because it is never 100\% efficient.
    \item The useful work produced by the engine is $W = Q_1 - Q_2$.
  \end{enumerate}
\end{minipage}\lb

The efficiency of this process $\eta$ is given by
\begin{equation}\label{eq:heat_engine_efficiency}
  \eta = \frac{Q_1 - Q_2}{Q_1} = 1 - \dfrac{Q_2}{Q_1}
\end{equation}

\pagebreak

\subsection{The Carnot Cycle}

\img{carnot.png}{0.5}{Carnot cycle; area enclosed is work done by the gas}{carnot}

The Carnot cycle is a theoretical cycle that is the most efficient possible heat engine cycle. It consists of four stages, two isothermal and two adiabatic stages. The stages are as follows:
\begin{enumerate}
  \item AB: Isothermal expansion as $Q_h = Q_1$ is supplied by the hot reservoir.
  \item BC: Adiabatic expansion as the gas continues to expand, and the temperature drops to the lower $T_c$. There is a drop in the internal energy of the system, and the gas does work on the surroundings.
  \item CD: Isothermal compression as $Q_c = Q_2 < 0$ is dumped to the cold reservoir. The surrounding is doing work on the gas.
  \item DA: Adiabatic compression; the gas is compressed, and the temperature rises back to $T_h$, completely returning to its original state.
\end{enumerate}

\pagebreak

The Carnot cycle is reversible:
\begin{center}
  A \textit{reversible process} is one in which a system can be returned to its previous state with only an extremely small change in the system or surroundings.
\end{center}
An alternative definition is
\begin{center}
  A \textit{reversible process} operates continuously in \hl{quasi-static state}.
\end{center}

This means that for a change to be reversible, it must have happened extremely slowly.\lb
By way of illustration, consider the melting of ice:
\begin{enumerate}
  \item Initial state: The ice and its surroundings are both at the same temperature, 0°C. There is no temperature difference between the ice and the environment, so they are in thermal equilibrium.

  \item Slow heat transfer: We now very slowly supply a tiny bit of heat energy to the ice, just enough for a small fraction of the ice to melt. Because the temperature of both the ice and the surroundings is the same, this tiny heat input occurs without a significant temperature gradient, ensuring that the process remains reversible. The ice starts to melt, but only a very small part, and the system stays close to equilibrium the entire time.

  \item Melting continues slowly: As we keep adding heat slowly, more ice melts. At each tiny step, we could reverse the process by removing the exact amount of heat we added. If we did this, the water would freeze back into ice. This reversibility means the system can return to its original state without any net change in entropy for the entire system (ice + surroundings).

  \item End state: Eventually, the entire block of ice melts. The process was done slowly enough that we can consider it a reversible operation. If we reversed the process step-by-step (cooling the water slowly), the water would freeze back into ice, and there would be no leftover changes.
\end{enumerate}

\subsubsection{Efficiency}
In practice, a perfect Carnot engine is not possible. However, one can achieve near-Carnot engines by maximizing efficiency.\lb
The efficiency of the Carnot cycle is given by
$$\eta_{\text{Carnot}} = 1 - \dfrac{T_c}{T_h} = 1 - \dfrac{Q_c}{Q_h}$$
where the temperatures are in Kelvin.\lb
To maximize the efficiency of the engine, one must \hl{maximize the temperature difference} between the hot and cold reservoirs.

\pagebreak

\subsection{Refrigerators and Heat Pumps}

As previously stated, an ideal heat engine is reversible. A diagram of the directions of energy transfer is shown below.

\img{reversecarnot.jpg}{0.7}{Reversed Carnot Cycle}{reversed}

\begin{enumerate}
  \item Isothermal compression (NM): The gas is compressed at a high temperature; to keep the temperature constant, some \hl{heat is released to the hotter reservoir}.
  \item Adiabatic expansion (MP): The gas continues to expand, but now without exchanging heat with its surroundings. Thus, the gas cools as it expands. The temperature of the gas drops below the temperature of the cold reservoir.
  \item Isothermal expansion (PO): The gas is now below the temperature of the cold reservoir. As the gas expands under constant temperature, it absorbs heat from the cold reservoir to maintain the temperature. \hl{Heat is absorbed from the cold reservoir}.
  \item Adiabatic compression (ON): The gas is compressed, and the temperature rises back to the original temperature.
\end{enumerate}

The main difference between the Carnot cycle and the reversed Carnot cycle is that the former transfers thermal energy to mechanical work, while the latter transfers mechanical work to thermal energy.

\subsubsection{Refrigerators}

The coils of a refrigerator contain a liquid called the \say{refrigerant}. A good refrigerant has the following properties
\begin{itemize}
  \item low boiling point
  \item high s.l.h. of evaporation
  \item low s.h.c. of liquid
  \item low vapor density
  \item easily liquefiable
\end{itemize}

\img{refrigerator.png}{0.7}{Refrigerator}{refrigerator}
We can match each component in this system to those in the heat engine:
\begin{itemize}
  \item The \say{gas} is the refrigerant.
  \item They \say{work} is done by the compressor and the expansion valve.
  \item The \say{cooler reservoir} is the internal heat exchange coil in the refrigerator; this provides the latent heat that will be extracted and ejected by the refrigerant.
  \item The \say{hotter reservoir} is the external heat exchange coil.
\end{itemize}

The workflow is as follows
\begin{enumerate}
  \item The compressor raises the temperature of the refrigerant (currently a gas)
  \item The refrigerant releases heat to the surroundings through the external coil.
  \item It cools down and condenses into a liquid.
  \item It then passes through the expansion valve, where it expands back into a gas using the latent heat from the internal coil.
  \item This process cools the internal coil, and the cycle repeats.
\end{enumerate}

\pagebreak

\subsection{Heat Pumps}

A heat pump follows the identical mechanism as a refrigerator, but with a different purpose. The goal of a heat pump is to, for example, transfer the heat from the outside of a house to the inside.

\section{The Second Law of Thermodynamics}

\begin{law}
  The second law of thermodynamics states that\\
  \begin{center}
    \begin{minipage}{0.9\textwidth}
      \begin{center}
        heat cannot spontaneously (without external work done on the system) flow from a colder body to a hotter body.
      \end{center}
    \end{minipage}
  \end{center}\vspace*{1cm}

  An alternative definition is due to Kelvin and Planck:\\
  \begin{center}
    \begin{minipage}{0.9\textwidth}
      \begin{center}
        Energy cannot be extracted from a hot object and transferred entirely into work.

      \end{center}

    \end{minipage}
  \end{center}

  The other claim as part of this law is that in an isolated system, the entropy spontaneously increases over time, and the system evolves toward a state of maximum entropy.

\end{law}

A consequence of this law is that the efficiency of a heat engine is always less than 100\%, or equivalently, the rejected (dumped) energy $Q_c$ is always greater than 0.\lb
Example question: A working refrigerator with the door open is placed in a sealed room. The entropy of the room
\begin{enumerate}[label=(\Alph*)]
  \item is zero
  \item decreases
  \item remains unchanged
  \item \textcolor{ForestGreen}{increases}
\end{enumerate}

\section{Entropy}

Entropy is a measure of the disorder of a system and it quantifies the amount of energy in a system that is not available to do work.

\subsection{A Macroscopic Interpretation}

On a macroscopic level, the change in entropy ($\si{\joule\per\kelvin}$) for a reversible change in a system is defined as
\begin{equation}\label{eq:macro_entropy}
  \Delta S = \dfrac{\Delta Q}{T}
\end{equation}
where $\Delta Q$ is the heat supplied to the system, and $T$ is the temperature of the system at which the change occurs. This means that
\begin{itemize}
  \item If energy is removed from the system $\Delta Q < 0$, then $\Delta S < 0$, i.e. the entropy decreases. However, whenever something happens, there can only be more disorder created than order (so entropy must increase), hence, this drop in entropy in this specific place is compensated by a greater increase in entropy somewhere else. This leads to the conclusion that \hl{the entropy of a non-isolated system can only decrease if the entropy of the surroundings increases.}
  \item If energy is added to the system $\Delta Q > 0$, then $\Delta S > 0$, i.e. the entropy increases.
\end{itemize}
Consider a reversible action $A \rightarrow B$
\begin{itemize}
  \item One can calculate the entropy change using \cref{eq:macro_entropy}
  \item The total entropy change for a cycle is 0. This means that $A\rightarrow B \rightarrow A$ has a total entropy change of 0.
\end{itemize}
For a thermodynamic process, the entropy of the universe never decreases during the process. It can remain the same -- in fact, for a reversible change, the entropy of the surrounding remains unchanged due to the idealization and inexistence of disturbance of the surroundings.

\pagebreak

\subsubsection{Entropy for Irreversible Changes}

The entropy change for an irreversible change, where heat flows from a hotter body to a colder body (the surroundings) is given by
\begin{equation}\label{eq:irreversible}
  \Delta S =\Delta Q\left(\frac{1}{T_\text{surroundings}} - \frac{1}{T_\text{gas}}\right)
\end{equation}
This change results in an increase in the entropy, and this implies that the \hl{entropy of the universe is always increasing}. This is another way of stating the second law of thermodynamics.\lb
A key point to note about irreversible changes is that the final state has a greater number of microstates than the initial state.

\pagebreak

\subsection{A Microscopic Interpretation}

This definition of entropy is based on the number of possible microstates of a system. Consider a system with $\Omega$ different arrangements (a.k.a. \textbf{microstates}) of its particles, then, the entropy (\textbf{not the entropy change}) of these molecules is defined as

\begin{equation}\label{eq:microentropy}
  S(\Omega) = k_B\ln\Omega
\end{equation}

Consider a substance that originally has a perfect fixed lattice structure.
\begin{enumerate}
  \item Initially, there is just a single microstate, as the particles are all in a fixed position and cannot exchange positions with others to form new arrangements.
  \item The entropy, in this case, is $$S(1) = k_B\ln 1 = 0$$
  \item Remove one atom from the lattice, now there are more than just one microstate. Denote the number of arrangements now as $\Omega > 1$.
  \item The entropy in this case is $$S(\Omega) = k_B\ln\Omega > 0$$
        in fact, it would be a very large number.
\end{enumerate}

\pagebreak

\subsubsection{Alternative Definition}

Another definition is that the change in entropy arises from the energy $\Delta Q$ that has been required to remove the atom from the lattice at temperature $T$. This also leads to
$$
  \Delta S = \dfrac{\Delta Q}{T}
$$

\pagebreak

\subsection{Microstate vs. Macrostate}

\begin{itemize}
  \item A microstate is one unique arrangement of a system assuming that each entity in the system is distinguishable
  \item A macrostate is an arrangement of microstates which have an identical outcome when individuals are not treated as distinguishable.
  \item Consider a simplified scenario of a system of 10 counters, where each can be red or blue.
  \item \begin{itemize}
          \item A macrostate in this case is the combination of the number of red counters and the number of blue counters. For example, if we denote $(r, b)$ as the combination of numbers, then, a macrostate could be $(3, 7)$.
          \item A microstate is the exact arrangement of the particles, e.g. rrrbbbbbbb is a microstate of the macrostate of $(3, 7)$.
        \end{itemize}
\end{itemize}

\section{Exam Questions}

\subsection{Identifying Processes}

\img{ex/1.png}{0.3}{Cycle}{ex1}

\begin{itemize}
  \item YZ is a \textbf{vertical line}, and so it is an \textbf{isovolumetric} process.
  \item Now, we know that an adiabatic change involves skipping from one isothermal to another and so an adiabatic curve should be steeper than an isothermal curve.
\end{itemize}

\subsection{Microstate Calculations}

In one throw the coins all land heads upwards. The following throw results in 7 heads and 3 tails. Calculate, in terms of $k_B$, the change in entropy between the two throws.
\begin{itemize}
  \item The initial state has a single microstate, whereas the final state has $\binom{10}{3}$ microstates.
  \item The initial entropy is $$S(1) = k_B\ln 1 = 0$$
  \item The final entropy is $$S(\binom{10}{3}) = k_B\ln\binom{10}{3}$$
  \item Then, the change in entropy is $$\Delta S = k_B\ln\binom{10}{3} = 4.8k_B$$
\end{itemize}

\pagebreak

\subsection{Entropy and Irreversible Processes}

The particles of an ideal gas initially occupy one half of an isolated container, whose second half is
initially empty. The gas is then allowed to expand freely into the second half. The diagram shows two
configurations of the gas: the initial configuration A and configuration B, in which equal numbers of
particles occupy each half of the container.
\img{ex/2.png}{0.5}{Two configurations}{ex2}
When a particle moves to a new position within the same half of the container, the microstate of the gas is
considered unchanged. When a particle moves to the other half of the container, a new microstate is
formed.
\begin{enumerate}[label=(\alph*)]
  \item Explain why the gas in configuration B has a greater number of microstates than in A.
        \begin{itemize}
          \item Configuration A has only one microstate
          \item In configuration B, pairs of particles can be swapped between the halves
          \item Every such change gives rise to a new microstate so there is a larger number of microstates in B
        \end{itemize}
  \item Deduce, with reference to entropy, that the expansion of the gas from the initial configuration A is irreversible.
        \begin{itemize}
          \item $S_A = k_B \ln \Omega_A$ and $S_B = k_B \ln \Omega_B$
          \item Since $\Omega_B > \Omega_A$, the entropy in configuration B is greater
          \item A process that results in an increase of entropy in an isolated system is irreversible.
        \end{itemize}

\end{enumerate}
\pagebreak

\subsection{Microstate Calculations (2)}

An isolated system consists of six particles.The total energy of the system is 6E, where E is a constant.The
particles can randomly exchange energy between one another, in integer multiples of E.

\img{ex/3.png}{0.5}{Energy Diagram}{ex3}

The energy diagram shows two possible configurations of the system. Each dot in the diagram represents one particle. In configuration A, one particle has energy 6E and the remaining particles have zero energy. In configuration B, three particles have energies 3E, 2E and E, and the remaining particles have zero energy.

\begin{enumerate}[label=(\alph*)]
  \item State and explain the number of microstates of the system in configuration A.
        \begin{itemize}
          \item 6 microstates
          \item Any of the six particles can be the one of the highest energy
        \end{itemize}
  \item Configuration B has 120 microstates. Calculate the entropy difference between configurations B and A. State the answer in terms of $k_B$.
        \begin{align*}
          S_B - S_A = k_B(\ln 120 - \ln 6) = k_B\ln 20 \approx 3.0 k_B
        \end{align*}
  \item The system is initially in configuration A. Comment, with reference to the second law of thermodynamics and your answer in (c), on the likely evolution of the system.
        \begin{itemize}
          \item The second law predicts that isolated systems spontaneously evolve towards high-entropy states.
          \item From (c), the entropy of B is greater than that of A.
          \item The final state will likely be similar to B / contain relatively many low-energy particles of different
                energies.
        \end{itemize}

\end{enumerate}

\pagebreak

\subsection{May 2019 Paper 3 TZ2 HL Question 9}

\begin{enumerate}[label=(\alph*)]
  \item Show that during an adiabatic expansion of an ideal monatomic gas the temperature $T$ and volume $V$ are given by.
        $$TV^{\frac{2}{3}} = \text{constant}$$
        Solution: We use the equation $PV^{\frac{5}{3}} = \text{constant}$ without needing to prove it.
        \begin{align*}
          P                              & = \frac{nRT}{V}   \\
          (\frac{nRT}{V})V^{\frac{5}{3}} & = \text{constant} \\
          nRTV^{\frac{2}{3}}             & = \text{constant} \\
          TV^{\frac{2}{3}}               & = \text{constant}
        \end{align*}
  \item The diagram shows a Carnot cycle for an ideal monatomic gas.
        \img{ex/4.png}{0.3}{Carnot Cycle}{ex4}
        The highest temperature in the cycle is 620 K and the lowest is 340 K.
        \begin{enumerate}[label=(\roman*)]
          \item Calculate the efficiency of the cycle.
                \begin{align*}
                  \eta & = 1 - \frac{340}{620} = 0.45
                \end{align*}
          \item The work done during the isothermal expansion A $\to$ B is 540 J. Calculate the
                thermal energy that leaves the gas during one cycle.
                \begin{align*}
                  Q_\text{in}                    & = 540\si{\joule} \\
                  Q_\text{out} = 0.55 \times 540 & = 297\si{\joule}
                \end{align*}
                The important bit here is to realize that the question is asking for the heat released and not the useful work done, which means that we want the complement of the efficiency ($1 - 0.45 = 0.5$) to calculate the portion of the total 540 Joules that we need.
          \item Calculate $\dfrac{V_C}{V_B}$ where $V_C$ and $V_B$ are the volumes at C and B respectively.
                \begin{align*}
                  V_CT_C^\frac{3}{2}                                         & = V_BT_B^\frac{3}{2}                             \\
                  \frac{V_C}{V_B} = \left(\frac{T_B}{T_C}\right)^\frac{3}{2} & = \left(\frac{620}{340}\right)^\frac{3}{2} = 2.5
                \end{align*}
        \end{enumerate}
        \begin{enumerate}[label=(\roman*)]
          \item Calculate the change in the entropy of the gas during the change A to B.
                \begin{align*}
                  \Delta S & = \frac{Q}{T} = \frac{540}{620} = 0.87\si{\joule\per\kelvin}
                \end{align*}
          \item Explain, by reference to the second law of thermodynamics, why a real engine
                operating between the temperatures of 620 K and 340 K cannot have an
                efficiency greater than the answer to (b)(i).
                \begin{itemize}
                  \item The idea that the Carnot cycle is an ideal cycle with the maximum possible efficiency.
                  \item A real engine can never work at the ideal Carnot efficiency...
                  \item ... because it would have additional losses due to friction.
                  \item The 2nd of thermodynamics states that it is impossible to
                        convert all the input heat into mechanical work
                \end{itemize}
        \end{enumerate}

\end{enumerate}


\subsection{May 2019 Paper 3 TZ2 HL Question 12}

A heat pump is modelled by the cycle A→B→C→A.

\img{ex/5.png}{0.5}{Heat Pump}{ex5}

The heat pump transfers thermal energy to the interior of a building during processes
C→A and A→B and absorbs thermal energy from the environment during process B→C.
The working substance is an ideal gas.

\begin{enumerate}[label=(\alph*)]
  \item Show that the work done on the gas for the isothermal process C→A is approximately 440 J.
        \begin{itemize}
          \item By counting squares, we estimate the total area under the AC curve to be about 17-18 squares.
          \item Each square is the equivalent of $$0.5 \times 10^{-3} \times 0.5 \times 10^5 = 25\si{\joule}$$
          \item The estimated work done is given by the following, which lies within the acceptable range in the mark scheme $$17 \times 25 = 425\si{\joule}$$
        \end{itemize}
  \item Calculate the
        \begin{enumerate}[label=(\roman*)]
          \item change in internal energy of the gas for the process A→B.
                \begin{align*}
                  \Delta U & = \frac{3}{2}\Delta(PV)                                  \\
                           & = \frac{3}{2} \times (-2.5 \times 10^5) \times (10^{-3}) \\
                           & = -375\si{\joule}
                \end{align*}
          \item temperature at A if the temperature at B is -40°C.
                \begin{align*}
                  T_A & = V_A\left(\frac{T_B}{V_B}\right) \\
                      & = 3.5\times233 = 816\si{\kelvin}
                \end{align*}
        \end{enumerate}
  \item Determine, using the first law of thermodynamics, the total thermal energy transferred to the building during the processes C→A and A→B.
        \begin{itemize}
          \item For the isothermal change CA, the work done has the same sign and magnitude as the heat transferred away from the system. Namely $Q = W$. From part (a), we know that $W = W = 440\si{\joule}$.
          \item For the isovolumetric change from A to B, we have $\Delta U = Q$, this is exactly the negative equivalent of the value calculated in part (b)(i), namely $375\si{\joule}$.
          \item Hence, the total is given by $440 + 375 = 815\si{\joule}$.
        \end{itemize}
  \item Suggest why this cycle is not a suitable model for a working heat pump.
        \begin{itemize}
          \item The temperature changes in the cycle are too large
          \item Energy/power output would be too small
          \item Note: No answer related to efficiency and the idea that a real heat pump has more loss is accepted.
        \end{itemize}
\end{enumerate}

\pagebreak

\subsection{A Nice Cycle}

The cycle consists of an isobaric expansion AB, adiabatic expansion BC, isobaric compression CD and adiabatic compression DA. The cycle is drawn for a quantity of 1.0 mol of monatomic ideal gas.

\img{ex/6.png}{0.8}{Cycle}{ex6}

\begin{enumerate}[label=(\alph*)]
  \item Calculate the maximum temperature of the gas during the cycle.
        \begin{itemize}
          \item We can identify that the point B is at the maximum temperature -- reading off the pressure and volume at that point gives
                \begin{align*}
                  P_BV_B = nRT_B \implies T_B & = \frac{P_BV_B}{nR}                                                                                            \\
                  T_B                         & = \frac{8 \times 10^6 \times 1.6\times 10^{-3}}{1.0 \times 8.31} = 1540 \si{\kelvin} \approx 1500 \si{\kelvin}
                \end{align*}
        \end{itemize}
        \pagebreak

        The following data are given about the work W done by the gas and thermal energy Q transferred to the gas during each change:

        \begin{table}[H]
          \centering
          \begin{tabular}{|ccc|}
            \hline Change & $\boldsymbol{W} / \mathbf{k J}$ & $\boldsymbol{Q} / \mathbf{k J}$ \\
            AB            & 8.23                            & 20.58                           \\
            BC            & 9.11                            & 0                               \\
            CD            & -4.32                           & -10.81                          \\
            DA            & -3.25                           & 0                               \\
            \hline
          \end{tabular}
        \end{table}

  \item Outline why the entropy of the gas remains constant during changes BC and DA.
        \begin{itemize}
          \item From $\Delta S = \frac{\Delta Q}{T}$, if $Q = 0$, then $\Delta S = 0$.
        \end{itemize}
  \item Determine the efficiency of the cycle.
        \begin{itemize}
          \item The efficiency is given as $$\frac{\text{net work done}}{\text{heat input}}$$
          \item We obtain that
                \begin{align*}
                  \text{Net work done} & = 8.23 + 9.11 - 4.32 - 3.25 = 9.77\si{\kilo\joule} \\
                  \text{Heat input}    & = 20.58 \si{\kilo\joule}                           \\
                  \text{Efficiency}    & = \frac{9.77}{20.58} = 0.47
                \end{align*}
        \end{itemize}
\end{enumerate}


\pagebreak

\subsection{M23 Paper 3 SL TZ1 Question 7}

The pV diagram shows a heat engine cycle consisting of adiabatic, isothermal and isovolumetric parts. The working substance of the engine is an ideal gas.

\img{ex/7.png}{0.5}{Cycle}{ex7}

The following data are available:
\begin{itemize}[label={}]
  \item $p_A = 5.00 \times 10^5 \si{\pascal}$
  \item $V_A = 2.00 \times 10^{-3} \si{\metre\cubed}$
  \item $T_A = 602 \si{\kelvin}$
  \item $p_B = 3.00 \times 10^4 \si{\pascal}$
  \item $p_C = 4.60 \times 10^3 \si{\pascal}$
\end{itemize}

\begin{enumerate}[label=(\alph*)]
  \item Suggest why AC is the adiabatic part of the cycle.
        \begin{itemize}
          \item An adiabatic process changes temperature and so must be the steeper curve.
          \item Better alternative:
                \begin{itemize}
                  \item BC results in a decrease in temperature, since AB is isothermal, to return to the original temperature at B, CA must increase the temperature back up to the original value and so must be adiabatic and cannot be isothermal.
                \end{itemize}
        \end{itemize}
  \item Show that the volume at C is $\SI{3.33e-2}{\metre\cubed}$.
        \begin{itemize}
          \item We invoke the equation specifically for adiabatic changes
                \begin{align*}
                  P_AV_A^{\frac{5}{3}} & = P_CV_C^{\frac{5}{3}}                                                                    \\
                  V_C                  & = V_A\left(\frac{P_A}{P_C}\right)^{\frac{3}{5}}                                           \\
                  V_C                  & = 2.00 \times 10^{-3}\left(\frac{5.00 \times 10^5}{4.60 \times 10^3}\right)^{\frac{3}{5}} \\
                                       & = 3.33 \times 10^{-2} \si{\metre\cubed}
                \end{align*}

        \end{itemize}
  \item Suggest, for the change A $\to$ B, whether the entropy of the gas is increasing,
        decreasing or constant.
        \begin{itemize}
          \item Increasing
          \item Because thermal energy is being added to the gas; or equivalently, since $\Delta S = \frac{\Delta Q}{T} = \frac{W}{T}$
                and the area under (W), is positive, the entropy must increase.
        \end{itemize}
  \item Calculate the thermal energy (heat) taken out of the gas from B to C.
        \begin{itemize}
          \item Recall that, for an isovolumetric change, $\Delta U = Q$ and we can use $\Delta U = \frac{3}{2}\Delta PV$
                \begin{align*}
                  \Delta U = \frac{3}{2} (25.4 \times 10^3) \times (3.33 \times 10^{-2}) = 1270 \si{\joule}
                \end{align*}
        \end{itemize}
  \item The highest and lowest temperatures of the gas during the cycle are 602 K and 92 K.
        The efficiency of this engine is about 0.6. Outline how these data are consistent with
        the second law of thermodynamics.
        \begin{itemize}
          \item We calculate the theoretical efficiency
                $$\eta = 1 - \frac{T_C}{T_H} = 1 - \frac{92}{602} = 0.847$$
          \item The real engine has efficiency $0.6 < 0.847$.
        \end{itemize}
\end{enumerate}

\pagebreak

\subsection{M23 Paper 3 TZ2 HL Question 7}

A frictionless piston traps a fixed mass of an ideal gas. The gas undergoes three thermodynamic processes in a cycle.

\img{ex/8.png}{0.5}{Fucking piston mate fuck the questionbank the IBO is a total muppet if you want to make a questionbank for people to use then make it complete and usable yeah have you wondered why this fucking piston is this big? go home and waste your life on that}{ex8}

The initial conditions of the gas at A are:
\begin{center}
  \begin{enumerate}[label={}]
    \item volume = $\SI{0.330}{\meter\cubed}$
    \item pressure = $\SI{129}{\kilo\pascal}$
    \item temperature = $\SI{27.0}{\degreeCelsius}$
  \end{enumerate}
\end{center}

Process AB is an isothermal change, as shown on the pressure volume (pV) diagram, in which the gas expands to three times its initial volume.

\img{ex/9.png}{0.6}{Ugliest-looking graph}{ex9}

\begin{enumerate}[label=(\alph*)]
  \item Calculate the pressure of the gas at B.
        \begin{align*}
          \frac{P_A}{P_B} & = \frac{V_A}{V_B}        \\
          \frac{P_A}{P_B} & = 3                      \\
          P_B             & = \frac{1}{3} \times 129 \\
                          & = \qty{43}{\kilo\pascal}
        \end{align*}


\end{enumerate}

The gas now undergoes adiabatic compression BC until it returns to the initial volume. To complete the cycle, the gas returns to A via the isovolumetric process CA.

\begin{enumerate}[label=(\alph*)]
  \setcounter{enumi}{1}
  \item Sketch, on the pV diagram, the remaining two processes BC and CA that the gas undergoes.
        \begin{itemize}
          \item \textbf{concave} curved from B to locate C with a \textbf{higher pressure} than A
          \item vertical line joining C to A
                \img{ex/10.png}{0.5}{Answer}{ex10}
        \end{itemize}

  \item Show that the temperature of the gas at C is approximately 350 °C.
        \begin{itemize}
          \item We consider the adiabatic change BC
        \end{itemize}
        \begin{align*}
          T_BV_B^{\frac{2}{3}} & = T_CV_C^{\frac{2}{3}}   \\
          T_B                  & = T_A                    \\
          V_C                  & = V_A                    \\
          V_B                  & = 0.99 \si{\meter\cubed} \\
          \implies T_C         & = 624 \si{\kelvin}
        \end{align*}
  \item Explain why the change of entropy for the gas during the process BC is zero.
        \begin{itemize}
          \item An adiabatic change is carried out while the system is thermally isolated and so there is no energy transfer between the system and the surroundings.
          \item Since $\Delta S = \frac{\Delta Q}{T}$ and $\Delta Q = 0$, $\Delta S = 0$.
        \end{itemize}
  \item Explain why the work done by the gas during the isothermal expansion AB is less than the work done on the gas during the adiabatic compression BC.
        \begin{itemize}
          \item Because the work done is the area under the curve and the area under the curve AB is less than the area of the curve BC.
        \end{itemize}
  \item The quantity of trapped gas is 53.2 mol. Calculate the thermal energy removed from the
        gas during process CA.
        \begin{align*}
          \Delta U & = \frac{3}{2}nR\Delta T        \\
                   & = \frac{3}{2}(53.2)(8.31)(324) \\
                   & = 2.15 \times 10^5 \si{\J}
        \end{align*}
\end{enumerate}


\end{document}