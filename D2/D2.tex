\documentclass[a4paper,12pt]{article}
\usepackage{setspace}
\usepackage{sectsty}
\usepackage{siunitx}
\usepackage{graphicx}
\usepackage[a4paper, total={3in, 9in}, textwidth=16cm,bottom=1in,top=1.4in]{geometry}
\usepackage[dvipsnames]{xcolor}
\usepackage{amsmath}
\usepackage{esvect}
\usepackage{soul}
\usepackage{dirtytalk}
\usepackage{amsthm}
\usepackage{hyperref}
\usepackage{float}
\usepackage{amssymb}
\usepackage{outlines}
\usepackage{draftwatermark}
\usepackage{caption}
\usepackage{fancyvrb}
\usepackage{subcaption}
\usepackage{esdiff}
\usepackage{setspace}
\usepackage{mathtools}
\usepackage{tikz,pgfplots}
\usepackage[most]{tcolorbox}
\SetWatermarkText{timthedev07}
\SetWatermarkScale{4}
\SetWatermarkColor[gray]{0.97}
\usetikzlibrary{positioning,decorations.markings,calc}
\DeclarePairedDelimiter{\ceil}{\lceil}{\rceil}
\newtheorem{lemma}{Lemma}
\newtheorem{proposition}{Proposition}
\newtheorem{remark}{Remark}
\newtheorem{observation}{Observation}
\doublespacing
\let\oldsection\section
\renewcommand\section{\clearpage\oldsection}
\newcommand{\RNum}[1]{\uppercase\expandafter{\romannumeral #1\relax}}
\let\oldsi\si
\renewcommand{\si}[1]{\oldsi[per-mode=reciprocal-positive-first]{#1}}
\usepackage{enumitem}
\newcommand{\subtitle}[1]{%
  \posttitle{%
    \par\end{center}
    \begin{center}\large#1\end{center}
    \vskip0.5em}%
}
\newcommand{\degsym}{^{\circ}}
\newcommand{\Mod}[1]{\ (\mathrm{mod}\ #1)}
\usepackage{hyperref}
\hypersetup{
  colorlinks=true,
  linkcolor = blue
}
\newcommand{\lb}{\\[8pt]}
\newenvironment*{cell}[1][]{\begin{tabular}[c]{@{}c@{}}}{\end{tabular}}
\newcommand{\img}[4]{\begin{center}
  \begin{figure}[H]
    \centering
    \includegraphics[width=#2\textwidth]{#1}
    \caption{#3}
    \label{fig:#4}
  \end{figure}
\end{center}}
\parindent=0pt
\usepackage{fancyhdr}
\fancyfoot{}
\newcommand{\vect}[3]{\begin{bmatrix}
  #1 \\
  #2 \\
  #3
\end{bmatrix}}
\fancypagestyle{fancy}{\fancyfoot[R]{\vspace*{1.5\baselineskip}\thepage}}
\renewcommand{\contentsname}{Table of Contents}
\newcommand{\angled}[1]{\langle{#1}\rangle}
\newcommand{\paren}[1]{\left(#1\right)}
\newcommand{\sqb}[1]{\left[#1\right]}
\newcommand{\coord}[3]{\angled{#1,\, #2,\, #3}}
\newcommand{\pair}[2]{\paren{#1,\, #2}}
\usepackage[
  noabbrev,
  capitalise,
  nameinlink,
]{cleveref}
\crefname{lemma}{Lemma}{Lemmas}
\crefname{proposition}{Proposition}{Propositions}
\crefname{remark}{Remark}{Remarks}
\crefname{observation}{Observation}{Observations}

\newtcolorbox[auto counter]{defin}[1][]{fonttitle=\bfseries, title=\strut Definition.~\thetcbcounter,colback=black!5!white,colframe=black!65!gray,top=5mm,bottom=5mm}

\newtcolorbox[auto counter]{obs}[1][]{fonttitle=\bfseries, title=\strut Observation.~\thetcbcounter,colback=RedViolet!5!white,colframe=RedViolet!65!gray,top=5mm,bottom=5mm}

\setlength{\belowcaptionskip}{-20pt}

\begin{document}


\pagenumbering{arabic}
\pagestyle{fancy}


\begin{titlepage}
  \begin{center}

    \vspace*{8cm}
    \textbf{\Large {IB Physics Topic D2 Electric and Magnetic Fields; SL \& HL}} \\
    \vspace*{1cm}
    \large{By timthedev07, M25 Cohort}


  \end{center}
\end{titlepage}

\pagebreak
\tableofcontents
\pagebreak

\clearpage
\setcounter{page}{1}
\addtocontents{toc}{\protect\thispagestyle{empty}}

\section{Electrostatics --- Interaction of Charges}
Like charges attract; opposite charges repel. What else do you expect me to say here?

\section{Conservation of Charge}

The sum of the currents into a junction is equal to the sum of the currents away from the junction.

\img{chargeconservation.png}{0.5}{Conservation of Charge}{chargeconservation}
In the diagram above, the currents flowing into the junction are $I_1$, $I_2$, and $I_3$. The currents flowing out of the junction are $I_4$ and $I_5$. By conservation of charge, $I_1 + I_2 + I_3 = I_4 + I_5$.

\section{Mechanisms of Charge Transfer}

\subsection{Friction}

\begin{itemize}
  \item When two different materials are rubbed together, \hl{electrons can be transferred} from one material to the other.
  \item The objects end up with \hl{opposite charges} --- one has a deficit of electrons and hence a positive overall charge, while the other has an excess of electrons and hence a negative overall charge, causing them to attract each other due to electrostatic forces.
\end{itemize}

\subsection{Contact}

\begin{itemize}
  \item When a \hl{charged object touches a neutral object}, electrons are transferred.
  \item If the charged object has an excess of electrons, some will move to the neutral object; if it has a deficit, it may draw electrons from the neutral object.
  \item After contact, both objects have \hl{similar charges}, though the total charge is shared between them.
\end{itemize}

\subsection{Induction}
\begin{itemize}
  \item Bringing a \hl{charged object close to but without touching a neutral one}.
  \item This charged object causes electrons within the neutral object to \hl{move (but not transferred)}, either attracting them to the surface or repelling them to the other end of the object.
  \item The object temporarily develops opposite charges on opposite sides (polarization). \hl{If grounded, it can be left with a permanent charge opposite to the one on the inducing object}.
\end{itemize}

\section{Electric Fields}

\subsection{Coulomb's Law}

This is the attractive force between two charged objects.

\begin{equation}\label{eq:coulomb}
  F = k\frac{q_1q_2}{r^2}
\end{equation}
where
\begin{itemize}
  \item $k = 8.99 \times 10^9$ is Coulomb's constant.
  \item $q_1$ and $q_2$ are the charges; if they have the same sign, then the force is positive and thus repulsive; conversely, if they have opposite signs, the force is negative and attractive.
\end{itemize}

\subsubsection{Coulomb's Constant}

Coulumb's constant is given by
\begin{equation}\label{eq:coulombconstant}
  k = \frac{1}{4\pi\epsilon_0}
\end{equation}
where
\begin{itemize}
  \item $\epsilon_0 = 8.85 \times 10^{-12}$ is the \hl{permittivity of free space}.
\end{itemize}
The above only applies to calculations in a vacuum. In a different medium of permitivity $\epsilon$, $k$ would be newly defined as $k = \dfrac{1}{4\pi\epsilon}$.

\pagebreak

\subsection{Electric Field Strength}

The electric field strength $E$ is given as
\begin{equation}\label{eq:efield}
  F = Eq \quad \text{ or } \quad E = \frac{F}{q}
\end{equation}

Formally, it is defined as the force per unit charge experienced by a small positive test charge place at that point. This is analogous to the idea of gravitational field strength.\lb
Recall (if you don't, go read my notes) how we defined the gravitational field strength as
$$g = \frac{GM}{r^2}$$
Combining \cref{eq:coulomb} and \cref{eq:efield}, we can derive an analogous expression for the electric field strength experienced by a small positive test charge at a distance $r$ from a charge $Q$:
\begin{equation}\label{eq:efield2}
  E = \frac{kQ}{r^2}
\end{equation}

Observations of the similar forms:
\begin{itemize}
  \item Both are instances of the inverse square law.
  \item Both do not require information about the test object specifically.
  \item Both involve a constant of proportionality, one is $G$ and the other is $k$.
  \item Both are dependent on either the mass or charge of the source.
\end{itemize}

\pagebreak

\subsection{Field Line Patterns}

A general rule of thumb is that field lines always point away from positive charges and towards negative charges. The field lines are always perpendicular to the surface of the charge. Also, the denser the field lines are in a region, the stronger the electric field in that region.\lb
The following two diagrams show the field line patterns for two charges of opposite and similar signs respectively.
\img{fieldlines.jpg}{0.8}{Field Line Patterns}{fieldlines}

The following is the field line pattern between two plates of opposite charges
\img{plates.jpeg}{0.4}{Field Line Patterns for Parallel Plates}{plates}
\begin{itemize}
  \item In the middle, the field lines are parallel and equidistant; this is a region of uniform electric field.
  \item On the edges, the field lines are curved and weaker;  these are known as the edge effects.
\end{itemize}

\subsection{Parallel Plates Equations}

Recall that the definition of potential in general is the \hl{work done \textbf{per unit something} to move a test object from infinity to a point}. Using the $F = Eq$ equation, we can derive the following equation for the \textbf{potential difference} between two plates of opposite charges:
\begin{equation}\label{eq:plates}
  \Delta V_e = Ed
\end{equation}
where $d$ is the separation distance between the plates and $E$ is the field strength (\textbf{not the energy!}). This equation only works for uniform fields.\lb
This equation is not merely used for parallel plates; it applies to the calculation of \hl{any change in potential in a uniform electric field through a distance $d$}.

\subsection{Electron Volt Conversions}

One electronvolt (eV) is \hl{the energy gained by one electron when it moves through a potential difference of one volt}. This is equivalent to $1.6 \times 10^{-19}$ Joules.\lb
Conversely, one Joule is equivalent to $6.24 \times 10^{18}$ electronvolts.\lb
If we consider a particle of charge $ne$ with $n\in \mathbb{N}$ that is accelerated through a potential difference of $V$, the work done on the particle is $neV$.

\pagebreak

\subsection{Field Close to a Conductor}

Consider a plate with area $A$ and total charge $q$, we define its surface charge density as $\sigma = \dfrac{q}{A}$. We desire to find the electric field close to the plate.\lb
It is known that the total charge $q$ is given as the following
\begin{equation}\label{eq:charge}
  q = \frac{VA}{4\pi kd}
\end{equation}

We can then use $E = \dfrac{V}{d}$ to obtain
\begin{align*}
  q & = \frac{V}{d}\times\frac{A}{4\pi k} \\
    & = \frac{EA}{4\pi k}                 \\
  E & = \frac{q}{A} \times 4\pi k         \\
    & =  4\pi k\sigma
\end{align*}
This expression gives the electric field between the two parallel plates, where each plate contributes half and so close to one plate, the field is $2\pi k\sigma$.\lb
Recall that $E = \dfrac{V}{d}$ uses the assumption that the field is uniform; this gives away why we need the "close to the conductor" condition, because at the surface, the surface is locally flat and so the field can be treated as a uniform field.

\pagebreak

\subsection{Millikan's Oil Drop Experiment}

\img{oildrop.jpg}{0.7}{Millikan's Oil Drop Experiment}{oildrop}
Set-up:
\begin{itemize}
  \item Millikan sprayed a fine mist of oil droplets between two parallel, horizontal conducting plates.
  \item  The droplets were tiny enough to remain suspended in the air for a while and were able to be observed through a microscope.
  \item An electric field was applied across the plates by connecting them to a power source, with one plate positively charged and the other negatively charged.
  \item X-rays were used to ionize the air, causing some oil droplets to pick up extra electrons, giving them a net negative charge.
\end{itemize}
Measurements:
\begin{itemize}
  \item Before turning on the electric field, the droplets were allowed to fall and eventually reach a terminal velocity.
  \item At this constant velocity, \hl{weight = drag force + upthrust}.
        \begin{itemize}
          \item The upthrust the buoyancy force acting on the droplet due to the air and can be found by $\rho gV$.
          \item The drag force can be approximated as $6\pi\eta rv$.
        \end{itemize}
  \item Thus, he was able to work out the mass of the droplet.
  \item With the electric field present, the droplets were observed to be eventually suspended in air, at which stage, the forces of the electric field ($F_e = Eq$, upwards) and the weight of the droplet ($w = mg$, downwards) were balanced.
  \item Rearranging gives that the charge on the droplet is $q = mg/E$.
\end{itemize}
Evidence for quantization of charge:
\begin{itemize}
  \item Millikan measured the charge on many droplets and found that these charges were always whole-number multiples of a smallest, consistent value, $e$.
  \item This observation led to the conclusion that charge is quantized --- it comes in discrete packets, rather than being continuous.
  \item Experiments failed to find any charge less than $1.6 \times 10^{-19}$ Coulombs; this is then the value of the elementary charge, $e$.
\end{itemize}
\textbf{Note}: it is helpful to remember the derivation of this experiment --- it may appear in exams. Quite simply it is the highlighted line, from there you can form equations and rearrange to find desired expressions.

\pagebreak

\subsection{Electric Potential}

Recall the definition of electric potential
\begin{center}
  \say{The work done per unit charge to move a positive test charge from infinity to a point.}
\end{center}
Which means that the electric potential different between two points is given by
\begin{equation}\label{eq:potential}
  \Delta V_e = \frac{W}{q}
\end{equation}
where $q$ is the test charge and not the charge creating the field.
The unit for electric potential is the volt, which is equivalent to a Joule per Coulomb ($\text{V} \equiv \si{\J\per\C}$).\lb
The electric potential at a distance $r$ from the center of a field created by a charge $\pm Q$ is given by
\begin{equation}\label{eq:potential__}
  V_e = \pm \frac{kQ}{r}
\end{equation}
The $\pm$ signs just indicate that $Q$ and $V$ have the same sign.

\img{linespotential.png}{0.5}{Potential: Points on field lines}{linespotential}

In the image above, the potential at point Y is greater than the potential at point X because the potential increases in the direction opposite to field strength.

\pagebreak

\subsection{Accelerating an Electron}

Consider an electron accelerated through a potential difference $V$. The work done on the electron is given by
\begin{equation}\label{eq:work}
  W = eV
\end{equation}
where $e$ is the elementary charge. This work done is then converted into kinetic energy, so the maximum speed of the electron is given by
\begin{align*}
  \frac{1}{2}m_ev^2 & = eV                     \\
  v                 & = \sqrt{\frac{2eV}{m_e}}
\end{align*}

\subsubsection{Equipotentials}

\begin{minipage}{0.45\textwidth}
  \img{sphereequipotential.png}{1}{Equipotentials}{sphereequipotential}

  We have met this idea in D1. Around a charge $Q$, the equipotentials are \textit{concentric spheres} not equally spaced. The closer the spheres are to the charge, the closer they are to each other. As before, the electric field lines are always perpendicular to the equipotentials.
\end{minipage}\hspace*{0.1\textwidth}
\begin{minipage}{0.45\textwidth}
  \img{platesequipotential.png}{1}{Equipotentials for Parallel Plates}{platesequipotential}

  For parallel plates, the equipotentials are \textit{parallel lines} equally spaced. The electric field lines are always perpendicular to the equipotentials.
\end{minipage}\lb

It is worth noting that in \cref{fig:platesequipotential}, the most work is done when one moves the positive charge towards the positive plate against the arrows, and conversely, the least work is done when one moves the positive charge towards the negative plate with the arrows.\lb
Also, the motion of the charge in \cref{fig:platesequipotential} is uniformly accelerated, since the field is uniform.\lb
Lastly, it must be noted that in \cref{fig:platesequipotential}, any motion perpendicular to the field lines does not require work to be done; there is no change in potential.

\pagebreak

\subsection{Electric Potential and Field Strength}
Recall, from D1, that gravitational field strength can be given as
$$g = -\diff{V_g}{r}$$
Analogously, we can define the electric field strength as
\begin{equation}\label{eq:potential2}
  E = -\diff{V_e}{r}
\end{equation}
The graphical interpretation is that the electric field strength is the tangential gradient of the electric potential graph at some point.
We can rewrite by integrating both sides:
$$
  \int E\,\mathrm{d}r = -V_e
$$
The graphical interpretation of this is that on a graph of electric field strength against distance, the area under the curve between two points $r_1$ and $r_2$ is the potential difference between the two points.
\img{Ergraph.png}{0.5}{Electric Field Strength/Distance Graph}{Ergraph}

\pagebreak

\subsection{Potential and Fields Inside a Hollow Sphere}

Outside the sphere, the calculations are the same as when we treat the sphere as a point charge.\lb
Inside the sphere, however, the field is zero. This is because the field lines are canceled out by the opposite charges on the inner surface of the sphere.\lb
Recall that $E = -\diff{V_e}{r}$; if the field is zero, then $$E = -\diff{V_e}{r} = 0$$
This means that the potential inside the sphere is constant (since its rate of change is 0) and equal to the potential at the surface of the sphere.

\pagebreak

\subsection{Electric Potential Energy}

The electric potential energy is the work done to bring a charge from infinity to a point in an electric field. Similar to GPE, it is a property of a two-charge system. It is given by
\begin{equation}\label{eq:epe}
  E_p = F_e\times r = k\frac{q_1q_2}{r}
\end{equation}
If we know the potential of charge $q_1$ at position $r$ in the field of $q_2$, we can find the potential energy of charge $q_1$ at that point by multiplying the potential by the charge:
\begin{equation}\label{eq:epe2}
  E_p = q_1V
\end{equation}

\pagebreak

\subsection{Combined Potentials}

First consider two identical point charges $Q_1$ and $Q_2$ at a distance $R$ from each other.
\img{combined.png}{0.9}{Combined Potentials}{combined}
The resulting potential at a point P that is $x$ away from $Q_1$ can be derived as follows:
\begin{align*}
  \sum V = V_1 + V_2 & = \frac{kQ_1}{x} + \frac{kQ_2}{R - x}
\end{align*}
It is worth noting that this graph has a minimum at $x = \dfrac{1}{2} R$
Now consider the case when the two charges are replaced by two identical conducting spheres.
\img{combinedhollow.png}{0.9}{Combined Potentials for Conducting Spheres}{combinedhollow}
The above is the shape of the graph of $\sum V$.\lb
In the region between the two spheres, the graph has the same shape as if the two charges were point charges. However, once reaching into one of the spheres, the potential becomes a horizontal segment. This is because the field inside a conductor is zero, and so the potential is constant, since $$E = -\diff{V}{r} = 0$$
In this case, the potential due to the sphere on the other side no longer matters, because any contribution to the electric field from external charges (e.g., right sphere) is canceled out by the redistribution of charge on the surface of the left sphere. This is a property of conductors, where external fields cannot penetrate inside.

\section{Magnetic Fields}

Magnetic field lines have similar properties to those of electric field lines:
\begin{itemize}
  \item North to south
  \item Higher field line density $\implies$ stronger field
  \item Field lines never cross
  \item Opposite poles attract, similar poles repel
\end{itemize}

\subsection{Field Line Patterns}

\subsubsection{Straight Wire}

Consider a straight current-carrying wire --- the field lines are concentric circles around the wire. The direction of the circular field lines can be determined using the right-hand grip rule. \textbf{This rule assumes conventional current flow, from positive to negative.}

\img{straightwire.jpg}{0.5}{Field Lines for a Straight Wire}{straightwire}

\subsubsection{Single Loop of Circular Coil}

\img{singleloop.png}{0.7}{Field Lines for a Single Loop}{singleloop}

The direction of the field lines can also be determined using the right-hand grip rule. This time, the fingers curl in the direction of the current and the thumb points in the direction of the field lines.\lb
On either sides of the coil, the field lines are in the opposite direction to the field lines running through the center.

To strengthen the field:
\begin{itemize}
  \item Increase current
  \item Add more turns to turn into a solenoid
\end{itemize}

\pagebreak

\subsubsection{Solenoid}

\img{solenoid.jpg}{0.9}{Field Lines for a Solenoid}{solenoid}

Again, the right-hand rule determines the direction of the field. The current in the diagram is conventional current.\lb
To strengthen the field:
\begin{itemize}
  \item Increase the current
  \item Increase the number of turns
  \item Add an iron core
\end{itemize}
The shape of the field lines look similar to those of a bar magnet; the end of the solenoid from which the field lines emerge is the north pole, and vice versa.

\section{Exam Questions}

\subsection{Force and Field Strength Directions}

P and Q are two opposite point charges. The force F acting on P due to Q and the electric field strength E at P are shown.
\img{ex/estrengthdirection.png}{0.5}{Force and Field Strength Directions}{estrengthdirection}
Which diagram shows the force on Q due to P and the electric field strength at Q?
\img{ex/estrengthdirection_options.png}{0.8}{Options}{estrengthdirection}

\begin{enumerate}
  \item Firstly, since P and Q have opposite charges, there must be an attractive force between them, this means that while P points toward Q, Q also points toward P. This eliminates options C and D.
  \item Recall that the electric field strength is the force per unit charge experienced by a small positive test charge at that point. The fact that $E_P$ and $F_P$ are in opposite directions suggests that P has a negative charge. This means that Q must have a positive charge, and so the field lines must point toward P (since field lines go from positive to negative). This gives option B.
\end{enumerate}

\pagebreak
\subsection{Parallel Plates --- Work Done in Moving a Charge}

\begin{quote}
  Two parallel plates are a distance apart with a potential difference between them. A point charge moves from the negatively charged plate to the positively charged plate. The charge gains kinetic energy $W$. The distance between the plates is doubled and the potential difference between them is halved. What is the kinetic energy gained by an identical charge moving between these plates?
\end{quote}
\begin{enumerate}
  \item Here, the doubled distance is simply a distraction; the only thing that matters is the potential difference between the plates, since $$W = q\Delta V$$
\end{enumerate}

\end{document}