\documentclass[a4paper,12pt]{article}
\usepackage{setspace}
\doublespacing
\usepackage[backend=biber,style=apa]{biblatex}
\usepackage{sectsty}
\usepackage{siunitx}
\usepackage{graphicx}
\usepackage[a4paper, total={3in, 9in}, textwidth=16cm,bottom=1in,top=1.4in]{geometry}
\usepackage{xcolor}
\usepackage{amsmath}
\usepackage{esvect}
\usepackage{amsthm}
\usepackage{hyperref}
\usepackage{float}
\usepackage{amssymb}
\usepackage{outlines}
\usepackage{caption}
\usepackage{subcaption}
\usepackage{esdiff}
\usepackage{setspace}
\newtheorem{lemma}{Lemma}
\newtheorem{proposition}{Proposition}
\doublespacing
\newcommand{\RNum}[1]{\uppercase\expandafter{\romannumeral #1\relax}}
\let\oldsi\si
\renewcommand{\si}[1]{\oldsi[per-mode=reciprocal-positive-first]{#1}}
\usepackage{enumitem}
\newcommand{\subtitle}[1]{%
  \posttitle{%
    \par\end{center}
    \begin{center}\large#1\end{center}
    \vskip0.5em}%
}
\newcommand{\degsym}{^{\circ}}
\newcommand{\Mod}[1]{\ (\mathrm{mod}\ #1)}
\usepackage{hyperref}
\hypersetup{
  colorlinks,
  citecolor=black,
  filecolor=black,
  linkcolor=black,
  urlcolor=black
}
\newcommand{\lb}{\\[8pt]}
\newenvironment*{cell}[1][]{\begin{tabular}[c]{@{}c@{}}}{\end{tabular}}
\newcommand{\img}[4]{\begin{center}
  \begin{figure}[H]
    \centering
    \includegraphics[width=#2\textwidth]{#1}
    \caption{#3}
    \label{fig:#4}
  \end{figure}
\end{center}}
\newcommand{\doubleimg}[4]{\begin{center}
  \begin{figure}[H]
    \centering
    \begin{subfigure}{.45\textwidth}
      \centering
      \includegraphics[width=1\linewidth]{#1}
      \caption{#2}
      \label{fig:sub1}
    \end{subfigure}
    \begin{subfigure}{.45\textwidth}
      \centering
      \includegraphics[width=1\linewidth]{#3}
      \caption{#4}
      \label{fig:sub2}
    \end{subfigure}
  \end{figure}
\end{center}}
\usepackage{fancyhdr}
\fancyfoot{}
\newcommand{\vect}[3]{\begin{bmatrix}
  #1 \\
  #2 \\
  #3
\end{bmatrix}}
\fancypagestyle{fancy}{\fancyfoot[R]{\vspace*{1.5\baselineskip}\thepage}}
\renewcommand{\contentsname}{Table of Contents}
\newcommand{\angled}[1]{\langle{#1}\rangle}
\newcommand{\paren}[1]{\left(#1\right)}
\newcommand{\sqb}[1]{\left[#1\right]}
\newcommand{\coord}[3]{\angled{#1,\, #2,\, #3}}
\newcommand{\pair}[2]{\paren{#1,\, #2}}
\usepackage{cleveref}
\crefname{lemma}{Lemma}{Lemmas}
\crefname{proposition}{Proposition}{Propositions}

\begin{document}


\pagenumbering{arabic}
\pagestyle{fancy}


\begin{titlepage}
  \begin{center}
    \vspace*{3cm}

    \textbf{\Large  {D2}}

    \vspace{1cm}


    \vfill

    \vspace{1.5cm}

  \end{center}
\end{titlepage}
\pagebreak
\tableofcontents
\pagebreak

\clearpage
\setcounter{page}{1}
\addtocontents{toc}{\protect\thispagestyle{empty}}

\pagebreak

\section{Conservation of Charge}

The sum of currents into a junction is equal to the sum of the currents away from the junction.

\section{Electrostatics}

\subsection{Charge by Friction}
\begin{itemize}
  \item Two materials charged by friction end up with opposite charges.
  \item Friction, rubbing; electrons move
\end{itemize}
\subsection{Charge by Contact}
Contact between a charge and an uncharged object. Then, both have the same sign.
\subsection{Charge by Induction}

Opposite charges are induced in a neutral object by bringing a charged object close to it. Will have opposite signs after. Electrons don't move from one object to another.

\section{Forces Between Charged Objects}

For radial fields only:

$$F=\frac{kq_1q_2}{r^2} = \frac{q_1q_2}{4\pi\epsilon_0 r^2}$$
where $k=8.99\times 10^9 \si{\N\m\squared\per\C\squared}$, $\epsilon$ is the permittivity of free space.
Also $$k = \frac{1}{4\pi\epsilon_0}$$ and $F_e = qE$

\section{Electric Field Strength}
The electric field strength $E$ at a distance $r$ from an isolated point charge $q$ is $$E=\frac{F}{q_2}=k\frac{q_1}{r^2}=\frac{q_1}{4\pi\epsilon_0 r^2}$$

\subsection{Electric Fields Lines}

\begin{itemize}
  \item The lines start and end on charges of opposite sign.
  \item An arrow is essential to show the direction in which a positive charge would move (that is, away from positive charge and towards negative charge).
  \item Where the field is strong, the lines are close together
  \item The lines act to repel each other
  \item The lines never cross
  \item The lines meet a conducting surface at 90$\degsym$
\end{itemize}

\subsection{P.d. Between Parallel Plates}

$$V = \frac{\text{work done in a moving charge}}{\text{magnitude of a charge}} = \frac{W}{q} = Ed$$
where $d$ is the separation between the plates. Also
$$E = \frac{V}{d}$$

\subsection{The Electronvolt}

$$1\si{\eV} = 1.6 \times 10^{-19} \si{\J}$$
$$1\si{\J} = 6.24\si{\eV}$$

\subsection{Field close to a conductor}

$$q\propto \frac{VA}{d}$$
$$q=\frac{VA}{4\pi kd}$$
$$E = 4\pi k \sigma \text{ for two plates}$$
$$E = 2\pi k \sigma \text{ for one plate}$$
where
\begin{itemize}
  \item $d = $ plate separation
  \item $V  = $ p.d. between plates
  \item $A = $ area of the plates
  \item $\sigma = $ surface charge density
  \item $k =\frac{1}{4\pi \epsilon_0} $

\end{itemize}

\subsection{Two Wires}

$$\frac{F}{L} = \mu_0\frac{I_1I_2}{2\pi r}$$

\end{document}