\documentclass[a4paper,12pt]{article}
\usepackage{setspace}
\usepackage{sectsty}
\usepackage{siunitx}
\usepackage{graphicx}
\usepackage[a4paper, total={3in, 9in}, textwidth=16cm,bottom=1in,top=1.4in]{geometry}
\usepackage[table,dvipsnames]{xcolor}
\usepackage{amsmath}
\usepackage{esvect}
\usepackage{soul}
\usepackage{amsthm}
\usepackage{hyperref}
\usepackage{float}
\usepackage{amssymb}
\usepackage{outlines}
\usepackage{caption}
\usepackage{fancyvrb}
\usepackage{subcaption}
\usepackage{esdiff}
\usepackage{setspace}
\usepackage{mathtools}
\usepackage{tikz,pgfplots}
\usepackage{dirtytalk}
\usepackage{draftwatermark}
\usepackage{helvet}
\renewcommand{\familydefault}{\sfdefault}
\usepackage[most]{tcolorbox}
\SetWatermarkText{timthedev07}
\SetWatermarkScale{4}
\SetWatermarkColor[gray]{0.97}
\usetikzlibrary{positioning,decorations.markings,calc}
\DeclarePairedDelimiter{\ceil}{\lceil}{\rceil}
\newtheorem{lemma}{Lemma}
\newtheorem{proposition}{Proposition}
\newtheorem{remark}{Remark}
\newtheorem{observation}{Observation}
\doublespacing
\let\oldsection\section
\renewcommand\section{\clearpage\oldsection}
\newcommand{\RNum}[1]{\uppercase\expandafter{\romannumeral #1\relax}}
\let\oldsi\si
\renewcommand{\si}[1]{\oldsi[per-mode=reciprocal-positive-first]{#1}}
\usepackage{enumitem}
\newcommand{\subtitle}[1]{%
  \posttitle{%
    \par\end{center}
    \begin{center}\large#1\end{center}
    \vskip0.5em}%
}
\newcommand{\degsym}{^{\circ}}
\newcommand{\Mod}[1]{\ (\mathrm{mod}\ #1)}
\usepackage{hyperref}
\hypersetup{
  colorlinks=true,
  linkcolor = blue
}
\newcommand{\lb}{\\[8pt]}
\newenvironment*{cell}[1][]{\begin{tabular}[c]{@{}c@{}}}{\end{tabular}}
\newcommand{\img}[4]{\begin{center}
  \begin{figure}[H]
    \centering
    \includegraphics[width=#2\textwidth]{#1}
    \caption{#3}
    \label{fig:#4}
  \end{figure}
\end{center}}
\parindent=0pt
\usepackage{fancyhdr}
\fancyfoot{}
\newcommand{\vect}[3]{\begin{bmatrix}
  #1 \\
  #2 \\
  #3
\end{bmatrix}}
\fancypagestyle{fancy}{\fancyfoot[R]{\vspace*{1.5\baselineskip}\thepage}}
\renewcommand{\contentsname}{Table of Contents}
\newcommand{\angled}[1]{\langle{#1}\rangle}
\newcommand{\paren}[1]{\left(#1\right)}
\newcommand{\sqb}[1]{\left[#1\right]}
\newcommand{\coord}[3]{\angled{#1,\, #2,\, #3}}
\newcommand{\pair}[2]{\paren{#1,\, #2}}
\usepackage[
  noabbrev,
  capitalise,
  nameinlink,
]{cleveref}
\crefname{lemma}{Lemma}{Lemmas}
\crefname{proposition}{Proposition}{Propositions}
\crefname{remark}{Remark}{Remarks}
\crefname{observation}{Observation}{Observations}

\newtcolorbox[auto counter]{defin}[1][]{fonttitle=\bfseries, title=\strut Definition.~\thetcbcounter,colback=black!5!white,colframe=black!65!gray,top=5mm,bottom=5mm}

\newtcolorbox[auto counter]{obs}[1][]{fonttitle=\bfseries, title=\strut Observation.~\thetcbcounter,colback=RedViolet!5!white,colframe=RedViolet!65!gray,top=5mm,bottom=5mm}

\setlength{\belowcaptionskip}{-20pt}

\begin{document}


\pagenumbering{arabic}
\pagestyle{fancy}


\begin{titlepage}
  \begin{center}

    \vspace*{8cm}
    \textbf{\Large {IB Physics Topic C2 The Wave Model; SL \& HL}} \\
    \vspace*{1cm}
    \large{By timthedev07, M25 Cohort}


  \end{center}
\end{titlepage}

\pagebreak
\tableofcontents
\pagebreak

\clearpage
\setcounter{page}{1}
\addtocontents{toc}{\protect\thispagestyle{empty}}

\section{Parts and Attributes of a Wave}

\begin{minipage}{0.45\textwidth}
  \img{transverse1.png}{1}{Transverse wave}{transverse1}
\end{minipage}\hspace*{0.1\textwidth}%
\begin{minipage}{0.45\textwidth}
  \img{longitudinal1.png}{1}{Longitudinal wave}{longitudinal1}
\end{minipage}

A wave has the following attributes:
\begin{itemize}
  \item Amplitude $x_0$: maximum displacement from the equilibrium position.
  \item Frequency $f$: number of oscillations per unit time.
  \item Time period $T$: time taken for one complete oscillation.
  \item Wavelength $\lambda$: distance between two consecutive points in phase.
  \item Wave speed $v$: speed at which the wave propagates.
\end{itemize}
These properties are linked by the following equation
\begin{align*}
  v = \lambda f = \frac{\lambda}{T} \\
  fT = 1
\end{align*}

\subsection{Standing vs Traveling Waves}

\begin{itemize}
  \item Traveling waves transfer \hl{energy} whereas standing waves don't.
  \item The \hl{amplitude} of oscillation varies along a standing wave, but is constant along a travelling wave.
  \item Standing waves have \hl{nodes and antinodes}, while travelling waves do not.
  \item Points in an internodal region have same phase in standing waves, but different phase in travelling waves
\end{itemize}

\section{Graphing Waves}

Transverse waves and longitudinal waves differ in their directions of energy transfer/particle movement:
\begin{itemize}
  \item Transverse waves: perpendicular to the direction of wave propagation.
  \item Longitudinal waves: parallel to the direction of wave propagation.
\end{itemize}

\subsection{Displacement-Time Graphs}

\img{distancetime.png}{0.9}{Displacement-time graph}{distancetime}

This provides information about the movement of one particular particle on the wave, but it is also the same for every particle on the wave.


\subsection{Displacement-Distance Graphs}

\subsubsection{Transverse Waves}

\img{dd_transverse.png}{0.9}{Displacement-distance graph for transverse waves}{dd_transverse}

\begin{itemize}
  \item This can be seen as a snapshot of the plot of the position of particles on the wave at a given instant.
  \item What cannot be inferred from the graph is the time period or frequency of the wave unless the wave speed is given.
\end{itemize}

\pagebreak

\subsubsection{Longitudinal Waves}

Recall that particles in a longitudinal wave essentially move left and right.

\img{dd_long.png}{0.9}{Displacement-distance graph for longitudinal waves}{dd_long}
\hl{The second row reflects the actual position of the particles at this particular instant}. Here is how the plot of them is obtained:
\begin{itemize}
  \item For each point, find its corresponding point on the curve below.
  \item If the point is below the x-axis, the curve would have been displaced in the negative direction, that is, on the distance $x$ axis, a movement to the left.
        \begin{itemize}
          \item  Take the second point in the top row as an example
          \item  Its corresponding point on the curve below is at $-0.5$.
          \item This means that it is displaced by $0.5$ units to the left, horizontally.
        \end{itemize}
  \item If the point is above the x-axis, the curve would have been displaced in the positive direction, that is, on the distance $x$ axis, a movement to the right.
\end{itemize}

\pagebreak

As a general rule of thumb:
\begin{itemize}
  \item The centers of rarefraction and compression are at the \hl{roots of the curve}.
        \begin{itemize}
          \item If gradient at a root is positive (increasing), then, it is a center of rarefraction.
                \img{rarefraction.png}{0.4}{Center of rarefraction}{rarefraction}
          \item Inversely, if the gradient at a root is negative (decreasing), then, it is a center of compression.
                \img{compression.png}{0.4}{Center of compression}{compression}
        \end{itemize}
\end{itemize}
More observations:
\begin{itemize}
  \item Neither rarefraction nor compression occurs at the peaks.
\end{itemize}
\pagebreak

\img{4.png}{0.7}{A sequence of snapshots}{4}

\section{Mechanical Waves}

Mechanical waves can be both transverse and longitudinal. They require a medium to propagate through.
\begin{itemize}
  \item Gases and liquids cannot support transverse waves due to the lack of a restoring force, but solids can transmit both wave types due to strong atomic bonds.
\end{itemize}
\img{sound.png}{0.6}{Sound wave}{sound}
Sound from a smartphone loudspeaker creates alternating high- and low-pressure regions in the air, which travel as a longitudinal wave to the listener's ear. These pressure variations correspond to compressions and rarefactions in the wave. As the wave spreads out, its amplitude decreases due to energy loss, partly from heating the air.\lb
Each adjacent pair of compressions and rarefactions is a quarter of a wavelength away, or $\frac{\pi}{2}$ out of phase.

\section{E.M. Waves}

All E.M. waves share the following properties:
\begin{itemize}
  \item Are transverse
  \item Can travel through a vacuum
  \item Travel at the speed of light $c = 3.00 \times 10^8 \si{\meter\per\second}$ in a vacuum
  \item Consist of oscillating electric and magnetic fields perpendicular to each other and the direction of wave propagation
        \img{em.png}{0.6}{E.M. wave}{em}
  \item They are generated when accelerated electrons or other charged particles emit photons due to energy changes.
  \item Like all waves, they have frequency and wavelength; however, only the wavelength changes in different media, while frequency remains constant.
\end{itemize}

\pagebreak

\subsection{The E.M. Spectrum}

It's very important to memorize the rough wavelength ranges of each type of E.M. wave in the spectrum.
\begin{table}[h]
  \centering
  \renewcommand{\arraystretch}{2}
  \begin{tabular}{|c|c|c|}
    \hline
    \textbf{Electromagnetic Waves} & \textbf{Range of Frequency (Hz)}                    & \textbf{Range of Wavelength}                \\
    \hline
    Gamma-rays                     & \(10^{20} \text{ to } 10^{24}\)                     & \(< 10^{-12} \, m\)                         \\
    \hline
    X-rays                         & \(10^{17} \text{ to } 10^{20}\)                     & \(1 \text{ nm to } 1 \text{ pm}\)           \\
    \hline
    Ultraviolet                    & \(10^{15} \text{ to } 10^{17}\)                     & \(400 \text{ nm to } 1 \text{ nm}\)         \\
    \hline
    Visible                        & \(4 \times 10^{14} \text{ to } 7.5 \times 10^{14}\) & \(750 \text{ nm to } 400 \text{ nm}\)       \\
    \hline
    Near-infrared                  & \(10^{14} \text{ to } 4 \times 10^{14}\)            & \(2.5 \, \mu m \text{ to } 750 \text{ nm}\) \\
    \hline
    Infrared                       & \(10^{13} \text{ to } 10^{14}\)                     & \(25 \, \mu m \text{ to } 2.5 \, \mu m\)    \\
    \hline
    Microwaves                     & \(3 \times 10^{11} \text{ to } 10^{13}\)            & \(1 \text{ mm to } 25 \, \mu m\)            \\
    \hline
    Radio waves                    & \(< 3 \times 10^{11}\)                              & \(> 1 \text{ mm}\)                          \\
    \hline
  \end{tabular}
  \caption{Electromagnetic Spectrum Table}
  \label{tab:em_spectrum}
\end{table}

\pagebreak

\section{Exam Questions}

\subsection{Displacement-Distance Graphs}

\img{ex/1.png}{0.7}{Displacement-distance graph}{ex1}

\img{ex/2.png}{0.7}{Options}{ex2}

\begin{itemize}
  \item The answer is A, if you consider the graph given in the question an instant later, the wave is to be shifted to the right and so the y-position of at P will be higher up. This implies a positive displacement about to occur.
  \item Hence, we see that only the first option is correct.
\end{itemize}

\pagebreak

\subsection{Wavelength from Compression-Rarefraction Graphs}

\img{ex/3.png}{0.7}{Compression-Rarefraction graph}{ex3}

As you can see, I am just too lazy to even copy the text and format it nicely. But who cares? I am burnt out mate. Right, back to the question
\begin{itemize}
  \item The wavelength is the distance between two consecutive centers of compression or rarefraction.
  \item In this case, we can spot that 0 and 8 are the two centers of rarefraction. Hence, the \textcolor{ForestGreen}{answer is D}.
\end{itemize}

\end{document}